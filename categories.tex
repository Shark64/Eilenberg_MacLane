\documentclass[11pt,a4paper]{report}
\usepackage[nonamelimits]{amsmath}
\usepackage{amssymb,amsthm}
\usepackage[german,french,english]{babel}
\usepackage{unnumberedtotoc}
\usepackage{csquotes}
\usepackage[style=authortitle-icomp,dashed=false]{biblatex}
\usepackage{lmodern}
\usepackage[T1]{fontenc}
\usepackage{tikz}
\usetikzlibrary{cd}
\usetikzlibrary{babel}
\usepackage[tracking=true,spacing=true,kerning=true]{microtype}
\usepackage{mathtools}
\usepackage[prevent-all]{widows-and-orphans}
\usepackage{hyphenat}
\usepackage{hyperref}
\usepackage[capitalise]{cleveref}

\newtheorem*{thdef}{Definition}
\newtheorem{axiom}{Axiom}
\newtheorem{lemma}{Lemma}
\newtheorem{theorem}{Theorem}
\newcommand{\R}{\mathbb{R}}
\DeclareMathOperator{\map}{Map}
\DeclareMathOperator{\hm}{Hom}
\DeclareMathOperator{\lin}{Lin}
\DeclareMathOperator{\Lim}{Lim}
\DeclareMathOperator{\dual}{Conj}
\DeclareMathOperator{\ch}{Ch}
\DeclareMathOperator{\chr}{Char}
\DeclareMathOperator{\an}{Annih}
\DeclareMathOperator{\ki}{Ki}
\DeclareMathOperator{\fc}{Fact}
\DeclareMathOperator{\tra}{Trans}
\DeclareMathOperator{\ex}{Ext}
\hypersetup{pageanchor=false}
\addbibresource{biblio.bib}
\crefname{case}{case}{cases}
\crefname{cond}{condition}{conditions}
\crefname{axiom}{axiom}{axioms}
\crefformat{section}{\S#2#1#3} %mimick the original format.
\frenchspacing
\date{15 May 1945}


\begin{document}
\author{Samuel Eilenberg \and Saunders MacLane}
\title{General Theory of Natural Equivalences}
\maketitle
\hypersetup{pageanchor=true}
\tableofcontents

\addchap{Introduction}\label{intro}

The subject matter of this paper is best explained by an example, such
as that of the relation between a vector space $L$ and its ``dual'' or
``conjugate'' space $T(L)$. Let $L$ be a finite\hyp{}dimensional real vector
space, while its conjugate $T(L)$ is, as is customary, the vector space
of all real valued linear functions $t$ on $L$. Since this conjugate $T(L)$
is in its turn a real vector space with the same dimension as $L$, it is
clear that $L$ and $T(L)$ are isomorphic. But such an isomorphism cannot
be exhibited until one choose a definite set of basis vectors for $L$,
and furthermore the isomorphism which results will differ for different
choices of this basis.

For the iterated conjugate space $T(T(L))$, on the other hand, it is well
known that one can exhibit an isomorphism between $L$ and $T(T(L))$ 
\emph{without} using any special basis in $L$. This exhibition of the
isomorphism $L\cong T(T(L))$ is ``natural'' in that it is given 
\emph{simultaneously} for \emph{all} finite\hyp{}dimensional vector spaces $L$.

This simultaneity can be further analyzed. Consider two finite\hyp{}dimensional
vector spaces $L_1$ and $L_2$ and a linear transformation $\lambda_1$ of
$L_1$ into $L_2$; in symbols
\begin{equation}\label{eq:trans}
	\lambda_1:L_1 \to L_2
\end{equation}
This transformation $\lambda_1$ induces a corresponding linear transformation
of the second conjugate space $T(L_2)$ into the first one, $T(L_1)$. 
Specifically, since each element $t_2$ in the conjugate space $T(L_2)$ is
itself a mapping, one has two transformations
\begin{equation*}
	L_1 \xlongrightarrow{\lambda_1} L_2 \xlongrightarrow{t_2} \R;
\end{equation*}
their product $t_2\lambda_1$ is thus a linear transformation of $L_1$ into
$\R$, hence an element $t_1$ in the conjugate space $T(L_1)$. We call this
correspondence of $t_2$ to $t_1$ the mapping $T(\lambda_1)$ \emph{induced}
by $\lambda_1$; thus $T(\lambda_1)$ is defined by setting 
$[T(\lambda_1)]t_2= t_2\lambda_1$, so that
\begin{equation}\label{eq:induced}
	T(\lambda_1): T(L_2)\to T(L_1).
\end{equation}
In particular, this induced transformation $T(\lambda_1)$ is simply the
identity when $\lambda_1$ is given as the identity transformation of
$L_1$ into $L_1$. Furthermore the transformation induced by a product
of $\lambda$'s is the product of the separately induced transformations,
for if $\lambda_1$ maps $L_1$ into $L_2$ while $\lambda_2$ maps $L_2$
into $L_3$, the definition of $T(\lambda)$ shows that
\begin{equation*}
	T(\lambda_2\lambda_1)=T(\lambda_1)T(\lambda_2).
\end{equation*} 
The process of forming the conjugate space thus actually
involves two different operations or functions. The first associates with
each space $L$ its conjugate space $T(L)$; the second associates with each
linear transformation $\lambda$ between vector spaces its induced linear
transformation $T(\lambda)$.\footnote{The two different functions $T(L)$
and $T(\lambda)$ may be safely denoted by the same letter $T$ because 
their arguments $L$ and $\lambda$ are always typographically distinct.}

A discussion of the ``simultaneous'' or ``natural'' character of the
isomorphism $L\cong T(T(L))$ clearly involves a simultaneous consideration
of all spaces $L$ and all transformations $\lambda$ connecting them; this
entails a simultaneous consideration of the conjugate space $T(L)$ and the
induced transformations $T(\lambda)$ connecting them. Both functions $T(L)$
and $T(\lambda)$ are thus involved; we regard them as the component parts
of what we call a ``functor'' $T$. Since the induced mapping $T(\lambda_1)$
of \cref{eq:induced} reversed the direction of the original $\lambda_1$ of \cref{eq:trans},
this functor $T$ will be called ``contravariant''.

The simultaneous isomorphisms
\begin{equation*}
	\tau(L): L\rightleftarrows T(T(L))
\end{equation*}
compare two \textsl{co}variant functors; the first is the identity functor
$I$, composed of the two functions
\begin{equation*}
	I(L)=L,\qquad I(\lambda)=\lambda;
\end{equation*}
the second is the iterated conjugate functor $T^2$, with components
\begin{equation*}
	T^2(L)=T(T(L)),\qquad T^2(\lambda)=T(T(\lambda)).
\end{equation*}
For each $L$, $\tau(L)$ is constructed as follows. Each vector $x\in L$ and
each functional $t\in T(L)$ determine a real number $t(x)$. If in this 
expression $x$ is fixed while $t$ varies, we obtain a linear transformation
of $T(L)$ into $\R$, hence an element $y$ in the double conjugate space
$T^2(L)$. This mapping $\tau(L)$ of $x$ to $y$ ma also be defined formally
by setting $[[\tau(L)]x]t= t(x)$.

The connections between these isomorphisms $\tau(L)$ and the transformations
$\lambda : L_1 \to L_2$ may be displayed thus:
\begin{equation*}
	\begin{tikzpicture}
		\node (a) at (0,0)
		{ \begin{tikzcd}[sep=huge]
		L_1 \arrow[r,"\tau(L_1)"] \arrow[d, "I(\lambda)"] & T^2(L_1) 
		\arrow[d,"T^2(\lambda)"] \\
		L_2 \arrow[r,"\tau(L_2)"] & T^2(L_2)
		\end{tikzcd}};
	\end{tikzpicture}
\end{equation*}
The statement that the two possible paths from $L_1$ to $T^2(L_2)$ in this
diagram are in effect identical is what we shall call the ``naturality''
or ``simultaneity'' condition for $t$; explicitly, it reads
\begin{equation}\label{eq:nat1}
	\tau(L_2)I(\lambda)=T^2(\lambda)\tau(L_1).
\end{equation}
This equality can be verified from the above definitions of $t(L)$ and
$T(\lambda)$ by straightforward substitution. A function $t$ satisfying this
``naturality'' condition will be called a ``natural equivalence'' of the
functors $I$ and $T^2$.

On the other hand, the isomorphism of $L$ to its conjugate space $T(L)$ is a
comparison of the covariant functor $I$ with the contravariant functor $T$.
Suppose that we are given simultaneous isomorphisms
\begin{equation*}
	\sigma(L): L\rightleftarrows T(L)
\end{equation*}
for each $L$. For each linear transformation $\lambda:L_1 \rightarrow L_2$ we
than have a diagram
\begin{equation*}
	\begin{tikzpicture}
		\node (a) at (0,0)
		{ \begin{tikzcd}[sep=huge]
		L_1 \arrow[r,"\sigma(L_1)"] \arrow[d, "I(\lambda)"] & T^2(L_1)\\
				L_2 \arrow[r,"\sigma(L_2)"] & T^2(L_2)\arrow[u,"T(\lambda)"]
		\end{tikzcd}};
	\end{tikzpicture}
\end{equation*}
The only ``naturality'' condition read from this diagram is $\sigma(L_1)\!=\!T(\lambda)\sigma(L_2)\lambda$. %negative space to avoid overfull box.
Since $\sigma(L_1)$ is an isomorphism, this condition certainly cannot hold unless $\lambda$ is an
isomorphism of $L_1$ into $L_2$. Even in the more restricted case in which $L_2=L_1=L$ is a single space
there can be no isomorphism $\sigma:L\rightarrow T(L)$ which satisfies this naturality condition
$\sigma=T(\lambda)\sigma\lambda$ for every non\hyp{}singular linear transformation $\lambda$.\footnote{For suppose $\sigma$
has this property. Then $(x,y)=[\sigma(x)]y$ is a non-singular bilinear form (not necessarily symmetric) in the
vectors $x,y$ of $L$, and we would have, for every $\lambda$,
$(x,y)=[\sigma(x)](y)=[T(\lambda)\sigma\lambda x]y=[\sigma\lambda x]\lambda y=(\lambda x,\lambda y)$,
so that the bilinear form is left invariant by every nonsingular linear transformation $\lambda$.
This is clearly impossible.} Consequently, with our definition of $T(\lambda)$, there is no ``natural''
isomorphism between the functors $I$ and $T$, even in a very restricted special case.

Such a consideration of vector spaces and their linear transformations is but one example of many similar 
mathematical situations; for instance, we may deal with groups and their homomorphisms, with topological
spaces and their continuous mappings, with simplicial complexes and their simplicial transformations, with
ordered sets and their order preserving transformations. In order to deal in a general way with such situations,
we introduce the concept of a \emph{category}. Thus a category $\mathbf{A}$ will consist of abstract elements
of two types: the objects $A$ (for example, vector spaces, groups) and the mappings $\alpha$ (for example, linear
transformations, homomorphisms). For some pairs of mappings in the category there is defined a product (in the 
examples, the product is the usual composite of two transformations). Certain of these mappings act as identities with
respect to this product, and there is a  one\hyp{}to\hyp{}one correspondence between the objects of the category and these identities.
A category is subject to certain simple axioms, so formulated as to include all examples of the character described above.

Some of the mappings $\alpha$ of a category will have a formal inverse mapping in the category; such a mapping $\alpha$ is called
and equivalence. In the examples quoted the equivalences turn out to be, respectively, the isomorphisms for vector spaces,
the homeomorphisms for topological spaces, the isomorphism for groups and complexes, and so on.

Most of the standard constructions of a new mathematical object from given objects (such as the construction of the
direct product of two groups, the homology group of a complex, the Galois group of a field) furnish a function 
$T(A,B,\dotsc)=C$ which assigns to given objects $A,B,\dotsc$ in definite categories $\mathbf{A,B},\dotsc$ a new object
$C$ in a category $\mathbf{C}$. As in the special case of the conjugate $T(L)$ of a linear space, where there is a
corresponding induced mapping $T(\lambda)$, we usually find that mappings $\alpha,\beta,\dotsc$ in the categories
$\mathbf{A,B,\dotsc}$ also induce a definite mapping $T(\alpha,\beta,\dotsc)=\gamma$ in the category $\mathbf{C}$,
properly acting on the object $T(A,B,\dotsc)$.

These examples suggest the general concept of a functor $T$ on categories $\mathbf{A,B,\dotsc}$ to a category $\mathbf{C}$,
defined as an appropriate pair of functions $T(A,B,\dotsc),T(\alpha,\beta,\dotsc)$. Such a functor may well be covariant in
some of its arguments, contravariant in the others. The theory of categories and functors, with a few of the illustrations,
constitutes \cref{ch:cat_funct}.

The natural isomorphism $L\rightarrow T^2(L)$ is but one example of many natural equivalences occurring in mathematics.
For instance, the isomorphism of a locally compact abelian group with its twice iterated character group, most of the
general isomorphisms in group theory and in the homology theory of complexes and spaces, as well as many equivalences in
set theory and in general topology satisfy a naturality condition resembling \cref{eq:nat1}.
In \cref{ch:nat_equ}, we provide a general definition of equivalence between functors which includes there cases.
A more general notion of a transformation of one functor into another provides a means of comparing functors which may not
be equivalent. The general concepts are illustrated by several fairly elementary examples of equivalences and transformations
for topological spaces, groups, and Banach spaces.

The third chapter deals especially with groups. In the category of groups the concept of a subgroup establishes a natural
partial order for the objects (groups) of the category. For a functor whose values are in the category of groups there is
an induced partial order. The formation of a quotient group has as analogue the construction of the quotient functor of a
given functor by any normal subfunctor. In the uses of group theory, most groups constructed are obtained as quotient groups
of other groups; consequently the operation of building a quotient functor is directly helpful in the representation of such
group constructions by functors. The first and second isomorphism theorems of group theory are then formulated for functors;
incidentally, this is used to show that these isomorphisms are ``natural''. The latter part of the chapter establishes the 
naturality of various known isomorphisms and homomorphisms in group theory.\footnote{A brief discussion of this case and of 
the general theory of functors in the case of groups is given in the authors' note,~\cite{groups42}.\label{ft:groups42}}

The fourth chapter starts with a discussion of functors on the category of partially ordered sets, and continues with the
discussion of limits of direct and inverse systems of groups, which form the chief topic of this chapter. After suitable
categories are introduced, the operations of forming direct and inverse limits of systems of groups are described as functors.

In the fifth chapter we establish the homology and cohomology groups of complexes and spaces as functors and show the 
naturality of various know isomorphisms of topology, especially those which arise in duality theorems. The treatment of
the \u{C}ech homology theory utilizes the categories of direct and inverse systems, as discussed in \cref{ch:poset}.

The introduction of this study of naturality is justified, in our opinion, both by its technical and by its conceptual
advantages.

In the technical sense, it provides the exact hypotheses necessary to apply to both sides of an isomorphism a passage
to the limit, in the sense of direct or inverse limits for groups, rings or spaces.\footnote{Such limiting processes
are essential in the transition from the homology theory of complexes to that of spaces. Indeed, the general theory
developed here occurred to the authors as a result of the study of the admissibility of such a passage in a relatively
involved theorem in homology theory (\cite[especially, p.~777 and p.~815]{eilenberg42}).} Indeed, our naturality
condition is part of the standard isomorphism condition for two direct or two inverse systems.\footcite{freud37}

The study of functors also provides a technical background for the intuitive notion of naturality and makes it
possible to verify by straightforward computation the naturality of an isomorphism or of an equivalence in
all those cases where it has intuitively recognized that the isomorphisms are indeed ``natural''. In many cases
(for example, as in the above isomorphism of $L$ to $T(L)$) we can also assert that certain known isomorphisms are
in fact ``unnatural'', relative to the class of mappings considered.

In metamathematical sense our theory provides general concepts applicable to all branches of abstract mathematics,
and so contributes to the current trend towards uniform treatment of different mathematical disciplines. In particular,
it provides opportunities for the comparison of constructions and of the isomorphisms occurring in different branches
of mathematics; in this way it may occasionally suggest new results by analogy.

The theory also emphasizes that, whenever new abstract objects are constructed in a specified way out of given ones,
it is advisable to regard the construction of the corresponding induced mappings on these new objects an an integral
part of their definition. The pursuit of this program entails a simultaneous consideration of objects and their
mappings (in our terminology, this means the consideration not of individual objects but of categories). This 
emphasis on the specification of the type of mappings employed gives more insight int the degree of invariance of 
the various concepts involved. For instance, we show in \cref{sec:ex_funct}, that the concept of the commutator
subgroup of a group is in a sense a more invariant one than that of the center, which in its turn is more invariant
than the concept of the automorphism group of a group, even though in the classical sense all three concepts are invariant.

The invariant character of a mathematical discipline can be formulated in these terms. Thus, in group theory all
basic constructions can be regarded as the definitions of co- or contravariant functors, so we may formulate the
dictum: The subject of group theory is essentially the study of those constructions of groups which behave in a
covariant or contravariant manner under induced homomorphisms. More precisely, group theory studies functors
defined on well specified categories of groups, with values in another such category.

This may be regarded as a continuation of the Klein Erlanger Programm, in the sense that a geometrical space with
its group of transformations is generalized to a category with its algebra of mappings.

\chapter{Categories and functors}\label{ch:cat_funct}
\section{Definition of Categories}\label{sec:def_cat}
These investigations will deal with aggregates such as a class of groups together with a class of homomorphisms,
each of which maps one of the groups into another one, or such as a class of topological spaces together with 
their continuous mappings, one into another. Consequently we introduce a notion of ``category'' which will
embody the common formal properties of such aggregates.

From the examples ``groups plus homomorphisms'' or ``spaces plus continuous mappings'' we are led to the following
definition. A \emph{category} $\mathbf{A}=\{A,\alpha\}$ is aggregate of abstract elements $A$ (for example, groups),
called the \emph{objects} of the category, and abstract elements $\alpha$ (for example, homomorphisms), called 
\emph{mappings} of the category. Certain pair of mappings $\alpha_1,\alpha_2\in\mathbf{A}$ determine uniquely a
product mapping $\alpha=\alpha_2\alpha_1\in\mathbf{A}$, subject to \crefrange{ax:c1}{ax:c3} below. Corresponding
to each object $A\in\mathbf{A}$ there is a unique mapping, denoted by $e_A$ or by $e(A)$, and subject to the
\cref{ax:c4,ax:c5}. The axioms are:
\theoremstyle{definition}
\begin{axiom}\label[axiom]{ax:c1}
	The triple product $\alpha_3(\alpha_2\alpha_1)$ is defined if and only if $(\alpha_3\alpha_2)\alpha_1$
	is defined. When either is defined, the associative law
	\begin{displaymath}
		\alpha_3(\alpha_2\alpha_1)=(\alpha_3\alpha_2)\alpha_1
        \end{displaymath}
	holds. This triple product will be written as $\alpha_3\alpha_2\alpha_1$.
\end{axiom}

\begin{axiom}\label[axiom]{ax:c2}
	The triple product $\alpha_3\alpha_2\alpha_1$ is defined whenever both products $\alpha_3\alpha_2$ and
	$\alpha_2\alpha_1$ are defined.
\end{axiom}
\begin{thdef}
	A mapping $e\in\mathbf{A}$ will be called an \emph{identity} of $\mathbf{A}$ if and only if the
	existence of any product $e\alpha$ or $\beta e$ implies that $e\alpha=\alpha$ or $\beta e =\beta$.
\end{thdef}
\begin{axiom}\label[axiom]{ax:c3}
	For each mapping $\alpha\in\mathbf{A}$ there is at least on identity $e_1\in\mathbf{A}$ such that
	$\alpha e_1$ is defined, and at least on identity $e_2\in\mathbf{A}$ such that $e_2\alpha$ is
	defined.
\end{axiom}
\begin{axiom}\label[axiom]{ax:c4}
	The mapping $e_A$ corresponding to each object $A$ is an identity.
\end{axiom}
\begin{axiom}\label[axiom]{ax:c5}
	For each identity $e$ of $\mathbf{A}$ there is an unique object $A$ of $\mathbf{A}$ such that $e_A = e$.
\end{axiom}
\theoremstyle{plain}
These two axioms assert that the rule $A \rightarrow e_A$ provides a one\hyp{}to\hyp{}one correspondence between the set of
all objects of the category and the set of all its identities. 

\begin{lemma}
	For each mapping $\alpha\in\mathbf{A}$ there is exactly one object $A_1$ with the product $\alpha e(A_1)$
	defined, and exactly one $A_2$ with $e(A_2)\alpha$ defined.
\end{lemma}

The objects $A_1,A_2$ will be called the \emph{domain} and the \emph{range} of $\alpha$, respectively. We also say
that $\alpha$ acts on $A_1$ to $A_2$, and write
\begin{equation*}
	\alpha: A_1\rightarrow A_2 \text{ in }\mathbf{A}.
\end{equation*}
\begin{proof}
	Suppose that $\alpha e(A_1)$ and $\alpha e(B_1)$ are both defined. By the properties of an identity,
	$\alpha e(A_1)=\alpha$, so that \cref{ax:c1,ax:c2} insure that the product $e(A_1)e(B_1)$ is defined.
	Since both are identities, $e(A_1)=e(A_1)e(B_1)=e(B_1)$, and consequently $A_1=B_1$. The uniqueness of
	$A_2$ is similarly established.
\end{proof}
\begin{lemma}
	The product $\alpha_2\alpha_1$ is defined if and only if the range of $\alpha_1$ is the domain of
	$\alpha_2$. In other words, $\alpha_2\alpha_1$ is defined if and only if $\alpha_1 :A_1 \rightarrow
	A_2$ and $\alpha_2 :A_2 \rightarrow A_3$. In that case $\alpha_1\alpha_2:A_1\rightarrow A_3$.
\end{lemma}
\begin{proof}
	Let $\alpha_1:A_1\rightarrow A_2$. The product $e(A_2)\alpha_1$ is then defined and $e(A_2)\alpha_1=\alpha_1$.
	Consequently $\alpha_2\alpha_1$ is defined if and only if $\alpha_2 e(A_2)\alpha_1$ is defined. By \cref{ax:c1,ax:c2}
	this will hold precisely when $\alpha_2 e(A_2)$ is defined. Consequently $\alpha_2\alpha_1$ is 
	defined if and only if $A_2$ is the domain of $\alpha_2$ so that $\alpha_2 :A_2\rightarrow A_3$. To prove
	that $\alpha_2\alpha_1 :A_1 \rightarrow A_3$ note that since $\alpha_1 e(A_1)$ and $e(A_3)\alpha_2$ are defined
	the products $(\alpha_2\alpha_1)e(A_1)$ and $e(A_3)(\alpha_2\alpha_1)$ are defined.
\end{proof}
\begin{lemma}
	If $A$ is an object, $e_A:A\rightarrow A$.
\end{lemma}
\begin{proof}
	If we assume that $e(A):A_1\rightarrow A_2$ then $e(A)e(A_1)$ and $e(A_2)e(A)$ are defined. Since they are all
	identities it follows that $e(A)=e(A_1)e(A_2)$ and $A=A_1=A_2$
\end{proof}

A ``left identity'' $\beta$ is a mapping such that $\beta\alpha=\alpha$ whenever $\beta\alpha$ is defined. \Cref{ax:c3} shows
that every left identity is an identity. Similarly each right identity is an identity. Furthermore, the product $e e^1$ of
two identities is defined if and only if $e=e^1$.

If $\beta\gamma$ is defined and is an identity, $\beta$ is called a \emph{left inverse} of $\gamma$, $\gamma$ is a
\emph{right inverse} of $\beta$. A mapping $\alpha$ is called an \emph{equivalence} of $\mathbf{A}$ if it has in 
$\mathbf{A}$ at least one left inverse and at least one right inverse.
\begin{lemma}
	An equivalence $\alpha$ has exactly one left inverse and exactly one right inverse. These inverses are equal,
	so that the (unique) inverse may be denoted by $\alpha^{-1}$.
\end{lemma}
\begin{proof}
	It suffices to show that any left inverse $\beta$ of $\alpha$ equals any right inverse $\gamma$. Since $\beta\alpha$
	and $\alpha\gamma$ are both defined, $\beta\alpha\gamma$ is defined, by \cref{ax:c2}. But $\beta\alpha$ and $\alpha\gamma$
	are identities, so that $\beta=\beta(\alpha\gamma)=(\beta\alpha)\gamma=\gamma$, as asserted.
\end{proof}

For equivalences $\alpha,\beta$ one easily proves that $\alpha^{-1}$ and $\alpha\beta$ (if defined) are equivalences, and that
\begin{equation*}
	{(\alpha^{-1})}^{-1}=a,\qquad {(\alpha\beta)}^{-1}=\beta^{-1}\alpha^{-1}.
\end{equation*}
Every identity $e$ is an equivalence, with $e^{-1}=e$.

Two objects $A_1,A_2$ are called \emph{equivalent} if there is an equivalence $\alpha$ such that $\alpha:A_1\rightarrow A_2$.
The relation of equivalence between objects is reflexive, symmetric and transitive.

\section{Examples of Categories}\label{sec:ex_cat}
In the construction of examples, it is convenient to use the concept of a subcategory. A subaggregate $\mathbf{A}_0$ of
$\mathbf{A}$ will be called a \emph{subcategory} if the following conditions hold:
\begin{enumerate}
	\item If $\alpha_1,\alpha_2\in\mathbf{A}_0$ and $\alpha_2\alpha_1$ is defined in $\mathbf{A}$, then
		$\alpha_2,\alpha_1\in\mathbf{A}_0$.\label{sub_closed}
	\item If $A\in\mathbf{A}_0$, then $e_A\in\mathbf{A}_0$.\label{sub_id}
	\item If $\alpha:A_1\rightarrow A_2$ in $\mathbf{A}$ with $\alpha\in\mathbf{A}_0$,
		then $A_1,A_2\in\mathbf{A}_0$\label{sub_obj}
\end{enumerate}
Condition~\ref{sub_closed} insures that $\mathbf{A}_0$ is ``closed'' with respect to multiplication in $\mathbf{A}$;
from conditions~\ref{sub_id} and~\ref{sub_obj} it then follows that $\mathbf{A}_0$ is itself a category. The intersection
of any number of subcategories of $\mathbf{A}$ is again a subcategory of $\mathbf{A}$. Note, however, that an equivalence
$\alpha\in\mathbf{A}_0$ of $\mathbf{A}$ need not remain an equivalence in a subcategory $\mathbf{A}_0$, because the
inverse $\alpha^{-1}$ may not be in $\mathbf{A}_0$.

For example, if $\mathbf{A}$ is any category, the aggregate $\mathbf{A}_e$ of all the objects and all the equivalences
of $\mathbf{A}$ is a subcategory of $\mathbf{A}$. Also if  $\mathbf{A}$ is a category and $S$ a subclass of its objects,
the aggregate $\mathbf{A}_S$ consisting of all objects of $S$ and all mappings of $\mathbf{A}$ with both range and domain
in $S$ is a subcategory. In fact, every subcategory of $\mathbf{A}$  can be obtained in two steps: first, form a 
subcategory $\mathbf{A}_S$; second, extract from $\mathbf{A}_S$ a subaggregate, consisting of all the objects of 
$\mathbf{A}_S$ and a set of mappings of $\mathbf{A}_S$ which contains all the identities and is closed under multiplication.

The category $\mathbf{S}$ of all sets has as its objects all sets $S$.\footnote{This category obviously leads to the
paradoxes of set theory. A detailed discussion of this aspect of categories appears in \cref{sec:found}, below.}
A mapping $\sigma$ of $\mathbf{S}$ is determined by a pair of sets $S_1$ and $S_2$ and a many\hyp{}one correspondence between
$S_1$ and a subset of $S_2$, which assigns to each $x\in S_1$ a corresponding element $\sigma x\in S_2$; we then write
$\sigma : S_1\rightarrow S_2$. (Note that any deletion of elements from $S_1$ or $S_2$ \emph{changes} the mapping $\sigma$.)
The product of $\sigma_2: S_2^1\rightarrow S_3$ and $\sigma_1: S_1\rightarrow S_2$ is defined if and only if $S^1_2=S_2$;
this product then maps $S_1$ to $S_3$ by the usual composite correspondence $(\sigma_2\sigma_1)x=\sigma_2(\sigma_1 x)$,
for each $x\in S_1$.\footnote{This formal associative law allows us to write $\sigma_2\sigma_1 x$ without fear of ambiguity.
In more complicated formulas, parentheses will be inserted to make the components stand out.} The mapping $e_S$ corresponding
to the set $S$ is the identity mapping of $S$ onto itself, with $e_S x=x\text{ for }x\in S$. The \crefrange{ax:c1}{ax:c5}
are clearly satisfied. An equivalence $\sigma:S_1\rightarrow S_2$ is simply a one\hyp{}to\hyp{}one mapping of $S_1$ \emph{onto} $S_2$.

Subcategories of $\mathbf{S}$ include the category of all finite sets $S$, with their mappings as before. For any cardinal
number $m$ there are two similar categories, consisting of all sets $S$ of power less than $m$ (or, of power less than or 
equal to $m$), together with all their mappings. Subcategories of $\mathbf{S}$ can also be obtained by restricting the
mappings; for instance we may require that each $\sigma$ is a mapping of $S_1$ \emph{onto} $S_2$, or that each $\sigma$ is a
one\hyp{}to\hyp{}one mapping of $S_1$ into a subset of $S_2$.

The category $\mathbf{X}$ of all topological spaces has as its objects all topological spaces $X$ and as its mappings all
continuous transformations $\xi: X_1\rightarrow X_2$ of a space $X_1$ into a space $X_2$. The composition $\xi_2\xi_1$ and
the identity $e_X$ are both defined as before. An equivalence in $\mathbf{X}$ is a homeomorphism ($=$topological equivalence).

Various subcategories of $\mathbf{X}$ can again be obtained by restricting the type of topological space to be considered,
or by restricting the mappings, say to open mappings or to closed mappings.\footnote{A mapping $\xi:x_1\rightarrow x_2$
is \emph{open} (\emph{closed}) if the image under $\xi$ of every open (closed) subset of $X$ is open (closed) in $X_2$.}

In particular,  $\mathbf{S}$ can be regarded as a subcategory of $\mathbf{X}$, namely, as that subcategory consisting of
all spaces with a discrete topology.

The category $\mathbf{G}$ of all topological groups\footnote{A \emph{topological group} $G$ is a group which is also a
topological space in which the group composition and the group inverse are continuous functions (no separation axioms
are assumed on the space). If, in addition, $G$ is a Hausdorff space, then all the separation axioms up to and including
regularity are satisfied, so that we call $G$ a \emph{regular topological group}.} has as its objects all topological
group $G$ and as its mappings $\gamma$ all those many\hyp{}one correspondences of a group $G_1$ into a group $G_2$ which are
homomorphisms.\footnote{By homomorphism we always understand a continuous homomorphism.}

Subcategories of $\mathbf{G}$ can be obtained by restricting the groups (discrete, abelian, regular, compact and so on)
or by restricting the homomorphisms (open homomorphisms, homomorphisms ``onto'', and so on).

The category $\mathbf{B}$ of all Banach spaces is similar; its objects are the Banach spaces $B$, its mappings all linear
transformations $\beta$ of norm at most 1 of one Banach space into another.\footnote{For each linear transformation $\beta$
of the Banach space $B_1$ into $B_2$, the norm $\|\beta\|$ is defined as the least upper bound $\|\beta b\|$, for all
$b\in B_1$ with $\|b\|=1$.} Its equivalences between two Banach spaces (that is, one\hyp{}to\hyp{}one linear transformations which
preserve the norm). The assumption above the the mappings of the category $\mathbf{B}$ all have norm at most 1 is 
necessary in order to insure that the equivalences in $\mathbf{B}$ actually preserve the norm. If one admits arbitrary
linear transformations as mappings of the category, one obtains a larger category in which the equivalences are isomorphisms
(that is, one\hyp{}to\hyp{}one linear transformations)\footcite[180]{banach32}.

For quick reference, we sometimes describe a category by specifying only the objects involved (for examples, the category of
all discrete groups). In such a case, we imply the the mappings of this category are to be all mappings appropriate to the
objects in questions (for example, all homomorphisms).
	
\section{Functors in two arguments}\label{sec:bifunct}
For simplicity we define only the concept of a functor covariant in one argument and contravariant in another. The
generalization to any number of arguments of each type will be immediate.

Let $\mathbf{A}$, $\mathbf{B}$, and $\mathbf{C}$ be three categories. Let $T(A,B)$ be an \emph{object\hyp{}function}
which associates with each pair of objects $A\in\mathbf{A}, B\in\mathbf{B}$ an object $T(A,B)=C\in\mathbf{C}$,
and let $T(\alpha,\beta)$ be a \emph{mapping\hyp{}function} which associates with each pair of mappings $\alpha\in\mathbf{A},
\beta\in\mathbf{B}$ a mapping $T(\alpha,\beta)=\gamma\in\mathbf{C}$. For these functions we formulate certain conditions
already indicated in the example in the introduction.
\begin{thdef}
	The object\hyp{}function $T(A,B)$ and the mapping\hyp{}function $T(\alpha,\beta)$ form a \emph{functor} $T$, covariant in
	$\mathbf{A}$ and contravariant in $\mathbf{B}$, with values in $\mathbf{C}$, if
	\begin{equation}
		T(e_A,e_B)=e_{T(A,B)},\label{eq:funct_id}
	\end{equation}
	if, whenever $\alpha:A_1\rightarrow A_2\in\mathbf{A}$ and $\beta:B_1\rightarrow B_2\in\mathbf{B}$
	\begin{equation}
		T(\alpha,\beta): T(A_1,B_2)\rightarrow T(A_2,B_1),\label{eq:funct_prod1}
	\end{equation}
	and if, whenever $\alpha_2\alpha_1\in\mathbf{A}$ and $\beta_2\beta_1\in\mathbf{B}$,
	\begin{equation}
		T(\alpha_2\alpha_1,\beta_2\beta_1)=T(\alpha_2,\beta_1)T(\alpha_1,\beta_2)\label{eq:funct_prod2}
	\end{equation}
\end{thdef}
Condition \labelcref{eq:funct_prod1} guarantees the existence of the product of mappings appearing on the right of \cref{eq:funct_prod2}.

The formulas \labelcref{eq:funct_prod1,eq:funct_prod2} display the distinction between co- and contravariance.
The mapping $T(\alpha,\beta)=\gamma$ induced by $\alpha$ and $\beta$ from $T(A_1,-)$ to $T(A_2,-)$; that is, in the same
direction as does $\alpha$ hence the \emph{co}variance of $T$ in the argument $\mathbf{A}$. The induced mapping
$T(\alpha,\beta)$ at the same time operates in the direction opposite from that of $\beta$; thus it is contravariant
in $\mathbf{B}$. Essentially the same shift in direction is indicated by the orders of factors in formula \labelcref{eq:funct_prod2}
(the covariant $\alpha$'s appear in the same order on both sides; the contravariant $\beta$'s appear in on order
on the left and in the opposite order on the right). With this observation, the requisite formulas for functors in more
arguments can be set down.

According to this definition, the functor $T$ is composed of an object function and a mapping function. The latter is the
more important of the two; in fact, \cref{eq:funct_id} means that it determines the object function and 
therefore the whole functor, as stated in the following theorem.
\begin{theorem}\label{thm:functor}
	A function $T(\alpha,\beta)$ which associates to each pair of mappings $\alpha$ and $\beta$ in the respective
	categories $\mathbf{A}, \mathbf{B}$ a mapping $T(\alpha,\beta)=\gamma$ in a third category $\mathbf{C}$ is
	the mapping function of a functor $T$ covariant in $\mathbf{A}$ and contravariant in $\mathbf{B}$ if and
	only if the following two conditions hold:
\begin{enumerate}
		\item  $T(e_A,e_B)$ is an identity mapping in $\mathbf{C}$ for all identities $e_A,e_B$ of 
			$\mathbf{A}$ and $\mathbf{B}$.\label{idmap}
		\item  Whenever $\alpha_2\alpha_1\in\mathbf{A}$ and $\beta_2\beta_1\in\mathbf{B}$, then\label{func_map}
			$T(\alpha_2,\beta_1)T(\alpha_1,\beta_2)$ is defined and satisfies the equation
			\begin{equation}
				T(\alpha_2\alpha_1,\beta_2\beta_1)=T(\alpha_2,\beta_1)T(\alpha_1,\beta_2).\label{eq:funct_comp}
			\end{equation}
\end{enumerate}
	If  $T(\alpha,\beta)$ satisfies \cref{idmap,func_map}, the corresponding functor $T$ is uniquely determined, with
	an object function $T(A,B)$ given by the formula
	\begin{equation}\label{eq:id_map}
		e_{T(A,B)}=T(e_A,e_B)
	\end{equation}
\end{theorem}
\begin{proof}
	The necessity of \cref{idmap} and \cref{func_map} and the second half of the theorem are obvious.

	Conversely, let $T(\alpha,\beta)$ satisfy conditions~\ref{idmap} and~\ref{func_map}. Condition~\ref{idmap}
	means that an object function $T(A,B)$ can be defined by \cref{eq:id_map}. We must
	show that if $\alpha: A_1\rightarrow A_2$ and $\beta: B_1\rightarrow B_2$, then \cref{eq:funct_prod1}
	holds. Since $e(A_2)\alpha$ and $\beta e(B_1)$ are defined, the product $T(\alpha,\beta)T(e(A_1),e(B_2))$
	is defined.

	In virtue of the definition \labelcref{eq:id_map}, the products
	\begin{equation*}
		e(T(A_2,B_1))T(\alpha,\beta),\qquad T(\alpha,\beta)T(e(A_1),e(B_2))
	\end{equation*}
	are defined. This implies \cref{eq:funct_prod1}. 
\end{proof}

In any functor, the replacement of the arguments $A,B$ by equivalent arguments $A',B'$ will replace the 
value $T(A,B)$ by  an equivalent value $T(A',B')$. This fact may be alternatively stated as follows:
\begin{theorem}\label{thm:nat_eq}
	If $T$ is a functor on $\mathbf{A},\mathbf{B}$ to $\mathbf{C}$, and if $\alpha\in\mathbf{A}$
	and $\beta\in\mathbf{B}$ are equivalences, then $T(\alpha,\beta)$ is an equivalence in $\mathbf{C}$,
	with inverse $T{(\alpha,\beta)}^{-1}=T(\alpha^{-1},\beta^{-1})$.
\end{theorem}
\begin{proof}
	For the proof we assume that $T$ is covariant in $\mathbf{A}$ and contravariant in $\mathbf{B}$.
	The products $\alpha\alpha^{-1}$ and $\alpha^{-1}\alpha$ are then identities, and the definition
	of a functor shows that
	\begin{align*}
		T(\alpha,\beta)T(\alpha^{-1},\beta^{-1})&=T(\alpha\alpha^{-1},\beta^{-1}\beta),\\
		T(\alpha^{-1},\beta^{-1})T(\alpha,\beta)&=T(\alpha^{-1}\alpha,\beta\beta^{-1}).
	\end{align*}
	By \cref{eq:funct_id}, the terms on the right are both identities, which means
	that $T(\alpha^{-1},\beta^{-1})$ is an inverse for $T(\alpha,\beta)$, as asserted.
\end{proof}

\section{Examples of functors}\label{sec:funct_examples}
The same object function may appear in various functors, as is shown by the following example of one
covariant and one contravariant functor both with the same object function. In the category $\mathbf{S}$
of all set, the ``power'' functors $P^+$ and $P^-$ have the object function
\begin{equation*}
	P^+(S)=P^-(S)=\text{the set of all subsets of $S$.}
\end{equation*}
For any many\hyp{}one correspondence $\sigma:S_1\rightarrow S_2$ the respective mapping functions are defined
for any subset $A_1\subset S_1$ (or $A_2\subset S_2$) as\footnote{Here $\sigma A_1$ is the set of all
elements of $S_2$ of the form $\sigma x$ for $x\in A_1$, while $\sigma^{-1}A_2$ consists of all elements
$x\in S_1$ with $\sigma x\in A_2$. When $\sigma$ is an equivalence, with and inverse $\tau$,
$\tau A_2=\sigma^{-1}A_2$, so that no ambiguity as to the meaning of $\sigma^{-1}$ can arise.}
\begin{equation*}
	P^+(\sigma)A_1=\sigma A_1, \qquad P^-(\sigma)A_2=\sigma^{-1}A_2.
\end{equation*}
It is immediate that $P^+$ is a covariant functor and $P^-$ a contravariant one.

The cartesian product $X\times Y$ of two topological spaces is the object function of a functor of two
covariant variables $X$ and $Y$ in the category $\mathbf{X}$ of all topological spaces. For a continuous
transformations $\xi:X_1\rightarrow X_2$ and $\eta:Y_1\rightarrow Y_2$ the corresponding mapping function
$\xi\times\eta$ is defined for any point $(x_1,y_1)$ in the cartesian product $X_1\times Y_1$ as
\begin{equation*}
	\xi\times\eta(x_1,y_1)= (\xi x_1,\eta y_1).
\end{equation*}
One verifies that
\begin{equation*}
	\xi\times\eta:X_1\times Y_1\rightarrow X_2\times Y_2,
\end{equation*}
whenever the products $\xi_2\xi_1$ and $\eta_2\eta_1$ are defined. In virtue of these facts, the functions
$X\times Y$ and $\xi\times\eta$ constitute a covariant functor of two variables on the category $\mathbf{X}$.

The direct product of two groups is treated in exactly similar fashion; it gives a functor with the set function
$G\times H$ and the mapping function $\gamma\times\eta$, defined for $\gamma:G_1\rightarrow G_2$ and 
$\eta:H_1\rightarrow H_2$ exactly as was $\xi\times\eta$. The same applies to the category $\mathbf{B}$ of Banach
spaces, provided one fixes one of the usual possible definite procedures of norming the cartesian product of two
Banach spaces.

For a topological space $Y$ and a locally compact ($=$locally bicompact) Hausdorff space $X$ one may construct the
space $Y^X$ of all continuous mappings $f$ of the whole space $X$ into $Y$ ($f x\in Y$ for $x\in X$). A topology is
assigned to $Y^X$ as follows. Let $C$ be any compact subset of $X$, $U$ any open set in $Y$. The the set $[C,U]$ of
all $f\in Y^X$ with $f C\subset U$ is an open set in $Y^X$, and the most general open set in $Y^X$ is any union of 
finite intersections $[C_1,U_1]\cap\dotsm\cap [C_n,U_n]$.

This space $Y^X$ may be regarded as the object function of a suitable functor, $\map(X,Y)$. To construct a suitable
mapping function, consider any continuous transformations $\xi:X_1\rightarrow X_2$, $\eta:Y_1\rightarrow Y_2$.
For each $f\in Y_1^{X_2}$, one then has mappings acting thus:
\begin{equation*}
	X_1\xlongrightarrow{\xi} X_2\xlongrightarrow{f} Y_1\xlongrightarrow{\eta} Y_2,
\end{equation*}
so that one may derive a continuous transformation $\eta f\xi$ of $Y_2^{X_1}$. This correspondence 
$f\rightarrow\eta f\xi$, may be shown to be a continuous mapping of $Y_1^{X_2}$ into $Y_2^{X_1}$.
Hence we may define object and mapping functions ``Map'' by setting
\begin{equation}
	\map(X,Y)=Y^X,\qquad [\map(\xi,\eta)]f=\eta f\xi.\label{eq:map_func}
\end{equation}
The construction shows that
\begin{equation*}
	\map(\xi,\eta): \map(X_2,Y_1)\rightarrow\map(X_1,Y_2),
\end{equation*}
and hence suggests that this functor is contravariant in $X$ and covariant in $Y$. One observes at once
that $\map(\xi,\eta)$ is an identity when bot $\xi$ and $\eta$ are identities. Furthermore, if the products
$\xi_2\xi_1$ and $\eta_2\eta_1$ are defined, the definition of ``Map'' gives first,
\begin{equation*}
	[\map(\xi_2\xi_1,\eta_2\eta_1)] f = \eta_2\eta_1 f \xi_2\xi_1 = \eta_2(\eta_1 f \xi_2)\xi_1,
\end{equation*}
and second,
\begin{equation*}
	\map(\xi_1,\eta_2)\map(\xi_2,\eta_1) f = [\map(\xi_1,\eta_2)]\eta_1 f \xi_2 = \eta_2(\eta_1 f \xi_2)\xi_1.
\end{equation*}
Consequently
\begin{equation*}
	\map(\xi_2\xi_1,\eta_2\eta_1) = \map(\xi_1,\eta_2)\map(\xi_2,\eta_1),
\end{equation*}
which completes the verification that ``Map'', defined as in \cref{eq:map_func}, is a functor on
$\mathbf{X}_{\mathrm{lc}},\mathbf{X}$ to $\mathbf{X}$, contravariant in the first variable, covariant in the 
second, where $\mathbf{X}_{\mathrm{lc}}$ denotes the subcategory of $\mathbf{X}$ defined by the locally compact
Hausdorff spaces.

For abelian groups there is a similar functor ``Hom''. Specifically, let $G$ be a locally compact regular topological
group, $H$ a topological abelian group. We construct the set $\hm(G,H)$ of all (continuous) homomorphisms  $\phi$ of $G$
into $H$. The sum of two such homomorphisms $\phi_1$ and $\phi_2$ is defined by setting $(\phi_1+\phi_2)g =
\phi_1 g + \phi_2 g$, for each $g\in G$;\footnote{The group operation in $G$,$H$, and so on, will be written as
addition.} this sum is itself a homomorphism because $H$ is abelian.

Under this addition $\hm(G,H)$ is an abelian group. It is topologized by the family of neighborhoods $[C,U]$ of zero
defined as follows. Given $C$, any compact subset of $G$, and $U$, any open set in $H$ containing the zero of $H$,
$[C;U]$ consists of all $\phi\in\hm(G,H)$ with $\phi C\subset U$. With these definitions, $\hm(G,H)$ is a topological
group. If $H$ has a neighborhood of the identity containing no subgroup but the trivial one, one may prove that
$\hm(G,H)$ is locally compact.

This function of groups is the object function of a functor ``Hom''. For given $\gamma:G_1\rightarrow G_2$ and
$\eta: H_1\rightarrow H_2$ the mapping function is defined by setting
\begin{equation}\label{eq:func_grp}
	[\hm(\gamma,\eta)]\phi =  \eta\phi\gamma
\end{equation}
for each $\phi\in\hm{G_2,H_1}$. Formally this definition is exactly like \cref{eq:map_func}. One may show that
this definition \labelcref{eq:func_grp} does yield a continuous homomorphism
\begin{equation*}
	\hm(\gamma,\eta):\hm(G_2,H_1)\rightarrow\hm(G_1,H_2).
\end{equation*}
As in the previous case, Hom is a functor with values in the category $\mathbf{G}_\mathrm{a}$ of abelian groups,
defined for arguments in two appropriate subcategories of $\mathbf{G}$, contravariant in the first argument, $G$,
and covariant in the second, $H$.

For Banach spaces there is a similar functor. If $B$ and $C$ are two Banach spaces, let $\lin(B,C)$ denote the Banach
space of all linear transformation $\lambda$ of $B$ into $C$, with the usual definition of the norm of the transformation.
To describe the corresponding mapping function, consider any linear transformations $\beta:B_1\rightarrow B_2$ and
$\gamma:C_1\rightarrow C_2$ with $\|\beta\|\le 1$ and $\|\gamma\|\le 1$, and set, for each $\lambda\in\lin(B_2,C_1)$,
\begin{equation}\label{eq:func_banach}
	[\lin(\beta,\gamma)]\lambda= \gamma\lambda\beta.
\end{equation}
This is in fact a linear transformation
\begin{equation*}
	\lin(\beta,\gamma):\lin(B_2,C_1)\rightarrow\lin(B_1,C_2)
\end{equation*}
of norm at most 1. As in the previous cases, Lin is a functor on $\mathbf{B},\mathbf{B}$ to $\mathbf{B}$, contravariant
in its first argument and covariant in the second.

In case $C$ is fixed to be the Banach space $\R$ of all real numbers with the absolute value as norm, $\lin(B,C)$ is just
the Banach space conjugate to $B$, in the usual sense. This leads at once to the functor
\begin{equation*}
	\dual(B)= \lin(b,\R),\qquad \dual(\beta)= \lin(\beta,e_\mathbf{R}).
\end{equation*}
This is a contravariant functor on $\mathbf{B}$ to $\mathbf{B}$.

Another example of a functor on groups is the tensor product $G\otimes H$ of two abelian groups. This functor has been
discussed in more detail in our Proceedings note cited above.

\section{Slicing of functors}\label{sec:funct_slice}
The last example involved the process of holding one of the arguments of a functor constant. This process occurs
elsewhere (for example, in the character group theory, \cref{sec:char} below), and falls at once under the following
theorem.
\begin{theorem}\label{thm:currying}
	If  $T$ is a functor covariant in $\mathbf{A}$, contravariant in $\mathbf{B}$, with values in $\mathbf{C}$,
	then for each fixed $B\in\mathbf{B}$ the definitions
	\begin{equation*}
		S(A)= T(A,B),\qquad S(\alpha)= T(\alpha,e_B)
	\end{equation*}
	yield a functor $S$ on $\mathbf{A}$ to $\mathbf{C}$ with the same variance (in $\mathbf{A}$) as $T$.
\end{theorem}

This ``slicing'' of a functor may be partially inverted, in that the functor $T$ is determined by its object function
and its two ``sliced'' mapping functions, in the following sense.
\begin{theorem}\label{thm:func_comm}
	Let $\mathbf{A}$, $\mathbf{B}$, $\mathbf{C}$ be three categories and $T(A,B)$, $T(\alpha,B)$, $T(A,\beta)$
	three functions such that for each fixed $B\in\mathbf{B}$ the functions $T(A,B)$, $T(\alpha,B)$ form a
	covariant functor on $\mathbf{A}$ to $\mathbf{C}$, while for each $A\in\mathbf{A}$ the functions $T(A,B)$
	and $T(A,\beta)$ give a contravariant functor on $\mathbf{B}$ to $\mathbf{C}$. If in addition for each
	$\alpha:A_1\rightarrow A_2$ in $\mathbf{A}$ and $\beta:B_1\rightarrow B_2$ in $\mathbf{B}$ we have
	\begin{equation}\label{eq:func_comm1}
		T(A_2,\beta) T(\alpha,B_2) = T(\alpha,B_1) T(A_1,\beta),
	\end{equation}
	then the functions $T(A,B)$ and
	\begin{equation}\label{eq:func_comm2}
		T(\alpha,\beta)=T(\alpha,B_1)T(A_1,\beta)
	\end{equation}
	from a functor covariant in $\mathbf{A}$, contravariant in $\mathbf{B}$, with values in $\mathbf{C}$
\end{theorem}
\begin{proof}
	\Cref{eq:func_comm1} merely states the equivalence of the two paths about the following
	commutative square:
\begin{equation*}
	\begin{tikzpicture}
		\node (a) at (0,0)
		{ \begin{tikzcd}[sep=huge]
			T({A_1,B_2}) \arrow[r,"{T(\alpha,B_2)}"] \arrow[d, "{T(A_1,\beta)}"] & T(A_2,B_2) 
			\arrow[d,"{T(A_2,\beta)}"] \\
			T(A_1,B_1) \arrow[r,"{T(\alpha,B_1)}"] & T(A_2,B_1)
		\end{tikzcd}};
	\end{tikzpicture}
\end{equation*}
	The result of either path is then taken in \cref{eq:func_comm2} to define the mapping function,
	which then certainly satisfies condition \labelcref{eq:funct_id,eq:funct_prod1} of the
	definition of a functor. The proof of the basic product condition \labelcref{eq:funct_prod2} is best
	visualized by writing out a $3\times 3$ array of values $T(A_i,B_j)$.
\end{proof}
\pagebreak[2]
The significance of this theorem is essentially this: in verifying that a given object and mapping functions
do yield a functor, one may replace the verification of the product condition \labelcref{eq:funct_prod2} in
two variables by a separate verification, one variable at a time, provided one \emph{also} proves that the
order of application of these one-variable mappings can be interchanged (condition \labelcref{eq:func_comm1}).

\section{Foundations}\label{sec:found}
We remarked in \cref{sec:ex_cat} that such examples as the ``category of \emph{all} sets'', the ``category
of \emph{all} groups'' are illegitimate. The difficulties and antinomies here involved are exactly those of
ordinary intuitive \textsl{\foreignlanguage{german}{Mengenlehre}}; no essentially new paradoxes are apparently
involved. Any rigorous foundation capable of supporting the ordinary theory of classes would equally well 
support our theory. Hence we have chosen to adopt the intuitive standpoint, leaving the reader free to insert
whatever type of logical foundation (or absence thereof) he may prefer. These ideas will now be illustrated,
with particular reference to the category of groups.

It should be observed first that the whole concept of a category is essentially an auxiliary one; our basic
concept are essentially those of a \emph{functor} and of a natural transformation (the latter is defined
in the next chapter). The idea of a category is required only by the precept that every function should have
a definite class as domain and a definite class as range, for the categories are provided as the domains and
range of functors. Thus one could drop the category concept altogether and adopt an even more intuitive 
standpoint, in which a functor such as ``Hom'' is not defined over the category of ``all'' groups, but for
each particular pair of groups which may be given. The standpoint would suffice for the applications, inasmuch
as none of our developments will involve elaborate constructions on the categories themselves.

For a more careful treatment, we may regard a group $G$ as a pair, consisting of a set $G_0$ and a ternary
relation $g\cdot h = k$ on this set, subject to the usual axioms of group theory. This makes explicit the
usual tacit assumption that a group is not just the set of its elements (two groups can have the \emph{same}
elements, yet different operations). If a pair is constructed in the usual manner as a certain class, this
means that each subcategory of the category of ``all'' groups is a class of pairs; each pair being a class
of groups with a class of mappings (binary relations). Any given system of foundations will then legitimize
those subcategories which are allowable classes in the system in question.

Perhaps the simplest precise device would be to speak not of \emph{the} category of groups, but of \emph{a}
category of groups (meaning, any legitimate such category). A functor such as ``Hom'' is then a functor which
can be defined for any two suitable categories of groups, $\mathbf{G}$ and $\mathbf{H}$. This procedure
has the advantage of precision, the disadvantage of a multiplicity of categories and functors. This multiplicity
would be embarrassing in the study of composite functors (\cref{sec:funct_comp} below).

One might choose to adopt the (unramified) theory of types as a foundation for the theory of classes.
One then can speak of the category $\mathbf{G}_m$ of all abelian groups of type $m$. The functor ``Hom''
could then have both arguments in $\mathbf{G}_m$, while its values would be in the same category
$\mathbf{G}_{m+k}$ of groups of higher type $m+k$. This procedure affects each functor with the same
sort of typical ambiguity adhering to the arithmetical concepts in the Whitehead-Russell development.
Isomorphisms between groups of different types would have to be considered, as in the simple isomorphism
$\hm(\mathbf{F,G})\cong\mathbf{G}$ (see \cref{sec:trans_ex}); this would somewhat complicate the 
natural isomorphisms treated below.

One can also choose a set of axioms for classes as in the Fraenkel\hyp{}von Neumann\hyp{}Bernays system.
A category is then any (legitimate) class in the sense of this axiomatics. Another device would be that
of restricting the cardinal number, considering the category of all denumerable groups, of all groups of
cardinal at most the cardinal of the continuum, and so on. The subsequent developments may be suitably
interpreted under any on of these viewpoints.

\chapter{Natural equivalence of functors}\label{ch:nat_equ}
\section{Transformations of functors}\label{sec:nat_tra}
Let $T$ and $S$ be two functors on $\mathbf{A,B}$ to $\mathbf{C}$ which are \emph{concordant}; that is,
which have the same variance in $\mathbf{A}$ and the same variance in $\mathbf{B}$. To be specific,
assume both $T$ and $S$ covariant in $\mathbf{A}$ and contravariant $\mathbf{B}$. Let $\tau$ be a
function which associates to each pair of objects $A\in\mathbf{A},B\in\mathbf{B}$ a mapping 
$\tau(A,B)=\gamma$ in $\mathbf{C}$.
\begin{thdef}
	The function $\tau$ is a ``natural'' transformation of the functor $T$, covariant in $\mathbf{A}$
	and contravariant in $\mathbf{B}$, into the concordant functor $S$ provided that, for each pair
	of objects $A\in\mathbf{A},B\in\mathbf{B}$,
	\begin{equation}\label{eq:nat_trans}
		\tau(A,B): T(A,B)\rightarrow S(A,B)\text{ in }\mathbf{C},
	\end{equation}
	and provided, whenever $\alpha:A_1\rightarrow A_2$ in $\mathbf{A}$ and $\beta:B_1\rightarrow B_2$
	in $\mathbf{B}$, that
	\begin{equation}\label{eq:nat_trans2}
		\tau(A_2,B_1) T(\alpha,\beta) = S(\alpha,\beta)\tau(A_1,B_2).
	\end{equation}
	When these conditions hold, we write
	\begin{equation*}
		\tau:T\rightarrow S.
	\end{equation*}
\end{thdef}
If in addition each $\tau(A,B)$ is an equivalence mapping of the category $\mathbf{C}$, we call $\tau$ a
\emph{natural equivalence} of $T$ to $S$ (notation: $\tau: T\rightleftarrows S$) and say that the functors
$T$ and $S$ are \emph{naturally equivalent}. In this case condition \labelcref{eq:nat_trans2} can be rewritten as
\begin{equation}\label{eq:nat_eq}
	\tau(A_2,B_1) T(\alpha,\beta){[\tau(A_1,B_2)]}^{-1}= S(\alpha,\beta).
\end{equation}

Condition \labelcref{eq:nat_trans} of this definition is equivalent to the requirement that both product in \cref{eq:nat_trans2}
are always defined. Condition \labelcref{eq:nat_trans2} is illustrated by the equivalence
of the two paths indicated in the following diagram:
\begin{equation*}
\begin{tikzpicture}
	\node (a) at (0,0)
	{ \begin{tikzcd}[sep=huge]
		T({A_1,B_2}) \arrow[r,"{T(\alpha,\beta)}"] \arrow[d, "{\tau(A_1,B_2)}"] & T(A_2,B_1) 
		\arrow[d,"{\tau(A_2,B_1)}"] \\
		S(A_1,B_2) \arrow[r,"{S(\alpha,\beta)}"] & S(A_2,B_1)
	\end{tikzcd}};
\end{tikzpicture}
\end{equation*}

Given three concordant functors $T$, $S$, and $R$ on $\mathbf{A,B}$ to $\mathbf{C}$, with natural
transformations $\tau:T\rightarrow S$ and $\sigma:S\rightarrow R$, the product
\begin{equation*}
	\rho(A,B)=\sigma(A,B)\tau(A,B)
\end{equation*}
is defined as a mapping in $\mathbf{C}$, and yields a natural transformation $\rho:T\rightarrow R$.
If $\tau$ and $\sigma$ are natural equivalences, so is $\rho = \sigma\tau$.

Observe also that if $\tau:T\rightarrow S$ is a natural equivalence, then the function $\tau^{-1}$
is defined by $\tau^{-1}(A,B)={[\tau(A,B)]}^{-1}$ is a natural equivalence $\tau^{-1}:S\rightarrow T$.
Given any functor $T$ on $\mathbf{A,B}$ to $\mathbf{C}$, the function
\begin{equation*}
	\tau_0(A,B) = e_{T(A,B)}
\end{equation*}
is a natural equivalence $\tau_0:T\rightleftarrows T$. These remarks imply that the concept of natural
equivalence of functors is reflexive, symmetric and transitive.

In demonstrating that a given mapping $\tau(A,B)$ is actually a natural transformation, it suffices to
prove the rule \labelcref{eq:nat_trans2} only in these cases in which all except one of the mappings
$\alpha,\beta,\dotsc$ is an identity. To state this result it is convenient to introduce a simplified 
notation for the mapping function when one argument is an identity, by setting
\begin{equation*}
	T(\alpha,\beta) = T(\alpha,e_B),\qquad T(A,\beta)= T(e_A,\beta).
\end{equation*}
\begin{theorem}\label{thm:nat_trans}
	Let $T$ and $S$ be functors covariant in $\mathbf{A}$ and contravariant in $\mathbf{B}$,
	with values in $\mathbf{C}$, and let $\tau$ be a function which associates to each pair of
	objects $A\in\mathbf{A}$, $B\in\mathbf{B}$ a mapping with \cref{eq:nat_trans}. A necessary
	and sufficient condition that $\tau$ be a natural transformation $\tau:T\rightarrow S$ is
	that for each mapping $\alpha:A_1\rightarrow A_2$ and each object $B\in\mathbf{B}$ one has
	\begin{equation}\label{eq:nat_comm}
		\tau(A_2,B) T(\alpha,B) = S(\alpha,B)\tau(A_1,B),
	\end{equation}
	and that, for each $A\in\mathbf{A}$ and each $\beta:B_1\rightarrow B_2$ one has
	\begin{equation}\label{eq:nat_comm2}
		\tau(A,B_1) T(A,\beta) = S(A,\beta)\tau(A,B_2).
	\end{equation}
\end{theorem}
\begin{proof}
	The necessity of these conditions is obvious, since they are simply the special case of \cref{eq:nat_trans2}
	in which $\beta=e_B$ and $\alpha=e_A$, respectively. The sufficiency
	can best be illustrated by the following diagram, applying to any mappings $\alpha:A_1\rightarrow A_2$
	and $\beta:B_1\rightarrow B_2$:
	\begin{equation*}
	\begin{tikzpicture}
	\node (a) at (0,0)
	{ \begin{tikzcd}[sep=huge]
		T({A_1,B_2}) \arrow[r,"{\tau(A_1,B_2)}"] \arrow[d, "{T(\alpha,B_2)}"] & S(A_1,B_2) 
		\arrow[d,"{S(\alpha,B_2)}"] \\
		T(A_2,B_2) \arrow[r,"{\tau(A_2,B_2)}"] \arrow[d, "{T(A_2,\beta)}"] & S(A_2,B_2)
		\arrow[d,"{S(A_2,\beta)}"] \\
		T(A_2,B_1) \arrow[r,"{\tau(A_2,B_1)}"] & S(A_2,B_1)
	\end{tikzcd}};
	\end{tikzpicture}
	\end{equation*}
	\Cref{eq:nat_comm} states the equivalence of the results found by following 
	either path around the upper small rectangle, and \cref{eq:nat_comm2} makes
	a similar assertion for the bottom rectangle. Combining these successive equivalences,
	we have the equivalence of the two paths around the edges of the whole rectangle; this
	is the requirement \labelcref{eq:nat_trans2}. This argument can be easily set down formally.
	
\end{proof}

\section{Categories of functors}\label{sec:funct_cat}
The functors may be made the objects of a category in which the mapping are natural transformation.
Specifically, given three fixed categories $\mathbf{A}$, $\mathbf{B}$ and $\mathbf{C}$, form
the category $\mathbf{F}$ for which the objects are the functors $T$ covariant in $\mathbf{A}$ and
contravariant $\mathbf{B}$, with values in $\mathbf{C}$, and for which the mappings are the natural
transformation $\tau:T\rightarrow S$. This requires some caution, because we may have $\tau:T\rightarrow
S$ and $\tau:T'\rightarrow S'$ for the same function $\tau$ with different functors $T$, $T'$ (which
would have the same object function but different mapping functions). To circumvent this difficulty
we define a mapping in the category $T$ to be a triple $[\tau,T,S]$ with $\tau:T\rightarrow S$.
The product of mappings $[\tau,T,S]$ and $[\sigma,S',R]$ is defined if and only if $S=S'$; in this
case it is
\begin{equation*}
	[\sigma,S,R]\circ[\tau,T,S]=[\sigma\tau,T,R].
\end{equation*}
We verify that the \cref{ax:c1,ax:c2,ax:c3} of \cref{sec:def_cat} are satisfied. Furthermore we define, for each
functor $T$
\begin{equation*}
	e_T = [\tau_T,T,T],\text{ with } \tau_T(A,B)=e_{T(A,B)},
\end{equation*}
and verify the remaining \cref{ax:c4,ax:c5}. Consequently $\mathbf{F}$ is a category. In this category it can
be proved easily that $[\tau,T,S]$ is an equivalence mapping if and only if $\tau:T\rightleftarrows S$;
consequently the concept of the natural equivalence of functors agrees with the concept of equivalence
of objects in the category $\mathbf{T}$ of functors.

This category $\mathbf{T}$ is useful chiefly in simplifying the statements and proofs of various facts
about functors, as will appear subsequently.

\section{Composition of functors}\label{sec:funct_comp}
This process arises by the familiar ``function of a function'' procedure, in which for the argument of a
functor we substitute the value of another functor. For example, let $T$ be a functor on $\mathbf{A,B}$
to $\mathbf{C}$, $R$ a functor on $\mathbf{C,D}$ to $\mathbf{E}$. Then $S=R\circ (T,I)$, defined by
setting
\begin{equation*}
	S(A,B,D) = R(T(A,B),D),\qquad S(\alpha,\beta,\gamma)= R(T(\alpha,\beta),\gamma),
\end{equation*}
for objects $A\in\mathbf{A}$, $B\in\mathbf{B}$, $D\in\mathbf{D}$ and mappings $\alpha\in\mathbf{A}$,
$\beta\in\mathbf{B}$, $\gamma\in\mathbf{D}$, is a functor on $\mathbf{A,B,D}$ to $\mathbf{E}$.
In the argument $\mathbf{D}$, the variance of $S$ is just the the variance of $R$. The variance of 
$R$ in $\mathbf{A}$ (or $\mathbf{B}$) may be determined by the rule of signs (with $+$ for covariance,
$-$ for contravariance): (variance of $S\in\mathbf{A})=(\text{variance of } R\in\mathbf{C})\times
(\text{variance of } T\in\mathbf{A}$).

Composition can also be applied to natural transformations. To simplify the notation, assume that $R$ is
a functor in \emph{one} variable, contravariant on $\mathbf{C}$ to $\mathbf{E}$, and that $T$ is
covariant in $\mathbf{A}$, contravariant in $\mathbf{B}$ with values in $\mathbf{C}$. The composite
$R\circ T$ is then contravariant in $\mathbf{A}$, covariant in $\mathbf{B}$. Any pair of natural
transformations
\begin{equation*}
	\rho:R\rightarrow R',\qquad \tau=T\rightarrow T'
\end{equation*}
give rise to a natural transformation
\begin{equation*}
	\rho\circ\tau : R\circ T'\rightarrow R'\circ T
\end{equation*}
defined by setting
\begin{equation*}
	\rho\circ\tau(A,B)=\rho(T(A,B)) R(\tau(A,B)).
\end{equation*}
Because $\rho$ is natural, $\rho\circ\tau$ could equally well be defined as
\begin{equation*}
	\rho\circ\tau(A,B)= R'(\tau(A,B))\rho(T'(A,B)).
\end{equation*}
This alternative means that the passage from $R\circ T'(A,B)$ to $R'\circ T(A,B)$
can be made either through  $R\circ T(A,B)$ or through  $R'\circ T'(A,B)$, without
altering the final result. The resulting \emph{composite transformation} $\rho\circ\tau$
has all the usual formal properties appropriate to the mapping function of the ``functor''
$R\circ T(A,B)$; specifically,
\begin{equation*}
	(\rho_2\rho_1)\circ (\tau_1\tau_2) = (\rho_2\circ \tau_2)(\rho_1\circ\tau_1),
\end{equation*}
as may be verified by a suitable $3\times 3$ diagram.

These properties show that the functions $R\circ T$ and $\rho\circ\tau$ determine a functor
$C$, defined on the categories $\mathbf{R}$ and $\mathbf{T}$ of functors, with values in
a category $\mathbf{S}$ of functors, covariant in $\mathbf{R}$ and contravariant in $\mathbf{T}$
(because of the contravariance of $R$). Here $\mathbf{R}$ is the category of all contravariant
functors $R$ on $\mathbf{C}$ to $\mathbf{E}$, while $\mathbf{S}$ and $\mathbf{T}$ are the
categories of all functors $S$ and $T$, of appropriate variances, respectively. In each case, the
mappings of the category of functors are natural transformations, as described in the previous
section. To be more explicit, the mapping function $C(\rho,\tau)$ of this functor is not the 
simple composite $\rho\circ\tau$, but the triple $[\rho\circ\tau, R\circ T', R'\circ T]$.

Since $\rho\circ\tau$ is essentially the mapping function of a functor, we know by \autoref{thm:nat_eq}
that if $\rho$ and $\tau$ are natural equivalences, then $\rho\circ\tau$ is an equivalence. Consequently,
if he pairs $R$ and $R'$, $T$ and $T'$ are naturally equivalent, so is the pair of composite $R\circ T$
and $R'\circ T'$.

It is easy to verify that the composition of functors and natural transformations is associative, so that
symbols like $R\circ T\circ S$ may be written without parentheses.

If in the definition of $\rho\circ\tau$ above it occurs that $T=T'$ and that $\tau$ is the identity
transformation $T\rightarrow T$ we shall write $\rho\circ T$ instead of $\rho\circ\tau$. Similarly
we shall write $R\circ\tau$ in the case when $R=R'$ and $\rho$ is the identity transformation on
$R\rightarrow R$.

\section{Examples of transformations}\label{sec:trans_ex}
The associative and commutative laws for the direct and cartesian products are isomorphisms which can
be regarded as equivalences between functors. For example, let $X$, $Y$ and $Z$ be three topological
spaces, and let the homeomorphism
\begin{equation}\label{eq:ass_prod}
	(X\times Y)\times Z \cong X\times (Y\times Z)
\end{equation}
be established by the usual correspondence $\tau=\tau(X,Y,Z)$, defined for any point $((x,y),z)$ in the
iterated cartesian product $(X\times Y)\times Z$ by
\begin{equation*}
	\tau(X,Y,Z)((x,y),z) =  (x,(y,z)).
\end{equation*}
Each $\tau(X,Y,Z)$ is then an equivalence mapping in the category $\mathbf{X}$ of spaces.
Furthermore each side of \cref{eq:ass_prod} may be considered as the object function of a covariant functor 
obtained by composition of the cartesian product functor with itself. The corresponding mapping function
are obtained by parallel composition as $(\xi\times\eta)\times\zeta$ and $\xi\times(\eta\times\zeta)$.
To show that $\tau(X,Y,Z)$ is indeed a natural equivalence, we consider three mappings $\xi:X_1\rightarrow X_2$,
$\eta:Y_1\rightarrow Y_2$ and $\zeta:Z_1\rightarrow Z_2$, and show that
\begin{equation*}
	\tau(X_2,Y_2,Z_2)[(\xi\times\eta)\times\zeta] = [\xi\times(\eta\times\zeta)]\tau(X_1,Y_1,Z_1).
\end{equation*}
This identity may be verified by applying each side to an arbitrary point $((x_1,y_1),z_1)$ in the space
$(X_1\times Y_1)\times Z_1$; each transforms it into the point $(\xi x_1,(\eta y_1, \zeta z_1))$ in
$X_2\times(Y_2\times Z_2)$.

In similar fashion the homeomorphism $X\times Y\cong Y\times X$ may be interpreted as a natural equivalence,
defined as $\tau(X,Y)(x,y)= (y,x)$. In particular, if $X$, $Y$ and $Z$ are discrete spaces (that is, are simply
sets), there remarks show that the associative and commutative laws for the (cardinal) product of two sets
are natural equivalences between functors.

For similar reasons, the associative and commutative laws for the direct product of groups are natural
equivalences (or \emph{natural isomorphisms}) between functors of groups. The same laws for Banach spaces, with
a fixed convention as to the construction of the norm in the cartesian product of two such spaces, are natural 
equivalences between functors.

If $\mathbb{Z}$ is the (fixed) additive group of integers, $H$ any topological abelian group, there is an
isomorphism
\begin{equation}\label{eq:grp_hom}
	\hm(\mathbb{Z},H)\cong H
\end{equation}
in which both sides may be regarded as covariant functors of a single argument $H$. This isomorphism
$\tau=\tau(H)$ is defined for any homomorphism $\phi\in\hm(\mathbb{Z},H)$ by setting $\tau(H)\phi=\phi(1)\in H$.
One observes that $\tau(H)$ is indeed a (bicontinuous) isomorphism, that is, an equivalence in the category
of topological abelian groups. That $\tau(H)$ actually is a natural equivalence between functors is shown by 
proving, for any $\eta:H_1\rightarrow H_2$, that
\begin{equation*}
	\tau(H_2)\hm(e_J,\eta)=\eta\tau(H_1).
\end{equation*}
There is also a second natural equivalence between the functors indicated in \cref{eq:grp_hom}, obtained by
setting $\tau'(H)\phi=\phi(-1)$.

With the fixed Banach space $\R$ of real numbers there is a similar formula
\begin{equation}\label{eq:ban_hom}
	\lin(\R,B)\cong B
\end{equation}
for any Banach space $B$. This gives a natural equivalence $\tau=\tau(B)$ between two covariant functors of
one argument in the category $\mathbf{B}$ of all Banach spaces. Here $\tau(B)$ is defined by setting
$\tau(B)l= l(1)$ for each linear transformation $l\in\lin(\R,B)$; another choice of $\tau$ would set
$\tau(B)l= l(-1)$.

For topological spaces there is a distributive law for the functors ``Map'' and the direct product functor,
which may be written as a natural equivalence
\begin{equation}\label{eq:top_eq}
	\map(Z,X)\times\map(Z,Y)\cong\map(Z,X\times Y)
\end{equation}
between two composite functors, each contravariant in the first argument $Z$ and covariant in the other two
arguments $X$ and $Y$. To define this natural equivalence
\begin{equation*}
	\tau(X,Y,Z): \map(Z,X)\times\map(Z,Y)\rightleftarrows \map(Z,X\times Y)
\end{equation*}
consider any pair of mappings $f\in\map(Z,X)$ and $g\in\map(Z,Y)$ and set, for each $z\in Z$,
\begin{equation*}
	[\tau(f,g)](z) = (f(z),g(z)).
\end{equation*}
It can be shown that this definition does indeed give the homeomorphism \labelcref{eq:top_eq}. It is furthermore
natural, which means that, for mappings $\xi:X_1\rightarrow X_2$, $\eta:Y_1\rightarrow Y_2$ and $\zeta:Z_1
\rightarrow Z_2$,
\begin{equation*}
	\tau(X_2,Y_2,Z_1)[\map(\zeta,\xi)\times\map(\zeta,\eta)]=\map(\zeta,\xi\times\eta)\tau(X_1,Y_1,Z_2).
\end{equation*}
The proof of this statement is a straightforward application of the various definitions involved. Both sides
are mappings carrying $\map(Z_2,X_1)\times\map(Z_2,Y_1)$ into $\map(Z_1,X_2\times Y_2)$. They will be equal
if they give identical results when applied to an arbitrary element $(f_2,g_2)$ in the first space.
These applications give, by the definition of the mapping functions of the functors ``Map'' and ``$\times$'',
the respective elements
\begin{equation*}
	\tau(X_2,Y_2,Z_1)(\xi f_2\zeta, \eta g_2\zeta), \qquad (\xi\times\eta)\tau(X_1,Y_1,Z_2)(f_2,g_2)\zeta.
\end{equation*}
Both are in $\map(Z_1,X_2\times Y_2)$. Applied to an arbitrary $z\in Z_1$, we obtain in both cases, by the
definition of $\tau$, the same element $(\xi f_2\zeta(z), \eta g_2\zeta(z))\in X_2\times Y_2$.

For groups and Banach spaces there are analogous natural equivalences
\begin{equation}\label{eq:grp_eq}
	\hm(G,H)\times\hm(G,K)\cong\hm(G,H\times K),
\end{equation}
\begin{equation}\label{eq:ban_eq}
	\lin(B,C)\times\lin(B,D)\cong\lin(B,C\times D).
\end{equation}
In each case the equivalence is given by a transformation defined exactly as before. In the formula for Banach
spaces we assume that the direct product is normed by the maximum formula. In the case of any other formula for
the norm in a direct product, we can assert only that $\tau$ is a one\hyp{}to\hyp{}one linear transformation of
norm one, but not necessarily a transformation preserving the norm. In such a case $\tau$ then gives merely a
natural transformation of the functor on the left to the functor on the right.

For groups there is another type of distributive law, which is an equivalence transformation,
\begin{equation*}
	\hm(G,K)\times\hm(H,K)\cong\hm(G\times H, K),
\end{equation*}
The transformation $\tau(G,H,K)$ is defined for each pair $(\phi,\psi)\in\hm(G,K)\times\hm(H,K)$ by setting
\begin{equation*}
	[\tau(G,H,K)(\phi,\psi)](g,h)= \phi g + \psi h
\end{equation*}
for every element $(g,h)$ in the direct product $G\times H$. The properties of $\tau$ are proved as before.

It is well known that a function $g(x,y)$ of two variables $x$ and $y$ may be regarded as a function $\tau g$
of the first variable $x$ for which the values are in turn functions of the second variable $y$. In other
words, $\tau g$ is defined by
\begin{equation*}
	[[\tau g](x)](y)= g(x,y).
\end{equation*}
It may be shown that the correspondence $g\rightarrow\tau g$ does establish a homeomorphism between the spaces
\begin{equation*}
	Z^{X\times Y}\cong {(Z^Y)}^X,
\end{equation*}
where $Z$ is any topological space and $X$ and $Y$ are locally compact Hausdorff spaces. This is a ``natural''
homeomorphism, because the correspondence $\tau=\tau(X,Y,Z)$ defined above is actually a natural equivalence
\begin{equation*}
	\tau(X,Y,Z):\map(X\times Y, Z)\rightleftarrows\map(X,\map(Y,Z))
\end{equation*}
between the two composite functors whose object functions are displayed here.

To prove that $\tau$ is natural, we consider any mappings $\xi:X_1\rightarrow X_2$, $\eta:Y_1\rightarrow Y_2$,
$\zeta:Z_1\rightarrow Z_2$, and show that
\begin{equation}\label{eq:top_map}
	\tau(X_1,Y_1,Z_2)\map(\xi\times\eta,\zeta)=\map(\xi,\map(\eta,\zeta))\tau(X_2,Y_2,Z_1).
\end{equation}
Each side of this equation is a mapping which applies to any element $g_2\in\map(X_2\times Y_2,Z_1)$ to give
an element of $\map(X_1,\map(Y_1,Z_2))$. The resulting elements may be applied to an $x_1\in X_1$ to give an
element of $\map(Y_1,Z_2)$, which in turn may be applied to any $y_1\in Y_1$. If each side of \cref{eq:top_map}
is applied in this fashion, and simplified by the definitions of $\tau$ and of the mapping functions of the
functors involved, one obtains in both cases the same element $\zeta g_2(\xi x_1,\eta y_1)\in Z_2$.
Hence \cref{eq:top_map} holds, and $\tau$ is natural.

Incidentally, the analogous formula for groups uses the tensor product $G\otimes H$ of two groups, and gives
an equivalence transformation
\begin{equation*}
	\hm(G\otimes H, k)\cong\hm(G,\hm(H,K)).
\end{equation*}
The proof appears in our Proceedings note~\footcite{groups42} quoted in the introduction.

Let $D$ be a fixed Banach space, while $B$ and $C$ are two (variable) Banach spaces. To each pair of linear
transformations $\lambda$ and $\mu$, with $\|\lambda\|\le 1$ and $\|\mu\|\le 1$, and with
\begin{equation*}
	B\xlongrightarrow{\lambda} C \xlongrightarrow{\mu} D,
\end{equation*}
there is associated a composite linear transformation $\mu\lambda$, with $\mu\lambda:B\rightarrow D$.
Thus there is a correspondence $\tau=\tau(B,C)$ which associates to each $\lambda\in\lin(B,C)$ a
linear transformation $\tau\lambda$ with
\begin{equation*}
	[\tau\lambda](\mu)=\mu\lambda\in\lin(B,D).
\end{equation*}
Each  $\tau\lambda$ is a linear transformation of $\lin(C,D)$ into $\lin(B,D)$ with norm at most one;
consequently $\tau$ establishes a correspondence
\begin{equation}\label{eq:ban_nat}
	\tau(B,C):\lin(B,C)\rightarrow\lin(\lin(C,D),\lin(B,D)).
\end{equation}
It can be readily shown that $\tau$ itself is a linear transformation, and that $\|\tau(\lambda)\|=\|\lambda\|$,
so that $\tau$ is an isometric mapping.
This mapping $\tau$ actually gives a transformation between the functors in \cref{eq:top_map}.
If the space $D$ is kept fixed\footnote{We keep the space $D$ fixed because in one of these functors it appears twice, once as a
covariant argument and once as a contravariant one.}, the functions $\lin(B,C)$ and $\lin(\lin(C,D),\lin(B,D))$
are object functions of functors contravariant in $B$ and covariant in $C$, with values in the category $\mathbf{B}$
of Banach spaces. Each $\tau=\tau(B,C)$ is a mapping of this category; thus $\tau$ is a natural transformation
of the first functor in the second provided that, whenever $\beta:B_1\rightarrow B_2$ and $\gamma:C_1\rightarrow C_2$,
\begin{equation}\label{eq:ban_nat2}
	\tau(B_1,C_2)\lin(\beta,\gamma) = \lin(\lin(\gamma,e),\lin(\beta,e))\tau(B_2,C_1),
\end{equation}
where $e=e_D$ is the identity mapping of $D$ into itself. Each side of \cref{eq:ban_nat2} is a mapping of
$\lin(B_2,C_1)$ into $\lin(\lin(C_2,D),\lin(B_1,D))$. Apply each side to any $\lambda\in\lin(B_2,C_1)$, and
let the result act on any $\mu\in\lin(C_2,D)$. On the left side, the result of these applications simplifies
as follows (in each step the definition used is cited at the right):
\begin{align*}
	\{[\tau(& B_1, C_2)]\lin(\beta,\gamma)\lambda\}\mu &&\\ 
	&= \{[\tau(B_1,C_2)](\gamma\lambda\beta)\}\mu &&\text{(Definition of $\lin(\beta,\gamma))$}\\
	&=\mu\gamma\lambda\beta &&\text{(Definition of $\tau(B_1,C_2))$}.\\
	\text{The } & \text{right side similarly becomes}\\
	\{&\lin(\lin(\gamma,e),\lin(\beta,e))[\tau(B_2,C_1)\lambda]\}\tau &&\\
	&= \{\lin(\beta,e)[\tau(B_2,C_1)\lambda]\lin(\gamma,e)\}\tau &&\text{(Definition of $\lin(-,-))$}\\
	&= \lin(\beta,e)\{[\tau(B_2,C_1)\lambda](\mu\gamma)\} &&\text{(Definition of $\lin(\gamma,e))$}\\
	&= \lin(\beta,e)(\mu\gamma\lambda) &&\text{(Definition of $\tau(B_2,C_1))$}\\
	&=\mu\gamma\lambda\beta &&\text{(Definition of $\lin(\beta,e)$)}.
\end{align*}
The identity of these two results shows that $\tau$ is indeed a natural transformation of functors.

In the special case when $D$ is the space of real numbers, $\lin(C,D)$ is simply the conjugate space
$\dual(C)$. Thus we have the natural transformation
\begin{equation}\label{eq:ban_dual}
	\tau(B,C):\lin(B,C)\rightarrow\lin(\dual(C),\dual(B)).
\end{equation}

A similar argument for locally compact abelian groups $G$ and $H$ yields a natural transformation
\begin{equation}\label{eq:grp_char}
	\tau(G,H):\hm(G,M)\rightarrow\hm(\ch(H),\ch(G)).
\end{equation}
In the theory of character groups it is shown that each $\tau(G,H)$ is an isomorphism, so \cref{eq:grp_char}
is actually a natural isomorphism. The well known isomorphism between a locally compact abelian group $G$ and
its twice iterated character group is also a natural isomorphism
\begin{equation*}
	\tau(G):G\rightleftarrows\ch(\ch(G))
\end{equation*}
between functors.\footnote{The proof of naturality appears in the note quoted in \cref{ft:groups42}.}
The analogous natural transformation
\begin{equation*}
	\tau(B):B\rightarrow\dual(\dual(B))
\end{equation*}
for Banach spaces is an equivalence only when $B$ is restricted to the category of reflexive Banach spaces.

\section{Groups as categories}\label{sec:groups_as_cat}
Any group $G$ may be regarded as a category $\mathbf{G}_G$ in which there is only one object. This object may
either be the set $G$ or, if $G$ is a transformation group, the space on which $G$ acts. The mappings of the 
category are to be the elements $\gamma$ of the group $G$, and the product of two elements in the group is to
be their product as mappings in the category. In this category every mapping is an equivalence, and there is
only one identity mapping (the unit element of $G$). A covariant functor $T$ with one argument in $\mathbf{G}_G$
and with values in (the category of) the group $H$ is just a homomorphic mapping $\eta=T(\gamma)$ of $G$ into $H$.
A natural transformation $\tau$ of one such functor $T_1$ into another one, $T_2$, is defined by a single element
$\tau(G)= \eta_0\in H$. Since $\eta_0$ has an inverse, every natural transformation is automatically an equivalence.
The naturality condition \labelcref{eq:nat_eq} for $\tau$ becomes simply $\eta_0 T_1(\gamma)\eta_0^{-1}=T_2(\gamma)$.
Thus the functors $T_1$ and $T_2$ are naturally equivalent if and only if $T_1$ and $T_2$, considered as
homomorphisms, are conjugate.

Similarly, a contravariant functor $T$ on a group $G$, considered as a category, is simply a ``dual'' or ``counter''
homomorphism $(T(\gamma_2\gamma_1)=T(\gamma_1)T(\gamma_2))$.

A ring $R$ with unity also gives a category, in which the mappings are the elements of $R$, under the operation of
multiplication in $R$. The unity of the ring is the only identity of the category, and the units of the ring are
the equivalences of the category.

\section{Construction of functors as transformations}\label{sec:func_trans}

Under suitable conditions a mapping\hyp{}function $\tau(A,B)$ acting on a given functor $T(A,B)$ can be used to 
construct a new functor $S$ such that $\tau:T\rightarrow S$. The case in which each $\tau$ is an equivalence
mapping is the simplest, so will be stated first.
\begin{theorem}\label{thm:func_trans1}
	Let $T$ be a functor covariant in $\mathbf{A}$, contravariant in $\mathbf{B}$, with values in
	$\mathbf{C}$. Let $S$ and $\tau$ be functions which determine for each pair of objects $A\in\mathbf{A},
	B\in\mathbf{B}$ an object $S(A,B)\in\mathbf{C}$ and an equivalence mapping
	\begin{equation*}
		\tau(A,B):T(A,B)\rightarrow S(A,B)\text{ in }\mathbf{C}.
	\end{equation*}
	Then $S$ is the object function of a uniquely determined functor $S$, concordant with $T$ and such that
	$\tau$ is a natural equivalence $\tau:T\rightleftarrows S$.
\end{theorem}
\begin{proof}
	One may readily show that the mapping function appropriate to $S$ is uniquely determined for each
	$\alpha:A_1\rightarrow A_2$ in $\mathbf{A}$ and $\beta:B_1\rightarrow B_2$ in $\mathbf{B}$ by the
	formula
	\begin{equation*}
		S(\alpha,\beta)=\tau(A_2,B_1)T(\alpha,\beta){[\tau(A_1,B_2)]}^{-1}.
	\end{equation*}
\end{proof}

The companion theorem for the case of a transformation which is not necessarily an equivalence is somewhat more
complicated. We first define mappings cancellable from the right. A mapping $\alpha\in\mathbf{A}$ will be
called cancellable from the right if $\beta\alpha=\gamma\alpha$ always implies $\beta=\gamma$. To illustrate,
if each ``formal'' mapping is an actual many\hyp{}to\hyp{}one mapping of one set into another, and if the 
composition of formal mappings is the usual composition of correspondences, it can be shown that every mapping
$\alpha$ of one set \emph{onto} another is cancellable from the right.
\begin{theorem}\label{thm:func_comp}
	Let $T$ be a functor covariant in $\mathbf{A}$ and contravariant in $\mathbf{B}$, with values in
	$\mathbf{C}$. Let $S(A,B)$ and $S(\alpha,\beta)$ be two functions on the objects (and mappings) of
	$\mathbf{A}$ and $\mathbf{B}$, for which it is assumed only, when $\alpha:A_1\rightarrow A_2
	\in\mathbf{A}$ and $\beta:B_1\rightarrow B_2\in\mathbf{B}$, that
	\begin{equation*}
		S(\alpha,\beta):S(A_1,B_2)\rightarrow S(A_2,B_1)\in\mathbf{C}.
	\end{equation*}
	If a function $\tau$ on the objects of $\mathbf{A,B}$ to the mappings of $\mathbf{C}$ satisfies the
	usual conditions for a natural transformation $\tau:T\rightarrow S$; namely that
	\begin{align}
		\tau(A,B):T(A,B)&\rightarrow S(A,B)\in\mathbf{C},\label{eq:functor_comp1}\\
		\tau(A_2,B_1)T(\alpha,\beta)&=S(\alpha,\beta)\tau(A_1,B_2),\label{eq:functor_comp2}
	\end{align}
	and if in addition each $\tau(A,B)$ is cancellable from the right, then the functions $S(\alpha,\beta)$
	and $S(A,B)$ form a functor $S$, concordant with $T$, and $\tau$ is a transformation $\tau:T\rightarrow S$.
\end{theorem}
\begin{proof}
	We need to show that
	\begin{align}
		S(e_A,e_B)&=e_{S(A,B)},\label{eq:func_id}\\
		S(\alpha_2\alpha_1,\beta_2\beta_1)&=S(\alpha_2,\beta_1)S(\alpha_1,\beta_2).\label{eq:func_nat}
	\end{align}
	Since $T$ is a functor, $T(e_A,e_B)$ is an identity, so that the condition \labelcref{eq:functor_comp2} with
	$A_1=A_2$, $B_1=B_2$ becomes
	\begin{equation*}
		\tau(A,B)=S(e_A,e_B)\tau(A,B).
	\end{equation*}
	Because $\tau(A,B)$ is cancellable from the right, it follows that $S(e_A,e_B)$ must be the identity
	mapping of $S(A,B)$, as desired.

	To consider the second condition, let $\alpha:A_1\rightarrow A_2$, $\alpha_2:A_2\rightarrow A_3$,
	$\beta_1:B_1\rightarrow B_2$ and $\beta_2:B_2\rightarrow B_3$, so that $\alpha_2\alpha_1$ and
	$\beta_2\beta_1$ are defined. By condition \labelcref{eq:functor_comp2} and the properties of the
	functor $T$
	\begin{align*}
		S(\alpha_2\alpha_1,\beta_2\beta_1)\tau(A_1,B_3) & = \tau(A_3,B_1)T(\alpha_2\alpha_1,\beta_2\beta_1)\\
				& = \tau(A_3,B_1)T(\alpha_2,\beta_1)T(\alpha_1,\beta_2)\\
				& = S(\alpha_2,\beta_1)\tau(A_2,B_2)T(\alpha_1,\beta_2)\\
				& = S(\alpha_2,\beta_1)S(\alpha_1,\beta_2)\tau(A_1,B_3)
	\end{align*}
	Again because $\tau(A_1,B_3)$ may be cancelled on the right, \cref{eq:func_nat} follows.
\end{proof}

\section{Combination of the arguments of functors}\label{sec:func_comb}
For $n$ given categories $\mathbf{A}_1,\dotsc,\mathbf{A}_n$, the cartesian product category
\begin{equation}\label{eq:prod_cat}
	\mathbf{A}=\prod_i\mathbf{A}_i=\mathbf{A}_1\times\mathbf{A}_2\times\dotsm\times\mathbf{A}_n
\end{equation}
is defined as a category in which the objects are the $n$-tuples of objects $[A_1,\dotsc,A_n]$, with
$A_i\in\mathbf{A}_i$, the mappings are the $n$-tuples $[\alpha_1,\dotsc,\alpha_n]$ of mappings
$\alpha_i\in\mathbf{A}_i$. The product
\begin{equation*}
	[\alpha_1,\dotsc,\alpha_n][\beta_1,\dotsc,\beta_n]=[\alpha_1\beta_1,\dotsc,\alpha_n\beta_n]
\end{equation*}
is defined if and only if each individual product $\alpha_i\beta_i$ is defined in $\mathbf{A}_i$, for
$i\in\{1,\dotsc,n\}$. The identity corresponding to the object $[A_1,\dotsc,A_n]$ in the product category
is to be the mapping $[e(A_1),\dotsc,e(A_n)]$. The axioms which assert that the product $\mathbf{A}$
is a category follow at once. The natural correspondence
\begin{align}
	P(A_1,\dotsc,A_n) &= [A_1,\dotsc,A_n]\label{eq:prod_cat1}\\
	P(\alpha_1,\dotsc,\alpha_n) &= [\alpha_1,\dotsc,\alpha_n]\label{eq:prod_cat2}
\end{align}
is a covariant functor on the $n$ categories $\mathbf{A}_1,\dotsc,\mathbf{A}_n$ to the product category.
Conversely, the correspondences given by ``projection'' into the $i$th coordinate,
\begin{equation}\label{eq:prod_proj}
	Q_i([A_1,\dotsc,A_n])=A_i,\qquad Q_i([\alpha_1,\dotsc,\alpha_n])=\alpha_i,
\end{equation}
is a covariant functor in one argument, on $\mathbf{A}$ to $\mathbf{A}_i$.

It is now possible to represent a functor covariant in any number of arguments as a functor in one argument.
Let $T$ be a functor on the categories $\mathbf{A}_1,\dotsc,\mathbf{A}_n,\mathbf{B}$, with the same variance
in $\mathbf{A}_i$, as in $\mathbf{A}_1$; define a new functor $T^\ast$ by setting
\begin{align*}
	T^\ast\left([A_1,\dotsc,A_n],B\right) &= T(A_1,\dotsc,A_n,B),\\
	T^\ast\left([\alpha_1,\dotsc,\alpha_n],\beta\right) &= T(\alpha_1,\dotsc,\alpha_n,\beta).
\end{align*}
This is a functor, since it is a composite of $T$ and the projections $Q_i$ of \cref{eq:prod_proj};
its variance in the first argument is that of $T$ in any $A_i$. Conversely, each functor $S$ with arguments in
$\mathbf{A}_i\times\dotsm\times\mathbf{A}_n$ and $\mathbf{B}$ can be represented as $S=T^\ast$, for a $T$ with
$n+1$ arguments in $\mathbf{A}_i,\dotsc,\mathbf{A}_n,\mathbf{B}$, defined by
\begin{align*}
	T(A_1,\dotsc,A_n,B) &=   S([A_1,\dotsc,A_n],B) = S(P(A_1,\dotsc,A_n),B),\\
	T(\alpha_1,\dotsc,\alpha_n,\beta) &=   S([\alpha_1,\dotsc,\alpha_n],\beta) = S(P(\alpha_1,\dotsc,\alpha_n),\beta).
\end{align*}
Again $T$ is a composite functor. These reduction arguments combine to give the following theorem.
\begin{theorem}
	For given categories $\mathbf{A}_1,\dotsc,\mathbf{A}_n,\mathbf{B}_1,\dotsc,\mathbf{B}_m,\mathbf{C}$,
	there is a one\hyp{}to\hyp{}one correspondence between the functors $T$ covariant in $\mathbf{A}_1,\dotsc,\mathbf{A}_n$,
	contravariant in $\mathbf{B}_1,\dotsc,\mathbf{B}_m$, with values in $\mathbf{C}$, and the functors $S$ in two arguments,
	covariant in $\mathbf{A}_1\times\dotsm\times\mathbf{A}_n$ and $\mathbf{B}_1\times\dotsm\times\mathbf{B}_m$, with values
	in the same category $\mathbf{C}$. Under this correspondence, equivalent functors $T$ correspond to equivalent functors
	$S$, and a natural transformation $\tau:T_1\rightarrow T_2$ gives rise to a natural transformation $\sigma:S_1\rightarrow S_2$
	between the functors $S_1$ and $S_2$ corresponding to $T_1$ and $T_2$ respectively.
\end{theorem}

By this theorem, all functors can be reduced to functors in two arguments. To carry this reduction further, we introduce the concept
of ``dual'' category.

Given a category $\mathbf{A}$, the dual category $\mathbf{A}^\text{op}$ is defined as follows. The objects of $\mathbf{A}^\text{op}$ are those
of $\mathbf{A}$; the mappings $\alpha^\text{op}$ of $\mathbf{A}^\text{op}$ are in one\hyp{}to\hyp{}one correspondence $\alpha\rightleftarrows\alpha^\text{op}$
with the mappings of $\mathbf{A}$. If $\alpha:A_1\rightarrow A_2$ in $\mathbf{A}$, then $\alpha^\text{op}:A_2\rightarrow A_1$ in $\mathbf{A}^\text{op}$.
The composition law is defined by the equation
\begin{equation*}
	\alpha_2^\text{op}\alpha_1^\text{op}= {(\alpha_1\alpha_2)}^\text{op},
\end{equation*}
if $\alpha_1\alpha_2$ is defined in $\mathbf{A}$. We verify that $\mathbf{A}^\text{op}$ is a category and that there are equivalences
\begin{equation*}
	{(\mathbf{A}^\text{op})}^\text{op}\cong\mathbf{A},\qquad \prod_i\mathbf{A}_i^\text{op}\cong{\left(\prod\mathbf{A}_i\right)}^\text{op}.
\end{equation*}
The mapping
\begin{equation*}
	D(A)= A,\qquad D(\alpha)= \alpha^\text{op}
\end{equation*}
is a contravariant functor on $\mathbf{A}$ to $\mathbf{A}^\text{op}$, while $D^{-1}$ is contravariant on $\mathbf{A}^\text{op}$ to $\mathbf{A}$.

Any contravariant functor $T$ on $\mathbf{A}$ to $\mathbf{C}$ can be regarded as a covariant functor $T^\text{op}$ on $\mathbf{A}^\text{op}$ to
$\mathbf{C}$, and vice versa. Explicitly, $T^\text{op}$ is defined as a composite
\begin{equation*}
	T^\text{op}(A)= T(D^{-1}(A)),\qquad  T^\text{op}(\alpha^\text{op})= T(D^{-1}(\alpha^\text{op})).
\end{equation*}
\begin{theorem}
	Every functor $T$ covariant on $\mathbf{A}_1,\dotsc,\mathbf{A}_n$ and contravariant on $\mathbf{B}_1,\dotsc,\mathbf{B}_m$ with
	values in $\mathbf{C}$ may be regarded as a covariant functor $T'$ on
	\begin{equation*}
		\left(\prod_i\mathbf{A}_i\right)\times\left(\prod_j\mathbf{B}_j^\text{op}\right)
	\end{equation*}
	with values in $\mathbf{C}$, and vice versa. Each natural transformation (or equivalence) $\tau:T_1\rightarrow T_2$ yields a
	corresponding transformation (or equivalence) $\tau':T_1'\rightarrow T_2'$.
\end{theorem}

\chapter{Functors and groups}\label{ch:funct_groups}
\section{Subfunctors}\label{sec:subfun}
This chapter will develop the fashion in which various particular properties of groups are reflected by properties of functors with
values in a category of groups. The simplest such case is the fact that subgroups can give rise to ``subfunctors''. The concept of
subfunctor thus developed applies with equal force to functors whose values are in the category of rings, spaces, and so on.

In the category $\mathbf{G}$ of all topological groups we say that a mapping $\gamma':G_1'\rightarrow G_2'$ is a \emph{submapping}
of a mapping $\gamma:G_1\rightarrow G_2$ (notation: $\gamma'\subset\gamma$) whenever $G_1'\subset G_1$, $G_2'\subset G_2$ and
$\gamma'(g_1)=\gamma(g_1)$ for each $g_1\in G_1'$. Here $G_1'\subset G_1$ means of course that $G_1'$ is a subgroup (not just a
subset) of $G_1$.

Given two concordant functors $T'$ and $T$ on $\mathbf{A}$ and $\mathbf{B}$ to $\mathbf{C}$, we say that $T'$ is a subfunctor
of $T$ (notation: $T'\subset T$) provided $T'(A,B)\subset T(A,B)$ for each pair of objects $A\in\mathbf{A}$, $B\in\mathbf{B}$ and
$T'(\alpha,\beta)\subset T(\alpha,\beta)$ for each pair of mapping $\alpha\in\mathbf{A}$, $\beta\in\mathbf{B}$. Clearly $T'\subset
T$ and $T\subset T'$ imply $T=T'$; furthermore this inclusion satisfies the transitive law. If $T'$ and $T''$ are both subfunctors of
the same functor $T$, then in order to prove that $T'\subset T''$ it is sufficient to verify that $T'(A,B)\subset T''(A,B)$ for all
$A$ and $B$.

A subfunctor can be completely determined by giving its object function alone. The requisite properties for this object function may
be specified as:
\begin{theorem}\label{thm:subfunc}
	Let the functor $T$ covariant in $\mathbf{A}$ and contravariant in $\mathbf{B}$ have values in the category $\mathbf{G}$
	of groups, while $T'$ is a function which assigns to each pair of objects $A\in\mathbf{A}$ and $B\in\mathbf{B}$ a subgroup
	$T'(A,B)$ of $T(A,B)$. Then $T'$ is the object function of a subfunctor of $T$ if and only if for each $\alpha:A_1\rightarrow
	A_2\in\mathbf{A}$ and each $\beta: B_1\rightarrow B_2\in\mathbf{B}$ the mapping $T(\alpha,\beta)$ carries the subgroup
	$T'(A_1,B_2)$ into part of $T'(A_2,B_1)$. If $T'$ satisfies this condition, the corresponding mapping function is uniquely
	determined.
\end{theorem}
\begin{proof}
	The necessity of this condition is immediate. Conversely, to prove the sufficiency, we define for each $\alpha$ and $\beta$ a
	homomorphism $T'(\alpha,\beta)$ of $T'(A_1,B_2)$ into $T'(A_2,B_1)$ by setting $T'(\alpha,\beta)g=T(\alpha,\beta)g$, for each
	$g\in T'(A_1,B_2)$. The fact that $T'$ satisfies the requisite conditions for the mapping function of a functor is then
	immediate, since $T'$ is obtained by ``cutting down'' $T$.
\end{proof}

The concept of a subtransformation may also be defined. If $T,S,T',S'$ are concordant functors on $\mathbf{A,B}$ to $\mathbf{C}$,
and if $\tau:T\rightarrow S$ and $\tau':T'\rightarrow S'$ are natural transformations, we say that $\tau'$ is a
\emph{subtransformation} of $\tau$ (notation: $\tau'\subset\tau$) if $T'\subset T$, $S'\subset S$ and if, for each pair of
arguments $A,B,\tau'(A,B)$ is a submapping of $\tau(A,B)$. Any such subtransformation of $\tau$ may be obtained by suitably
restricting both the domain and the range of $\tau$. Explicitly, let $\tau:T\rightarrow S$, let $T'\subset T$ and $S'\subset S$
be such that for each $A,B,\tau(A,B)$ maps the subgroup $T'(A,B)$ of $T(A,B)$ into the subgroup $S'(A,B)$ of $S(A,B)$.
If then $\tau'(A,B)$ is defined as the homomorphism $\tau(A,B)$ with its domain restricted to the subgroup $T'(A,B)$ and its
range to the subgroup $S'(A,B)$, it follows readily that $\tau'$ is indeed a natural transformation  $\tau':T'\rightarrow S'$.

Let $\tau$ be a natural transformation $\tau:T\rightarrow S$ of concordant functors $T$ and $S$ on $\mathbf{A}$ and $\mathbf{B}$
to the category $\mathbf{G}$ of groups. If $T'$ is a subfunctor of $T$, then the map of each $T'(A,B)$ under $\tau(A,B)$ is a
subgroup of $S(A,B)$, so that we may define an object function
\begin{equation*}
	S'(A,B)= \tau(A,B)[T'(A,B)],\qquad A\in\mathbf{A},B\in\mathbf{B}.
\end{equation*}
The naturality condition on $\tau$ shows that the function $S'$ satisfies the condition of \cref{thm:subfunc}; hence $S'=\tau T'$
gives a subfunctor of $S$, called the $\tau$-transformation of $T'$. Furthermore there is a natural transformation
$\tau':T'\rightarrow S'$ obtained by restricting $\tau$. In particular, if $\tau$ is a natural equivalence, so is $\tau'$.

Conversely, for a given $\tau:T\rightarrow S$ let $S''$ be a subfunctor of $S$. The inverse image of each subgroup $S''(A,B)$
under the homomorphism $\tau(A,B)$ is then a subgroup of $T(A,B)$, hence gives an object function
\begin{equation*}
	T''(A,B)=\tau{(A,B)}^{-1}[S''(A,B)],\qquad A\in\mathbf{A},B\in\mathbf{B}.
\end{equation*}
As before this is the object function of a subfunctor $T''\subset T$ which may be called the inverse transform
$\tau^{-1}S''=T''$ of $S''$. Again, $\tau$ ma be restricted to give a natural transformation $\tau'':T''\rightarrow S''$.
In case each $\tau(A,B)$ is homomorphism of $T(A,B)$ \emph{onto} $S(A,B)$, we may assert that $\tau(\tau^{-1}) S''=S''$.

Lattice operations on subgroups can be applied to functors. If $T'$ and $T''$ are two subfunctors of a functor $T$ with
values in $G$, we define their meet $T'\wedge T''$ and their join $T'\vee T''$ by giving object functions,
\begin{align*}
	[T'\wedge T''](A,B) & =  T'(A,B)\wedge T''(A,B),\\
	[T'\vee T''](A,B) & =  T'(A,B)\vee T''(A,B).
\end{align*}
We verify that the condition of \cref{thm:subfunc} is satisfied here, so that these object functions do uniquely determine
corresponding subfunctors of $T$. Any lattice identity for groups may then be written directly as an identity for the 
subfunctors of a fixed functor $T$ with values in $\mathbf{G}$.

\section{Quotient functors}\label{sec:quot_funct}
The operation of forming a quotient group leads to an analogous operation of taking the ``quotient functor'' of a
functor $T$ by a ``normal'' subfunctor $T'$. If $T$ is a functor covariant in $\mathbf{A}$ and contravariant in
$\mathbf{B}$, with values in $\mathbf{G}$, a \emph{normal functor} $T'$ will mean a subfunctor $T'\subset T$
such that each $T'(A,B)$ is a normal subgroup of $T(A,B)$, while a \emph{closed} subfunctor $T'$ will be one
in which each $T'(A,B)$ is a closed subgroup of the topological group $T(A,B)$. If $T'$ is a normal subfunctor
of $T$, the quotient functor $Q= T/T'$ has an object function given as the factor group,
\begin{equation*}
	Q(A,B)= T(A,B)/T'(A,B).
\end{equation*}
For homomorphisms $\alpha:A_1\rightarrow A_2$ and $\beta:B_1\rightarrow B_2$ the corresponding mapping function
$Q(\alpha,\beta)$ is defined for each coset\footnote{For convenience of notation we write the group operations 
(commutative or not) with a plus sign.} $x+T'(A_1,B_2)$ as
\begin{equation*}
	Q(\alpha,\beta)[x+T'(A_1,B_2)]= [T(\alpha,\beta)x]+T'(A_2,B_1).
\end{equation*}

Before we prove that $Q$ is actually a functor, we introduce for each $A\in\mathbf{A}$ and $B\in\mathbf{B}$
the homomorphism
\begin{equation*}
	\nu(A,B):T(A,B)\rightarrow Q(A,B)
\end{equation*}
defined for each $x\in T(A,B)$ by the formula
\begin{equation*}
	\nu(A,B)(x) = x + T'(A,B).
\end{equation*}
When $\alpha:A_1\rightarrow A_2$ and $\beta:B_1\rightarrow B_2$ we now show that
\begin{equation*}
	Q(\alpha,\beta)\nu(A_1,B_2) = \nu(A_2,B_1)T(\alpha,\beta).
\end{equation*}
For, given any $x\in T(A_1,B_2)$, the definitions of $\nu$ and $Q$ give at once
\begin{align*}
	Q(\alpha,\beta)[\nu(A_1,B_2)(x)] & = Q(\alpha,\beta)[x+T'(A_1,B_2)]\\
	&= [T(\alpha,\beta)(x)]+T'(A_2,B_1)\\
	&= \nu(A_2,B_1)[T(\alpha,\beta)(x)].
\end{align*}
Notice also that $\nu(A,B)$ maps $T(A,B)$ \emph{onto} the factor group $Q(A,B)$, hence is cancellable from the right.
Therefore, \cref{thm:subfunc} shows that $Q=T/T'$ is a functor, and that $\nu$ is a natural transformation of $T$
onto $T/T'$.

In particular, if the functor $T$ has its values in the category of regular topological groups, while $T'$ is a 
\emph{closed} normal subfunctor of $T$, the quotient functor $T/T'$ has its values in the same category of groups,
since a quotient of a regular topological group by a \emph{closed} subgroup is again regular.

To consider the behavior of quotient functors under natural transformations we first recall some properties of
homomorphisms. Let $\alpha:G\rightarrow H$ be a homomorphism of the group $G$ into $H$, while $\alpha':G'\rightarrow H'$
is a submapping of $\alpha$, with $G'$ and $H'$ normal subgroups of $G$ and $H$, respectively, and $\nu$ and $\mu$ are
the natural homomorphisms $\nu:G\rightarrow G/G'$, $\mu:H\rightarrow H/H'$. Then we may define a homomorphism 
$\beta:G/G'\rightarrow H/H'$ be setting $\beta(x+G')=\alpha x + H'$ for each $x\in G$. This homomorphism is the only
mapping of $G/G'$ into $H/H'$ with the property that $\beta\nu=\mu\alpha$, as indicated in the figure
\begin{equation*}
	\begin{tikzpicture}
		\node (a) at (0,0)
		{ \begin{tikzcd}[sep=huge]
		G \arrow[r,"\alpha"] \arrow[d, "\nu"] & H\arrow[d,"\mu"]\\
		G/G' \arrow[r,"\beta"] & H/H'
		\end{tikzcd}};
	\end{tikzpicture}
\end{equation*}
We may write $\beta=\alpha/\alpha'$. The corresponding statement for functors is as follows:
\begin{theorem}\label{thm:grp_mod}
	Let $\tau:T\rightarrow S$ be a natural transformation between functors with values in $\mathbf{G}$;
	and let $\tau':T'\rightarrow S'$ be a subtransformation of $\tau$ such that $T'$ and $S'$ are normal
	subfunctors of $T$ and $S$ respectively. Then the definition of $\rho(A,B)=\tau(A,B)/\tau'(A,B)$
	gives a natural transformation $\rho=\tau/\tau'$,
	\begin{equation*}
		\rho:T/T'\rightarrow S/S'.
	\end{equation*}
	Furthermore, $\rho\nu=\mu\tau$, where $\nu$ is the natural transformation $\nu:T\rightarrow T/T'$ and
	$\mu$ is the natural transformation $\mu:S\rightarrow S/S'$.
\end{theorem}
\begin{proof}
	This requires only the verification of the naturality condition for $\rho$, which follows at once from
	the relevant definitions.
\end{proof}

The ``kernel'' of a transformation appears as a special case of this theorem. Let $\tau:T\rightarrow S$ be
given, and take $S'$ be the identity\hyp{}element subfunctor of $S$; that is, let each $S'(A,B)$ be the
subgroup consisting only of the identity (zero) element of $S(A,B)$. Then the inverse transformation 
$T'=\tau^{-1}S'$ is by \cref{sec:subfun} a (normal) subfunctor of $T$, and $\tau$ may be restricted to give
the natural transformation $\tau':T'\rightarrow S'$. We may call $T'$ the kernel functor of the transformation
$\tau$. \Cref{thm:grp_mod} applied in this case shows that there is then a natural transformation $\rho:T/T'
\rightarrow S$ such that $\rho=\tau\nu$. Furthermore each $\rho(A,B)$ is a one\hyp{}to\hyp{}one mapping of the
quotient group $T(A,B)/T'(A,B)$ into $S(A,B)$. If in addition we assume that each $\tau(A,B)$ is an open mapping
of $T(A,B)$ \emph{onto} $S(A,B)$, we may conclude, exactly as in group theory, that $\rho$ is a natural equivalence.

\section{Examples of subfunctors}\label{sec:ex_funct}
Many characteristic subgroups of a group may be written as subfunctors of the identity functor. The (covariant)
identity functor $I$ on $\mathbf{G}$ to $\mathbf{G}$ is defined by setting
\begin{equation*}
	I(G)= G,\qquad I(\gamma)= \gamma.
\end{equation*}
Any subfunctor of $I$, is by \cref{thm:subfunc}, determined by an object function
\begin{equation*}
	T(G)\subset G
\end{equation*}
such that whenever $\gamma$ maps $G_1$ homomorphically into $G_2$, then $\gamma[T(G_1)]\subset T(G_2)$.
Furthermore, if each $T(G)$ is a normal subgroup of $G$, we can form the quotient functor $I/T$.

For example, the commutator subgroup $C(G)$ of the group $G$ determines in this fashion a normal subfunctor
of $I$. The corresponding quotient functor $(I/C)(G)$ is the functor determining for each $G$ the factor 
commutator group of $G$ (the group $G$ made abelian).

The center $Z(G)$ does no determine in this fashion a subfunctor of $I$, because a homomorphism of $G_1$
\emph{into} $G_2$ ma carry central elements of $G_1$ into non\hyp{}central elements of $G_2$. However, we
may choose to restrict the category $\mathbf{G}$ by using as mappings only homomorphisms of one group
\emph{onto} another. For \emph{this} category, $Z$ is a subfunctor of $I$, and we may form a quotient 
functor $I/Z$.

Thus various types of subgroups of $G$ may be classified in terms of the degree of invariance of the
``subfunctors'' of the identity which they generate. This classification is similar to, but not 
identical with, the known distinction between normal subgroups, characteristic subgroups, and strictly
characteristic subgroups of a single group.\footnote{A subgroup $S$ of $G$ is characteristic if 
$\sigma_1(S)\subset S$ for every automorphism $\sigma_1$ of $G$, and strictly (or ``strongly'') characteristic
if $\sigma_2(S)\subset S$ for every endomorphism $\sigma_2$ of $G$.} The present distinction by functors
refers not to the subgroups of an individual group, but to a definition yielding a subgroup for each
of the groups in a suitable category. It includes the standard distinction, in the sense that one may
consider functors on the category with only one object (a single group $G$) and with mappings which are
the inner automorphisms of $G$ (the subfunctors of $I=\text{normal subgroups}$), the automorphisms of
$G$ ($\text{subfunctors}=\text{characteristic subgroups}$), or the endomorphisms of $G$ ($\text{subfunctors}
=\text{strictly characteristic subgroups}$).

Still another example of the degree of invariance is given by the automorphism group $A(G)$ of a group $G$.
This is a functor $A$ defined on the category $\mathbf{G}$ of groups with the mappings restricted to the
isomorphisms $\gamma:G_1\rightarrow G_2$ of one group onto another. The mapping function $A(\gamma)$ for any
automorphism $\sigma_1$ of $G_1$ is then defined by setting
\begin{equation*}
	[A(\gamma)\sigma_1]g_2 =  \gamma\sigma_1\gamma^{-1}g_2,\qquad g_2\in G_2.
\end{equation*}

The types of invariance for functors on $\mathbf{G}$ may thus be indicated by a table, showing how the mappings
of the category must be restricted in order to make the indicated set function a functor:
\begin{center}
	\begin{tabular}{l l}
	\textit{Functor} & \textit{Mappings} $\gamma:G_1\rightarrow G_2$\\
	$C(G)$	& Homomorphisms into,\\
	$Z(G)$	& Homomorphisms onto,\\
	$A(G)$	& Isomorphisms onto.\\
	\end{tabular}
\end{center}

For the subcategory of $\mathbf{G}$ consisting of all (additive) abelian groups there are similar subfunctors
\begin{enumerate}
	\item $G_0$, the set of all elements of finite order in $G$;
	\item $G_m$, the set of all elements of $G$ of order dividing the integer $m$;
	\item $mG$, the set of all elements of the form $m g\in G$.
\end{enumerate}
The corresponding quotient functors will have object functions $G/G_0$ (the ``Betti group'' of $G$), $G/G_m$
and $G/mG$ (the group $G$ reduced modulo $m$).

\section{The isomorphism theorems}\label{sec:iso_thm}
The isomorphism theorems of group theory can be formulated for functors; from this it will follow that these
isomorphisms between groups are ``natural''.

The ``first isomorphism theorem'' asserts that if $G$ has two normal subgroups $G_1$ and
$G_2$ with $G_2 \subset G_1$, then $G_1/G_2$ is a normal subgroup of $G/G_2$, and there is an 
isomorphisms $\tau:(G/G_2)/(G_1/G_2)\rightarrow G/G_1$. The elements of the first group (in additive notation)
are coset of coset, of the form $(x+G_2)+G_1/G_2$, and the isomorphism $\tau$ is defined as
\begin{equation}\label{eq:first_iso}
	\tau[(x+G_2)+G_1/G_2]= x+G_1.
\end{equation}
This may be stated in terms of functors as follows.
\begin{theorem}\label{thm:first_iso_fun}
	Let $T_1$ and $T_2$ be two normal subfunctors of a functor $T$ with values in the category of groups
	$\mathbf{G}$. If $T_2\subset T_1$, then $T_1/T_2$ is a normal subfunctor of $T/T_2$ and the functors
	\begin{equation}\label{eq:grp_nateq}
		T/T_1\quad\text{and}\quad (T/T_2)/(T_1/T_2)
	\end{equation}
	are naturally equivalent.
\end{theorem}
\begin{proof}
	We assume that the given functor $T$ depends on the usual typical arguments $A$ and $B$. Since
	$(T_1/T_2)(A,B)$ is clearly a normal subgroup of $(T/T_2)(A,B)$, a proof that $T_1/T_2$ is a normal
	subfunctor of $T/T_2$ requires only a proof that each $(T_1/T_2)(\alpha,\beta)$, is a submapping of
	the corresponding $(T/T_2)(\alpha,\beta)$ for any $\alpha:A_1\rightarrow A_2$ and $\beta:B_1\rightarrow B_2$.
	To show this, apply $(T_1/T_2)(\alpha,\beta)$ to a typical coset $x+T_2(A_1,B_2)$. Applying the
	definitions, one has
	\begin{align*}
		(T_1/T_2)(\alpha,\beta)[x+T_2(A_1,B_2) &=(T_1)(\alpha,\beta)(x)+T_2(A_2,B_1)\\
						&=(T)(\alpha,\beta)(x)+T_2(A_2,B_1)\\
						&=(T/T_2)(\alpha,\beta)[x+T_2(A_2,B_1)],
	\end{align*}
	for  $T_1(\alpha,\beta)$ was assumed to be a submapping of $T(\alpha,\beta)$.
	The asserted equivalence \labelcref{eq:grp_nateq} is established by setting, as in \cref{eq:first_iso},
	\begin{equation*}
		\tau(A,B)\{[x+T_2(A,B)]+(T_1/T_2)(A,B)\}=x+T_1(A,B).
	\end{equation*}
	The naturality proof then requires that, for any mappings $\alpha:A_1\rightarrow A_2$ and 
	$\beta:B_1\rightarrow B_2$
	\begin{equation*}
		\tau(A_2,B_1)S(\alpha,\beta)=(T/T_1)(\alpha,\beta)\tau(A_1,B_2),
	\end{equation*}
	where $S=(T/T_2)/(T_1/T_2)$. This equality may be verified mechanically by applying each side to a
	general element $[x+T_2(A_1,B_2)]+(T_1/T_2)(A_1,B_2)$ in the group $S(A,B)$.
\end{proof}

The theorem may also be stated and proved in the following equivalent form:
\begin{theorem}
	Let $T'$ and $T''$ be two normal subfunctors of a functor $T$ with values in the category $\mathbf{G}$ of groups.
	Then $T'\wedge T''$ is a normal subfunctor of $T'$ and of $T$, $T'/T'\wedge T''$ is a normal subfunctor of
	$T/T'\wedge T''$, and the functors
	\begin{equation}\label{eq:grp-lattice}
		T/T'\qquad \text{and}\qquad (T/T'\wedge T'')/(T'/T'\wedge T'')
	\end{equation}
	are naturally equivalent.
\end{theorem}
\begin{proof}
	Set $T_1=T'$, $T_2=T'\wedge T''$.
\end{proof}

The second isomorphism theorem for groups is fundamental in the proof of the Jordan\hyp{}H{\"o}lder Theorem. It states
that if $G$ has normal subgroups $G_1$ and $G_2$, then $G_1\wedge G_2$ is a normal subgroup of $G_1$, $G_2$ is a normal 
subgroup of $G_1\vee G_2$, and there is an isomorphism $\mu$ of $G_1/(G_1\wedge G_2)$ to $(G_1\vee G_2)/G_2$. (Because
$G_1$ and $G_2$ are normal subgroups, the join $G_1\vee G_2$ consists of all ``sums'' $g_1+g_2$, for $g_i\in G_i$, so
is often written as $G_1\vee G_2= G_1+G_2$.) For any $x\in G_1$, this isomorphism is defined as
\begin{equation}\label{eq:second-iso}
	\mu[x+(G_1\wedge G_2)]= x + G_2.
\end{equation}
The corresponding theorem for functors reads:
\begin{theorem}
	If $T_1,T_2$ are normal subfunctors of a functor $T$ with values in $\mathbf{G}$, then $T_1\wedge T_2$ is a normal
	subfunctor of $T_1$, and $T_2$ is a normal subfunctor of $T_1\vee T_2$, and the quotient functors
	\begin{equation}\label{eq:func-lattice}
		T_1/(T_1\wedge T_2)\qquad\text{and}\qquad (T_1\vee T_2)/T_2
	\end{equation}
	are naturally equivalent.
\end{theorem}
\begin{proof}
	It is clear that both quotients in \cref{eq:func-lattice} are functors. The requisite equivalence $\mu(A,B)$ is given,
	as in \cref{eq:second-iso}, by  the definition
	\begin{equation*}
		\mu(A,B)[x+(T_1(A,B)\wedge T_2(A,B))] = x + T_2(A,B),
	\end{equation*}
	for any $x\in T_1(A,B)$. The naturality may be verified as before.
\end{proof}

From these theorems we may deduce that the first and second isomorphism theorems yield natural isomorphisms between groups
in another and more specific way. To this end we introduce an appropriate category $\mathbf{G}^\ast$. An object of 
$\mathbf{G}^\ast$ is to be a triple $G^\ast=[G,G',G'']$ consisting of a group $G$ and two of its normal subgroups. A mapping
$\gamma:[G_1,G_1',G_1'']\rightarrow [G_2,G_2',G_2'']\in\mathbf{G}^\ast$ is to be a homomorphism $\gamma:G_1\rightarrow G_2$
with the special properties that $\gamma(G_1')\triangleleft G_2'$ and  $\gamma(G_1'')\triangleleft G_2''$. It is clear that 
these definitions do yield a category $\mathbf{G}^\ast$. on this category $\mathbf{G}^\ast$ we may defined three (covariant)
functors with values in the category $\mathbf{G}$ of groups. The first is a ``projection'' functor
\begin{equation*}
	P([G,G',G''])= G\qquad P(\gamma)= \gamma;
\end{equation*}
the others are two normal subfunctors of $P$, which may be specified by their object functions as
\begin{equation*}
	P'([G,G',G''])= G',\qquad P''([G,G',G''])= G''.
\end{equation*}

Consider now the first isomorphism theorem, in the second form,
\begin{equation}\label{eq:second-iso2}
	G/G'\cong (G/(G'\wedge G''))/(G'/(G'\wedge G'')).
\end{equation}
If we set $G^\ast=[G,G',G'']$, the left side here is a value of the object function of the functor, $P/P'$, and the right
side is similarly a value of $(P/(P'\wedge P''))/(P'/(P'\wedge P''))$. \Cref{thm:first_iso_fun} asserts that these two functors
are indeed naturally equivalent. Therefore, the isomorphism \labelcref{eq:second-iso2} is itself natural, in that it can be
regarded as natural isomorphism between the object functions of suitable functors on the category $\mathbf{G}^\ast$.

The second isomorphism theorem
\begin{equation*}
	(G'\vee G'')/G''\cong G'/(G'\wedge G'')
\end{equation*}
is natural in a similar sense, for both sides can be regarded as object functions of suitable (covariant) functors on $\mathbf{G}^\ast$.

It is clear that this technique of constructing a suitable category $\mathbf{G}^\ast$ could be used to establish the naturality of
even more complicated ``isomorphism'' theorems.

\section{Direct product of functors}\label{sec:funct_prod}
We recall that there are essentially two different ways of defining the direct product of two groups $G$ and $H$. The ``external''
direct product $G\times H$ is the group of all pairs $(g,h)$ with $g\in G, h\in H$, with the usual multiplication. This product
$G\times H$ contains a subgroup $G'$, of all pairs $(g,0)$, which is isomorphic to $G$, and a subgroup $H'$ isomorphic to $H$.
Alternatively, a group $L$ with subgroups $G$ and $H$ is said to be the ``internal'' direct product $L=G\boxtimes H$ of its subgroup
$G$ and $H$ if $gh=hg$ for every $g\in G, h\in H$ and if every element in $L$ can be written uniquely as a product $gh$ with
$g\in G,h\in H$. The intimate connection between the two types of direct products is provided by the isomorphisms $G\times H\cong
G\boxtimes H$ and by the equality $G\times H=G'\boxtimes H'$, where $G'\cong G,H'\cong H$. 

As in \cref{sec:funct_examples}, the \emph{external} direct product can be regarded as a covariant functor on $\mathbf{G}$,
and $\mathbf{G}$ to $\mathbf{G}$, with object function $G\times H$, and mapping function $\gamma\times\eta$, defined as in
\cref{sec:funct_examples}.

Direct products of functors may also be defined, with the same distinction between ``external'' and ``internal'' products. We
consider throughout functors covariant in a category $\mathbf{A}$, contravariant in $\mathbf{B}$, with values in the category
$\mathbf{G}_0$ of discrete groups. If $T_1$ and $T_2$ are two such functors, the external direct product is a functor $T_1\times T_2$
for which the object and mapping functions are respectively
\begin{align}
	(T_1\times T_2)(A,B) &= T_1(A,B)\times T_2(A,B),\label{eq:func_ext1}\\
	(T_1\times T_2)(\alpha,\beta) &= T_1(\alpha,\beta)\times T_2(\alpha,\beta),\label{eq:func_ext2}
\end{align}
If $T_1'(A,B)$  denotes the set of all pairs $(g,0)$ in the direct product $T_1(A,B)\times T_2(A,B)$, $T_1'$ is a subfunctor of
$T_1\times T_2$, and the correspondence $g\rightarrow (g,0)$ provides a natural isomorphism of $T_1$ to $T_1'$. Similarly $T_2$
is naturally isomorphic to a subfunctor $T_2'$ of $T_1\times T_2$.

On the other hand, let $S$ be a functor on $\mathbf{A,B}$ to $\mathbf{G}_0$ with subfunctors $S_1$ and $S_2$. We call $S$ the
\emph{internal} direct product $S_1\boxtimes S_2$ if, for each $A\in\mathbf{A}$ and $B\in\mathbf{B}$, $S(A,B)$ is the internal
direct product $S_1(A,B)\boxtimes S_2(A,B)$. From this definition it follows that, whenever $\alpha:A_1\rightarrow A_2$ and
$\beta:B_1\rightarrow B_2$ are given mappings and $g_i\in S_i(A_1,B_2)$ are given elements (for $i\in\{1,2\}$), then, since
$S_i(\alpha,\beta)\subset S(\alpha,\beta)$
\begin{equation*}
	S(\alpha,\beta)g_1 g_2 = [S_1(\alpha,\beta)g_1][S_2(\alpha,\beta)g_2].
\end{equation*}
This means that the correspondence $\tau$ defined by setting $[\tau(A_1,B_2)](g_1 g_2)= g_2$ is a natural transformation 
$\tau:S\rightarrow S_2$. Furthermore this transformation is idempotent, for $\tau(A_1,B_2)\tau(A_1,B_2)=\tau(A_1,B_2)$.

The connection between the two definitions is immediate; there is a natural isomorphism of the internal direct product
$S_1\boxtimes S_2$ to the external direct product $S_1\times S_2$; furthermore any external product $T_1\times T_2$ is the
internal product $T_1'\boxtimes T_2'$ of its subfunctors $T_1'\cong T_1, T_2'\cong T_2$.

There are in group theory various theorems giving direct product decompositions. These decompositions can now be classified
as to ``naturality''. Consider for example the theorem that every finite abelian group $G$ can be represented as the 
(internal) direct product of its Sylow subgroups. This decomposition is ``natural''; specifically, we may regard the Sylow
subgroup $S_p(G)$ (the subgroup consisting of all elements in $G$ of order some power of the prime $p$) as the object function
of a subfunctor $S_p$ of the identity. The theorem in question then asserts in effect that the identity functor $I$ is the
internal direct product of (a finite number of) the functors $S_p$. This representation of the direct factors by functors
is the underlying reason for the possibility of extending the decomposition theorem in question to infinite groups in which
every element has finite order.

On the other hand consider the theorem which asserts that every finite abelian group is the direct product of cyclic
subgroups. It is clear here that the subgroups cannot be given as the values of functors, and we observe that in this
case the theorem does not extend to infinite abelian groups.

As another example of non\hyp{}naturality, consider the theorem which asserts that any abelian group $G$ with a finite
number of generators can be represented as a direct product of a free abelian group by the subgroup $T(G)$ of all 
elements of finite order in $G$. Let us consider the category $\mathbf{G}_\text{af}$ of all discrete abelian groups
with a finite number of generators. In this category the ``torsion'' subgroup $T(G)$ does determine the object function
of a subfunctor $T\subset I$. However, there is no such functor giving the complementary direct factor of $G$.
\begin{theorem}
	In the category $\mathbf{G}_\textrm{af}$ there is no subfunctor $F\subset I$ such that $I=F\boxtimes T$, that is 
	such that, for all $G$
	\begin{equation}\label{eq:grp-af}
		G=F(G)\boxtimes T(G).
	\end{equation}
\end{theorem}
\begin{proof}
	It suffices to consider just one group, such as the group $G$ which is the (external) direct product of the additive
	group if integers and the additive group of integers mod $m$, for $m\neq 0$. Then no matter which free subgroup 
	$F(G)$ may be chosen so that \cref{eq:grp-af} holds for this $G$, there clearly is an isomorphism of $G$ to $G$
	which does not carry $F$ into itself. Hence $F$ cannot be a functor.
\end{proof}

This result could also be formulated in the statement that, for any $G$ with $G\neq T(G)\neq\{0\}$, there is no decomposition
\labelcref{eq:grp-af} with $F(G)$ a (strongly) characteristic subgroup of $G$. In order to have a situation which cannot be
reformulated in this way, consider the closely related (and weaker) group theoretic theorem which asserts that for each
$G\in\mathbf{G}_\text{af}$ there is an isomorphism of $G/T(G)$ into $G$. This isomorphism cannot be natural.
\begin{theorem}\label{thm:not-nat}
	For the category $\mathbf{G}_\text{af}$ there is no natural transformation, $\tau:I/T\rightarrow I$, which gives for
	each $G$ an isomorphism $\tau(G)$ of $G/T(G)$ into a subgroup of $G$.
\end{theorem}
\begin{proof}
	This proof will require consideration of an infinite class of groups, such as the groups $G_m=\mathbb{Z}\times
	\mathbb{Z}_m$ where $\mathbb{Z}$ is the additive group of integers and $\mathbb{Z}_m$ the additive group of
	integers, modulo $m$. Suppose that $\tau(G): G/T(G)\rightarrow G$ existed. If $\mu(G):G\rightarrow G/T(G)$ is the
	natural transformation of $G$ into $G/T(G)$ the product $\sigma(G)=\tau(G)\mu(G)$ would be a natural transformation
	of $G$ into $G$ with kernel $T(G)$. For each of the groups $G_m$ with elements $(a,b_m)$ for $a\in \mathbb{Z}$ and 
	$b_m\in \mathbb{Z}_m$, this transformation $\sigma_m=\sigma(G_m)$
	must be a homomorphism with kernel $\mathbb{Z}_m$, hence must have the form
	\begin{equation*}
		\sigma_m(a,b_m)=(r_m a,{(s_m a)}_m),
	\end{equation*}
	where $r_m$ and $s_m$ are integers. Now consider the homomorphism $\gamma:G_m\rightarrow G_m$ defined by setting
	$\gamma(a,b_m)= (0,b_m)$. Since $\sigma_m$ is natural, we must have $\sigma_m\gamma=\gamma\sigma_m$. Applying this
	equality to an arbitrary element we conclude that $s_m\equiv 0 \pmod{m}$. Next consider $\delta(a,b_m)=(0,a_m)$.
	the condition $\sigma_m\delta=\delta\sigma_m$ here gives $r_m\equiv 0\pmod{m}$, so that we can write $r_m=mt_m$.
	Therefore for each $m$
	\begin{equation*}
		\sigma_m(a,b_m)=(mt_m a,0).
	\end{equation*}
	Now consider two groups $G_m,G_n$ with a homomorphism $\beta:G_m\rightarrow G_n$ defined by setting
	$\beta(a,b_m)= (a,0_n)$. The naturality condition $\sigma_n\beta=\beta\sigma_m$ now gives $mt_m=nt_n$.
	If we hold $m$ fixed and allow $n$ to increase indefinitely, this contradicts the fact that $m t_m$ is a finite
	integer.
\end{proof}

It may be observed that the use of an infinite number of distinct groups is essential to the proof of this theorem.
For any subcategory of $\mathbf{G}_\text{af}$ containing only a finite number of groups, \cref{thm:not-nat} would be
false, for it would be possible to define a natural transformation $\tau(G)$ bye setting $[\tau(G)]g=kg$ for
every $g$, where the integer $k$ is chosen as any multiple of the order of all the subgroups $T(G)$ for $G$ in the 
given category.

The examples of ``non\hyp{}natural'' direct products adduced here are all examples which mathematicians would usually
recognize as not in fact natural. What we have done is merely to show that our definition of naturality does indeed
properly apply to cases of intuitively clear non\hyp{}naturality.

\section[Characters]{Characters\footnote{General references:~\cite[chap.1]{weil1938};~\cite[chap.2]{lef1942}}}\label{sec:char}
The character group of a group may be regarded as a contravariant functor on the category $\mathbf{G}_\text{lca}$ of 
locally compact regular abelian groups, with values in the same category. Specifically, this functor ``Char'' may be
defined by ``slicing'' (see \cref{sec:funct_slice} the functor Hom of \cref{sec:funct_examples} as follows.
Let $P$ be the (fixed) topological group of real numbers modulo 1, define ``Char'' by setting
\begin{equation}\label{eq:character}
	\chr G =\hm(G,P),\qquad \chr\gamma=\hm(\gamma,e_P).
\end{equation}

Given $g\in G$ and $\chi\in\chr G$ it will be convenient to denote the element $\chi(g)$ of $P$ by $(\chi,g)$. Using
this terminology and the definition of Hom we obtain for $\gamma:G_1\rightarrow G_2, \chi\in\chr G_2$ and $g\in G_1$
\begin{equation}\label{eq:character2}
	(\chr(\gamma)\chi,g)=(\chi,\gamma g).
\end{equation}
As mentioned before (\cref{sec:trans_ex}) the familiar isomorphism $\chr(\chr G)\cong G$ is a natural equivalence.

The functor ``Char'' can be compounded with other functors. Let $T$ be any functor covariant in $\mathbf{A}$,
contravariant in $\mathbf{B}$, with values in $\mathbf{G}_\text{lca}$. The composite functor $\chr T$ is then
defined on the same categories $\mathbf{A}$ and $\mathbf{B}$ but is contravariant in $\mathbf{A}$ and 
covariant in $\mathbf{B}$. Let $S$ be any closed subfunctor of $T$. Then for each pair of objects $A\in\mathbf{A},
B\in\mathbf{B}$, the closed subgroup $S(A,B)\subset T(A,B)$ determines a corresponding subgroup $\an S(A,B)$ in 
$\chr T(A,B)$; this annihilator is defined as the set of all those characters $\chi\in\chr T(A,B)$ with $(\chi,g)=0$
for each $g\in S(A,B)$. This leads to a closed subfunctor $\an(S;T)$ of the functor $\chr T$ determined by the object
function
\begin{equation*}
	[\an(S;T)](A,B)=\an S(A,B)\text{ in } \chr T(A,B).
\end{equation*}
It is well known that
\begin{gather*}
	\chr [T(A,B)/S(A,B)]=\an S(A,B),\\
	\chr S(A,B)=\chr T(A,B)/\an S(A,B).
\end{gather*}
These isomorphisms in fact yield natural equivalences
\begin{align}
	\sigma &: \an(S;T)\rightleftarrows\chr (T/S),\label{eq:nat_an}\\
	\tau &: \chr T/\an (S;T)\rightleftarrows\chr S.\label{eq:nat_an2}
\end{align}

For example, to prove \cref{eq:nat_an2} one observes that each $\chi\in\chr T(A,B)$ may be restricted to give
a character $\tau_0(A,B)\chi$ of $S(A,B)$ by setting
\begin{equation}\label{eq:nat_an3}
	(\tau_0(A,B)\chi,h)= (\chi,h),\qquad h\in S(A,B).
\end{equation}
This gives a homomorphism
\begin{equation*}
	\tau_0(A,B):\chr T(A,B)\rightarrow\chr S(A,B)
\end{equation*}
with kernel $\an S(A,B)$. This homomorphism $\tau_0$ will yield the required isomorphism $\tau$ of \cref{eq:nat_an2};
by \cref{thm:grp_mod} a proof that $\tau_0$ is natural will imply that $\tau$ is natural.

To show that $\tau_0$ is natural, consider any mappings $\alpha:A_1\rightarrow A_2$ and $\beta:B_1\rightarrow B_2$ in
the argument categories of $T$. Then $\gamma=T(\alpha,\beta)$ maps $T(A_1,B_2)$ into $T(A_2,B_1)$, while $\delta = 
S(\alpha,\beta)$ is a submapping of $\gamma$. The naturality requirements for $\tau_0$ is
\begin{equation}\label{eq:char-nat}
	(\chr\gamma)\tau_0(A_2,B_1)=\tau_0 (A_1,B_2)\chr\gamma.
\end{equation}
Each side is a homomorphism of $\chr T(A_2,B_1)$ into  $\chr S(A_1,B_2)$. If the left\hyp{}hand side is applied to an
element $\chi\in\chr T(A_2,B_1)$, and the resulting character of $S(A_1,B_2)$ is then applied to an element $h$ in
the latter group, we obtain
\begin{equation*}
	(\chr\gamma(\tau_0(A_2,B_1)\chi),h) = (\tau(A_2,B_1)\chi,\delta h) = (\chi,\delta h)
\end{equation*}
by using the definition \labelcref{eq:character2} of $\chr\delta$ and the definition \labelcref{eq:nat_an3} of $\tau_0$. If the 
right\hyp{}hand side of \cref{eq:char-nat} be similarly applied to $\chi$ and then to $h$, the result is
\begin{equation*}
	(\tau_0(A_1,B_2)((\chr\gamma)\chi),h)=((\chr\gamma)\chi,h)=(\chi,\gamma h).
\end{equation*}
Since $\delta\subset\gamma$, these two results are equal, and both $\tau_0$ and $\tau$ are therefore natural.

The proof of naturality for \cref{eq:nat_an} is analogous.

if $R$ is a closed subfunctor of $S$ which is in turn a closed subfunctor of $T$, both of these natural isomorphisms may be
combined to give a single natural isomorphism
\begin{equation}\label{eq:ann_comp}
	\rho:\chr(S/R)\rightleftarrows\an(S;T)/\an(R;T).
\end{equation}

\chapter{Partially ordered sets and projective limits}\label{ch:poset}
\section{Quasi-ordered sets}\label{sec:quasi-ord}
The notions of functors and their natural equivalences apply to partially ordered sets, to lattices, and to related
mathematical systems. The category $\mathbf{P}$ of all quasi\hyp{}ordered sets\footnote{A \emph{quasi\hyp{}ordered sets $P$}
is a set of elements $p_1,p_2\dotsc$ with a reflexive and transitive binary relation $p_1 \le  p_2$ between the elements. If,
in addition, the antisymmetric law ($p_1 \le  p_2$ and $p_2 \le  p_1 \implies p_1 = p_2$) holds, $P$ is a \emph{partially ordered set}.}
has as its objects the quasi\hyp{}ordered sets $P$ and its mappings $\pi:P_1\rightarrow P_2$ the order preserving transformations
of one quasi\hyp{}ordered sets, $P$, into another. An equivalence in this category is thus an isomorphism in the sense of order.

An important subcategory of $\mathbf{P}$ is the category $\mathbf{P}_\text{d}$ of all directed sets.\footnote{A quasi\hyp{}ordered
set $P$ is \emph{directed} if for each pair of elements $p_1,p_2\in P$ there exist a $p_3\in P$ with $p_1 \le p_2 \le p_3$.} One may also
consider subcategories which are obtained by restricting both the quasi\hyp{}ordered sets and their mappings. For example, the
category of lattices has as objects all those partially ordered sets which are lattices and as mappings those correspondences which
preserve both joins and meets. Alternatively, by using these mappings which preserve only joins, or those which preserve only
meets, we obtain two other categories of lattices.

The category $\mathbf{S}$ of sets may be regarded as a subcategory of $\mathbf{P}$, if each set $S$ is considered as a (trivially)
quasi\hyp{}ordered set in which $p_1\le p_2$ in $S$ means that $p_1=p_2$. The category $\mathbf{W}$ of well\hyp{}ordered sets in
another subcategory of $\mathbf{P}$. These categories provide a basis for applying the study of functors to cardinal and ordinal
arithmetic. Specifically, the general theory of arithmetic of partially ordered sets, as developed recently by Birkhoff,\footcite{birk42}
can be viewed as the construction of a large number of functors (cardinal power, ordinal power, and so no) defined on suitable
subcategories of $\mathbf{P}$, together with a collection of natural equivalences and transformations between these functors.\footnote{Note,
however, that the ordinal cardinal sum of two sets $A$ and $B$ does not give rise to a functor, because the definition
applies only when the sets $A$ and $B$ are disjoint.}

The construction of the category $\mathbf{P}$ of all quasi\hyp{}ordered sets is not the only such interpretation of a partial
order. It is also possible to regard the elements of a \emph{single} quasi\hyp{}ordered set $P$ as the objects of a category;
with this device, one can represent an inverse or a direct system of groups (or of spaces) as a functor on $P$.

If a quasi\hyp{}ordered set $P$ be regarded as a category $\mathbf{C}_P$, the objects of the category are all the elements $p\in P$ and
the mappings are the pairs $\pi=(p_2,p_1)$ of elements $p_i\in P$ such that $p_1\le p_2$. To each object $p$ we assign 
the pair $e_p=(p,p)$ as the corresponding identity mapping, while the product $(p_3,p_2')(p_2,p_1)$ of two mappings of
$\mathbf{C}_P$ is defined if and only if $p_2'=p_2$ and is in this case the mapping $(p_3,p_1)$. The \crefrange{ax:c1}{ax:c5} for a
category are readily verified, and it develops that the only identities are the pairs $(p,p)$, that the equivalence mappings of 
$\mathbf{C}_P$ are the pairs $(p_2,p_1)$ with $p_1\le p_2$ and $p_2\le p_1$ and that any pair $(p_2,p_1)$ with $p_1\le p_2$
is a mapping $(p_2,p_1):p_1\rightarrow p_2$. It further follows that any two mappings $\pi_1:p_1\rightarrow p_2$ and 
$\pi_2:p_1\rightarrow p_2$ of this category which have the same range and the same domain are necessarily equal. Conversely
any given category $\mathbf{C}$ which has the property that any two mappings $\pi_1$ and $\pi_2$ of $\mathbf{C}$ with the same
range and the same domain are equal is isomorphic to the category $\mathbf{C}_P$ for a suitable quasi\hyp{}ordered set $P$.
In fact, $P$ can be defined to be the set of all objects $C\in\mathbf{C}$ with $C_1\le C_2$ if and only if there is in $\mathbf{C}$
a mapping $\gamma:C_1\rightarrow C_2$.

Consider now two quasi\hyp{}ordered sets $P$ and $Q$, with their corresponding categories $\mathbf{C}_P$ and $\mathbf{C}_Q$. A
covariant (contravariant) functor $T$ on $\mathbf{C}_P$ with values in $\mathbf{C}_Q$ is determined uniquely by and order
preserving (reversing) mapping $t:P\rightarrow Q$. Specifically, each such correspondence $t$ is the object function $t(p)=q$
of a functor $T$, for which the corresponding mapping function is defined as $T(p_2,p_1)= (t p_2, t p_1)$ (or, in case
$t$ is order\hyp{}reversing, as $(t p_1, t p_2)$). Each functor $T$ of one variable can be obtained in this way.

\section{Direct systems as functors}\label{sec:dir-func}
Let $D$ be a directed set. If for every $d\in D$ a discrete group $G_d$ is defined and for every pair $d_1\le d_2$ in $D$ a
homomorphism
\begin{equation}\label{eq:dir-func}
	\phi_{d_2,d_1}:G_{d_1}\rightarrow G_{d_2}
\end{equation}
is given such that $\phi_{d,d}$ is the identity and that
\begin{equation}\label{eq:dir-func-def}
	\phi_{d_3,d_1} =  \phi_{d_3,d_2} \phi_{d_2,d_1}\text{ for } d_1\le d_2\le d_3
\end{equation}
then we say that the groups $\{G_d\}$ and the homomorphisms $\{\phi_{d_2,d_1}\}$ constitute a direct systems of groups indexed
by $D$.

Let us now regard the directed set $D$ as a category. For every object $d\in D$ define
\begin{equation*}
	T(d)= G_d.
\end{equation*}
For every mapping $\delta=(d_2,d_1)$ in $D$ define
\begin{equation*}
	T(\delta)=T(d_2,d_1)= \phi_{d_2,d_1}.
\end{equation*}
\Cref{eq:dir-func,eq:dir-func-def} imply that $T$ is a contravariant functor on $D$ with values in the category $\mathbf{G}_0$ of discrete
groups. Conversely any such functor give rise to a unique direct system. Consequently the terms ``direct system of groups indexed by
directed set $D$'' and ``covariant functor on $D$ to $\mathbf{G}_0$'' may be regarded as synonyms.

With each direct system groups $T$ there is associated a discrete limit group $G=\Lim_{\rightarrow} T$ defined as follows.
The elements of the limit group $G$ are pairs $(g,d)$ for $g\in T(d)$; two elements $(g_1,d_1)$ and $(g_2,d_2)$ are considered equal
if and only if there is an index $d_3$ with $d_1\le d_3, d_2\le d_3$ and with $T(d_3,d_1)g_1 = T(d_3,d_2)g_2$. The sum is defined
by setting $(g_1,d)+(g_2,d)= (g_1+g_2,d)$; since the set $D$ is directed, this provides for the addition of any two pairs in
$G$. For a fixed $d\in D$ one may also consider the homomorphisms, called projections, $\lambda(d):T(d)\rightarrow G$ defined by
setting
\begin{equation}\label{eq:dir-proj}
	\lambda(d)g= (g,d)
\end{equation}
for $g\in T(d)$. Clearly
\begin{equation}\label{eq:dir-proj2}
	\lambda(d_1) = \lambda(d_2)T(d_2,d_1)\text{ for } d_1\le d_2.
\end{equation}
To treat this limit group, we enlarge the given directed set $D$ by adjoining one new element $\top$, ordered by the specification 
that $d\le\top,\forall d\in D$. This enlarge directed set $\overline{D}$ also determines a category containing $D$ as a subcategory, with
new mappings $(\top,d)$ for each $d\in D$. Let now $T$ be any covariant functor on $D$ to $\mathbf{G}_0$ (that is, any direct system
of groups indexed by $D$). We define an extension $\overline{T}$ of the object function $T$ by setting
\begin{equation}\label{eq:dir-ext}
	\overline{T}(\top)=\Lim_{\rightarrow} T = G,
\end{equation}
the limit group of the given directed system $T$, and we similarly extend the mapping function of $T$ by letting $\overline{T}$, for a new
mapping $(\top,d)$, be the corresponding projection of $T(d)$ into the limit group
\begin{equation}\label{eq:dir-ext2}
	\overline{T}(\top,d)=\lambda(d).
\end{equation}
\Cref{eq:dir-ext} implies that $\overline{T}$ is indeed a covariant function on $\overline{D}$ with values in $\mathbf{G}_0$. The properties of
the limit group may be described in terms of this extended functor $\overline{T}$.
\begin{theorem}\label{thm:dir-uniq}
	Let $D$ be a directed set and $T$ a covariant functor on $D$ (regarded as a category) to $\mathbf{G}_0$. Then the limit group $G$
	of the direct system $T$ and the projections of each group $T(d)$ into this limit determine as in \cref{eq:dir-ext,eq:dir-ext2}
	an extension of $T$ to a covariant functor $\overline{T}$ on $\overline{D}$ to $\mathbf{G}_0$. If $\overline{S}$ is any other
	extension of $T$ to a covariant functor on $\overline{D}$ to $\mathbf{G}_0$, there is a unique natural transformation
	$\sigma:\overline{T}\rightarrow\overline{S}$ such that each $\sigma(d)$ with $d\ne\top$ is the identity.
\end{theorem}
\begin{proof}
	We have already seen that $\overline{T}$ is a covariant functor on $\overline{D}$ to $\mathbf{G}_0$, extending $T$. Let now
	$\overline{S}$ be any other functor extending $T$. Since $S(d_2,d_1)=T(d_2,d_1)$ for $d_2\le d_1$ in $D$, it follows from 
	the functor condition on $\overline{S}$ that
	\begin{equation}\label{eq:dir-uniq}
		\overline{S}(\top,d_2)T(d_2,d_1) = \overline{S}(\top,d_1).
	\end{equation}
	We define a homomorphism
	\begin{equation*}
		\sigma(\top):\overline{T}(\top)\rightarrow\overline{S}(\top)
	\end{equation*}
	by setting  $\sigma(\top)(g,d)=\overline{S}(\top,d)g$ for every element $(g,d)\in\overline{T}(\top)=\Lim_{\rightarrow} T$.
	\Cref{eq:dir-uniq} implies that $\sigma(\top)$ is single\hyp{}valued. If we now set $\sigma(d)$ to be the identity mapping
	$\overline{T}(d)\rightarrow\overline{S}(d)$ for $d\ne\top$, we have the desired transformation $\sigma:\overline{T}\rightarrow\overline{S}$.
\end{proof}

The extension $\overline{T}$ and hence the limit group $G=\overline{T}(\top)$ of the given direct system is completely determined by
the property given in the last sentence of the theorem. In fact if $\hat{T}$ is any other extension of $T$ with the same
property as $\overline{T}$, there will exist transformations $\sigma:\overline{T}\rightarrow\hat{T}$ and
$\tau:\hat{T}\rightarrow\overline{T}$. Then $\rho=\tau\sigma:\overline{T}\rightarrow\overline{T}$ with $\rho(d)$
the identity whenever $d\ne\top$. It follows that
\begin{equation*}
	\rho(\top)(g,d)=\rho(\top)\lambda(d) g =\lambda(d) g = (g,d).
\end{equation*}
Hence $\rho(\top)$ is the identity and $\sigma$ a natural equivalence $\sigma:\overline{T}\rightarrow\hat{T}$. In this way a limit
group of a direct system of groups can be defined up to an isomorphism by means of such extensions of functors. This indicates that
the concept (but not necessarily the existence) of direct ``limits'' could be set up not only for groups, but also for objects of
any category.
\begin{theorem}\label{thm:dir-nat}
	If $T_1$ and $T_2$ are two covariant functors on the directed category $D$ with values in $\mathbf{G}_0$ and $\tau$ is a 
	natural transformation $\tau:T_1\rightarrow T_2$, there is only one extension $\overline{\tau}$ of $\tau$ which is a 
	natural transformation $\overline{\tau}:\overline{T_1}\rightarrow\overline{T_2}$ between the extended functors on 
	$\overline{D}$. When $\tau$ is a natural equivalence so is $\overline{\tau}$.
\end{theorem}
\begin{proof}
	The naturality condition for $\tau$, when applied to any mapping $(d_2,d_1)$ with $d_1\le d_2$ in the directed set $D$ reads
	\begin{equation}\label{eq:dir-nat}
		\tau(d_2)T_1(d_2,d_1)=T_2(d_2,d_1)\tau(d_1).
	\end{equation}
	Given any element $(g_1,d)$ of the limit group $\overline{T_1}(\top)=\Lim_{\rightarrow} T_1$ we define
	\begin{equation}\label{eq:dir-nat-def}
		\omega(g_1,d)=(\tau(d) g_1, d)\in \Lim_{\rightarrow} T_2 = \overline{T_2}(\top).
	\end{equation}
	\Cref{eq:dir-nat} implies that this definition of $\omega$ gives a result independent of the special representation $(g_1,d)$
	chosen for the limit element. Hence we get a homomorphism
	\begin{equation*}
		\omega:\overline{T_1}\rightarrow\overline{T_2}.
	\end{equation*}
	In virtue of \cref{eq:dir-ext2,eq:dir-proj}, the definition \labelcref{eq:dir-nat-def} becomes
	\begin{equation}\label{eq:dir-nat-def2}
		\omega\overline{T_1}(\top,d)=\overline{(T_2)}(\top,d)\tau(d).
	\end{equation}
	This means simply that by setting $\overline{\tau}(d)=\tau(d)$, $\overline{\tau}(\top)=\omega$ we get an extension of $\tau$
	which is still natural and which gives a transformation $\overline{\tau}:\overline{T_1}\rightarrow\overline{T_2}$. Since
	the naturality condition \labelcref{eq:dir-nat-def2} is equivalent with \cref{eq:dir-nat-def} which completely determines the
	value of $\overline{\tau}(\top)$, the requisite uniqueness follows.
\end{proof}
In particular, if $\tau$ is an equivalence, each $\tau(d)$ is an isomorphism ``onto'', hence it follows that
$\omega=\overline{\tau}(\top)$ is also an isomorphism onto, and is an equivalence. This is just a restatement of the known
theorem that ``isomorphic'' direct systems determine isomorphic limit groups.
\begin{theorem}\label{thm:dir-const}
	If $T$ is a direct system of groups indexed by a directed set $D$, while $H$ is a fixed discrete group, regarded as a
	(constant) covariant functor on $D$ to $\mathbf{G}_0$, then for each natural transformation $\tau:T\rightarrow H$ there
	is a unique homomorphism $\tau_0$ of the limit group $\Lim_{\rightarrow} T$ into $H$ with the property that $\tau(d)=\tau_0\lambda(d)$
	for each $d\in D$, where $\lambda(d)$ is the projection of $T(d)$ into $\Lim_{\rightarrow} T$.
\end{theorem}
\begin{proof}
	This follows from the preceding theorem and from the remark that $\overline{H}$ is also a constant functor from $\overline{D}$
	to $\mathbf{G}_0$.
\end{proof}

\section{Inverse limits as functors}\label{sec:inv_lim}
Let $D$ be a directed set. If for every $d\in D$ a topological group $G_d$ is defined and for every pair $d_1\le d_2$ in $D$ a
homomorphism
\begin{equation}\label{eq:inv-lim1}
	\phi(d_2,d_1):G_{d_2}\rightarrow G_{d_1}
\end{equation}
is given such that $\phi(d,d)$ is the identity and that
\begin{equation}\label{eq:inv-lim2}
	\phi(d_3,d_3)=\phi(d_2,d_1)\phi(d_3,d_2)\quad\text{for } d_1\le d_2\le d_3
\end{equation}
then we say that the groups $\{G_d\}$ and the homomorphisms $\{\phi(d_2,d_1)\}$ constitute an inverse systems of groups indexed by $D$.

If we now regard $D$ as a category, and define as before
\begin{equation}\label{eq:inv-func}
	T(d)= G_d
\end{equation}
for every object $d\in D$, and
\begin{equation}\label{eq:inv-func2}
	T(\delta)=T(d_2,d_1)=\phi(d_2,d_1)
\end{equation}
for every mapping $\delta=(d_2,d_1)\in D$, it is clear that $T$ is a contravariant functor on $D$ with values in the category $\mathbf{G}$
of topological groups. Conversely any such functor may be regarded as an inverse system of groups.

With each inverse system of groups $T$ there is associated a limit group $G=\Lim_{\leftarrow} T$ defined as follows. An element of $G$
is a function $g(d)$ which assigns to each element $d\in D$ an element $g(d)\in T(d)$, in such way that these elements ``match'' under
mappings; that is, such that $T(d_2,d_1)g(d_2)=g(d_1)$ whenever $d_1\le d_2$. The sum of $g_1+g_2$ is defined as $(g_1+g_2)(d)=
g_1(d)+g_2(d)$. This limit group $G$ is assigned a topology, in known fashion, by treating $G$ as a subgroup of the direct product of the
groups $T(d)$, with the usual direct product topology. For fixed $d$, the (continuous) projection $\mu(d)$ of the limit group $G$ into
$T(d)$ is defined by setting $[\mu(d)]g= g(d)$, for $g\in G$.

Again we may consider the extended category $\overline{D}$ and define the extension $\overline{T}$ of $T$ by setting
\begin{equation}\label{eq:inv-ext}
	\overline{T}(\top)=G\qquad \overline{T}(\top,d)=\mu(d).
\end{equation}
As before the following theorem can be established:
\begin{theorem}\label{thm:inv-lim1}
	Let $D$ be a directed set and $T$ a contravariant functor on $D$ (regarded as a category) to $\mathbf{G}$. Then the limit
	group $G$ of the inverse system $T$ and the projections of this limit group into each group $T(d)$ determine as in
	\cref{eq:inv-ext} an extension of $T$ to a contravariant functor $\overline{T}$ on $\overline{D}$ to $\mathbf{G}$. If
	$\overline{S}$ is any other extension of $T$ to a contravariant functor on $\overline{D}$ to $\mathbf{G}$, there is an
	unique natural transformation $\sigma:\overline{S}\rightarrow\overline{T}$ such that each $\sigma(d)$ with $d\ne\top$
	is the identity.
\end{theorem}
As before we can also verify that the second half of the theorem determines the extended functor $\overline{T}$ to within a 
natural equivalence, and therefore it determines the limit group to within an isomorphism.
The following two theorems may also be proved as in the preceding section.
\begin{theorem}\label{thm:inv-uniq1}
	If $T_1$ and $T_2$ are two contravariant functors on the directed category $D$ with values in $\mathbf{G}$, and $\tau$
	is a natural transformation $\tau:T_1\rightarrow T_2$, there is only one extension $\overline{\tau}$ of $\tau$ which is
	a natural transformation $\overline{\tau}:\overline{T_1}\rightarrow \overline{T_2}$, between the the extended functors
	on $\overline{D}$. When $\tau$ is a natural equivalence so is $\overline{\tau}$.
\end{theorem}
\begin{theorem}\label{thm:inv-uniq2}
	If $T$ is an inverse system of groups indexed by the directed set $D$, while $K$ is a fixed topological group regarded
	as a (constant) contravariant functor on $D$ to $\mathbf{G}$, then for each natural transformation $\tau:T\rightarrow K$
	there is a unique homomorphism $\tau_0:\Lim_{\leftarrow} T\rightarrow K$ such that $\tau_0=\tau(d)\lambda(d),\forall d\in D$.
\end{theorem}

The preceding discussion carries over to inverse systems of spaces, by a mere replacement of the category of topological groups
$\mathbf{G}$ by the category of topological spaces $\mathbf{X}$.

\section{The categories ``\textbf{Dir}'' and ``\textbf{Inv}''}\label{sec:dir-inv}
The process of forming a direct or inverse limit of a system of groups can be treated as a functor ``$\Lim_{\rightarrow}$'' or
``$\Lim_{\leftarrow}$'' which operates on an appropriately defined category. Thus, the functor `$\Lim_{\rightarrow}$'' will
operate on any direct system $T$ defined on any directed set $D$. Consequently we define a category ``\textbf{Dir}'' of
directed systems whose objects are such pairs $(D,T)$. Here we may regard $D$ itself as a category and $T$ as a covariant
functor on $D$ to $\mathbf{G}_0$. To introduce the mappings of this category, observe first that each order preserving transformation
$R$ of a directed set $D_1$ into another such set $D_2$ will give for each direct system $T_2$ of groups indexed by $D_2$ an induced
direct system indexed by $D_1$. Specifically, the induced direct system is just the composite $T_2\circ R$ of the (covariant) functor
$R$ on $D_1$ to $D_2$ and the (covariant) functor $T_2$ on $D_2$ to $\mathbf{G}_0$. Given two objects $(D_1,T_1)$ and $(D_2,T_2)$
of \textbf{Dir}, a mapping
\begin{equation*}
	(R,\rho):(D_1,T_1)\rightarrow (D_2,T_2)
\end{equation*}
of the category \textbf{Dir} is a pair $(R,\rho)$ composed of a covariant functor $R$ on $D_1$ to $D_2$ and a natural transformation
\begin{equation*}
	\rho:T_1\rightarrow T_2\circ R
\end{equation*}
of $T_1$ into the composite functor $T_2\circ R$.

To form the product of two such mappings
\begin{equation}\label{eq:dir_comp}
	(R_1,\rho_1):(D_1,T_1)\rightarrow (D_2,T_2),\qquad (R_2,\rho_2):(D_2,T_2)\rightarrow (D_3,T_3)
\end{equation}
observe first that the functors $T_2$ and $T_3\circ R_2$ on $D_2$ to $\mathbf{G}_0$ can be compounded with the functor $R_1$ on $D_1$
to $D_2$, and hence that the given transformation $\rho_2:T_2\rightarrow T_3\circ R_2$ can be compounded with the identity transformation
of $R_1$ into itself, just as in \cref{sec:funct_comp}.

The result is a composite transformation
\begin{equation}\label{eq:dir_comp2}
	\rho_2\circ R_1:T_2\circ R_1\rightarrow T_3\circ R_2\circ R_1
\end{equation}
which assigns to each object $d_1\in D_1$ the mapping $[\rho_2\circ R_1](d_1)=\rho_2(R_1 d)$ of $T_2(R_1 d)$ into $T_3\circ R_2(R_1 d_1)$.
The transformation \labelcref{eq:dir_comp2} and $\rho_1:T_1\rightarrow T_2\circ R_1$ yield as in \cref{sec:funct_comp}
a composite transformation $\rho_2\circ R_1\circ\rho_1: T_1\rightarrow (T_3\circ R_2\circ R_1)$.
We may now define the product of two mappings given by \cref{eq:dir_comp} to be
\begin{equation*}
	(R_2,\rho_2)(R_1,\rho_1)= (R_2\circ R_1,\rho_2\circ R_1\circ\rho_1).
\end{equation*}
With these conventions, we verify that \textbf{Dir} is a category. Its identities are the pairs $(R,\rho)$ in which both $R$ and $\rho$ are
identities; its equivalences are the pairs $(R,\rho)$ in which $R$ is an isomorphism and $\rho$ a natural equivalence.

The effect of fixing the directed set $D$ in the objects $(D,T)$ of the category \textbf{Dir} is to restrict \textbf{Dir} to the subcategory
which consist of all direct systems of groups indexed by $D$ (that is, the category of all covariant functors on $D$ to $\mathbf{G}_0$, as
defined in \cref{sec:funct_cat}).

We shall now define $\Lim_{\rightarrow}$ as a covariant functor on \textbf{Dir} with values in $\mathbf{G}_0$. For each object
$(D,T)$ of \textbf{Dir} we define $\Lim_{\rightarrow}(D,T)$ to be the group obtained as the direct limit of the direct system
of groups $T$ indexed by the directed set $D$. Given a mapping
\begin{equation}\label{eq:dir_map}
	(R,\rho):(D_1,T_1)\rightarrow (D_2,T_2)\quad\text{in \textbf{Dir}}
\end{equation}
we define the mapping function of $\Lim_{\rightarrow}$,
\begin{equation}\label{eq:dir_map2}
	\Lim_{\rightarrow}(R,\rho):\Lim_{\rightarrow}(D_1,T_1)\rightarrow\Lim_{\rightarrow}(D_2,T_2),
\end{equation}
as follows. An element in the limit group $\Lim_{\rightarrow}(D_1,T_1)$ is a pair $(g_1,d_1)$ with $d_1\in D_1, g_1\in T_1(d_1)$.
For each such element define $\phi(g_1,d_1)$ to be the pair $(\rho(d_1)g_1,R d_1)$. Since $\rho(d_1)$ maps $T_1(d_1)$ into
$T_2(R d_1)$ we have $\rho(d_1)g_1\in T_2(R d_1)$, so that the resulting pair is indeed in the limit group $\Lim_{\rightarrow}(D_2,T_2)$.
The mapping $\rho$ carries equal pairs into equal pairs, and yields the requisite homomorphism \labelcref{eq:dir_map2}.
We verify that $\Lim_{\rightarrow}$, defined in this manner, is a covariant functor on \textbf{Dir} to $\mathbf{G}_0$.

Alternatively, the mapping function of this functor ``$\Lim_{\rightarrow}$'' can be obtained by extensions of mappings to the
directed sets $\overline{D_1},\overline{D_2}$ (with $\top$ added), defined as in \cref{sec:dir-func}. Given the mapping $(R,\rho)$
of \cref{eq:dir_map}, first extend the given object of \textbf{Dir} to obtain new objects $(\overline{D_1},\overline{T_1})$ and
$(\overline{D_2},\overline{T_2})$. The given functor $R$ on $D_1$ to $D_2$ can also be extended by setting
$\overline{R}(\top)=\top$; this gives a functor $\overline{R}$ on $\overline{D_1}$ to $\overline{D_2}$. Furthermore, 
\cref{thm:dir-uniq} asserts that the transformation $\rho:T_1\rightarrow T_2\circ R$ has then a unique extension
$\overline{\rho}:\overline{T_1}\rightarrow \overline{T_2}\circ \overline{R}$.
All told, we have a new mapping
\begin{equation*}
	(\overline{R},\overline{\rho}):(\overline{D_1},\overline{T_1})\rightarrow (\overline{D_2},\overline{T_2})\quad\text{in \textbf{Dir}}
\end{equation*}
In particular, when $\overline{\rho}$ is applied to the new element $\top\in\overline{D_1}$, it yields a homomorphism of the limit group
of $T_1$ into the limit group of $T_2\circ R$. On the other hand, $R$ determines a homomorphism $\hat{R}$ of the limit group of
$T_2\circ R$ into the limit group of $T_2$; explicitly, for $(g_1,d_1)$ in the first limit group, the image $\hat{R}(g_1,d_1)$ is the
element $(g_1,R d_1)$ in the second limit group. The requisite mapping function of the functor ``$\Lim_{\rightarrow}$'' is now defined by
setting
\begin{equation*}
	\Lim_{\rightarrow}(R,\rho)=\hat{R}(\overline{\rho}(\top)).
\end{equation*}

In a similar way we define the category \textbf{Inv}. The objects of \textbf{Inv} are pairs $(D,T)$ where $D$ is a directed set and $T$
is an inverse system of topological groups indexed by $D$ (that is, $T$ is a contravariant functor on $D$ to $\mathbf{G}$). The mappings
in \textbf{Inv} are pairs $(R,\rho)$
\begin{equation*}
	(R,\rho):(D_1,T_1)\rightarrow (D_2,T_2)
\end{equation*}
where $R$ is a covariant functor on $D_2$ to $D_1$ (that is, an order preserving transformation of $D_2$ into $D_1$) and $\rho$ is a 
natural transformation of the functors
\begin{equation*}
	\rho:T_1\circ R\rightarrow T_2
\end{equation*}
both contravariant on $D_2$ to $\mathbf{G}$. The product of two mappings
\begin{equation*}
	(R_1,\rho_1):(D_1,T_1)\rightarrow (D_2,T_2),\qquad (R_2,\rho_2):(D_2,T_2)\rightarrow (D_3,T_3)
\end{equation*}
is defined as
\begin{equation*}
	(R_2,\rho_2)(R_1,\rho_1)= (R_1\circ R_2, \rho_2\circ\rho_1\circ R_2)
\end{equation*}
where $\rho_1\circ R_2$ is the transformation
\begin{equation*}
	\rho_1\circ R_2:T_1\circ R_1\circ R_2\rightarrow T_2\circ R_2
\end{equation*}
induced (as in \cref{sec:funct_comp}) by
\begin{equation*}
	\rho_1:T_1\circ R_1\rightarrow T_2.
\end{equation*}
With these conventions, we verify that \textbf{Inv} is a category.

We shall now define $\Lim_{\leftarrow}$ as a covariant functor on \textbf{Inv} with values in $\mathbf{G}$. For each object
$(D,T)$ in  \textbf{Inv} we define $\Lim_{\leftarrow}(D,T)$ to be the inverse limit of the inverse system of groups $T$
indexed by the directed set $D$. Given a mapping
\begin{equation}\label{eq:inv_map}
	(R,\rho):(D_1,T_1)\rightarrow (D_2,T_2)\in\text{\textbf{Inv}}
\end{equation}
we define the mapping function of $\Lim_{\leftarrow}$
\begin{equation}\label{eq:inv_map2}
	\Lim_{\leftarrow}(R,\rho):\Lim_{\leftarrow}(D_1,T_1)\rightarrow \Lim_{\leftarrow}(D_2,T_2)
\end{equation}
as follows. Each element of $\Lim_{\leftarrow}(D_1,T_1)$ is a function $g(d_1)$ with values $g(d_1)\in T_1(d_1)$, for
$d_1\in D_1$, which match properly under the projections in $T_1$. Now we define a new function $h$, with
\begin{equation*}
	h(d_2)= \rho(_d2)g(R d_2),\quad d_2\in D_2;
\end{equation*}
it is easy to verify that $h$ is an element of the limit group $\Lim_{\leftarrow}(D_2,T_2)$. The correspondence $g\rightarrow h$
is the homomorphism \labelcref{eq:inv_map2} required for the definition of the mapping function of $\Lim_{\leftarrow}$. One may
verify that this definition does yield a covariant functor $\Lim_{\leftarrow}$ on the category \textbf{Inv} to $\mathbf{G}$.

The mapping function of $\Lim_{\leftarrow}$ may again be obtained by first extending the given mapping \labelcref{eq:inv_map} to
\begin{equation*}
	(\overline{R},\overline{\rho}):(\overline{D_1},\overline{T_1})\rightarrow (\overline{D_2},\overline{T_2})\in\text{\textbf{Inv}}.
\end{equation*}
In particular, when the extended transformation $\overline{\rho}$ is applied to the element $\top$ of $\overline{D_1}$, we obtain
a homomorphism of the limit group of $T_1\circ R$ into the limit group $T_2$. On the other hand, the covariant functor
$R:D_2\rightarrow D_1$ determines a morphism $\breve{R}$ of the limit group $(D_1,T_1)$ into the limit group of $(D_2,T_1\circ R)$;
explicitly, for each function $g(d_1)$ in the first limit group, the image $h=\breve{R} g$ in the second limit group is defined
by setting $h(d_2)= g(R d_2),\,\forall d_2\in D_2$. The mapping function of the functor ``$\Lim_{\leftarrow}$'' is now 
$\Lim_{\leftarrow}(R,\rho)=\overline{\rho}(\top)\breve{R}$.

\section{The lifting principle}\label{sec:lift}
Let $Q$ be a functor whose arguments and values are groups, while $T$ is any direct or inverse system of groups. If the object
function of $Q$ is applied to each group $T(d)$ of the given system, while the mapping function of $Q$ is applied to each
projection $T(d_1,d_2)$ of the given system, we obtain a new system of groups, which may be called $Q\circ T$. If $Q$ is
covariant, $T$ and $Q\circ T$ are both direct or inverse, while if $Q$ is contravariant, $Q\circ T$ is inverse when $T$ is 
direct, and vice versa.

Actually this new system $Q\circ T$ is simply the composite of the functor $T$ with the functor $Q$ (see \cref{sec:funct_comp}).
We may regard this composition as a process which ``lifts'' a functor $Q$ whose arguments and values are groups to a functor
$Q_L$ whose arguments and values are direct (or inverse) systems of groups. We may then regard the lifted functor as one acting
on the categories \textbf{Dir} and \textbf{Inv}, as the case may be. In every case, the lifted functor has its object and mapping
functions given formally by the equations (in the ``$\circ$-notation'' for composites)
\begin{align}
	Q_L(D,T)&=(D,Q\circ T)\label{eq:func_lift1}\\
	Q_L(R,\rho)&=(R,Q\circ\rho)\label{eq:func_lift2}.
\end{align}

These formulas includes the following four cases:
\begin{enumerate}
	\item $Q$ covariant on $\mathbf{G}_0$ to $\mathbf{G}_0$; $Q_L$ covariant on \textbf{Dir} to \textbf{Dir}.\label[case]{cond1}
	\item $Q$ contravariant on $\mathbf{G}_0$ to $\mathbf{G}$; $Q_L$ contravariant on \textbf{Dir} to \textbf{Inv}.\label[case]{cond2}
	\item $Q$ covariant on $\mathbf{G}$ to $\mathbf{G}$; $Q_L$ covariant on \textbf{Inv} to \textbf{Inv}.\label[case]{cond3}
	\item $Q$ contravariant on $\mathbf{G}$ to $\mathbf{G}_0$; $Q_L$ contravariant on \textbf{Inv} to \textbf{Dir}.\label[case]{cond4}
\end{enumerate}

For illustration, we discuss \cref{cond2}, in which $Q$ is given contravariant on $\mathbf{G}_0$ to $\mathbf{G}$. The object
function of $Q_L$, as defined in \cref{eq:func_lift1}, assigns to each object $(D,T)$ of the category
\textbf{Dir} a pair $(D,Q\circ T)$. Since  $T$ is covariant on $D$ to $\mathbf{G}_0$ and $Q$ contravariant on $\mathbf{G}_0$
to $\mathbf{G}$, the composite $Q\circ T$ is contravariant on $D$ to $\mathbf{G}$, so that $Q\circ T$ is an inverse system
of groups, and the pair $(D,Q\circ T)$ is an object of \textbf{Inv}. On the other hand, given a mapping
\begin{equation*}
	(R,\rho):(D_1,T_1)\rightarrow (D_2,T_2)\in\mathbf{Dir},
\end{equation*}
with $\rho:T_1\rightarrow (T_2\circ R)$, the composite transformation $Q\circ\rho$ is obtained by applying the mapping function
of $Q$ to each homomorphism $\rho(d_1):T_1(d_1)\rightarrow(T_2\circ R(d_1))$, and this gives a transformation
$Q\circ\rho:(Q\circ T_2\circ R)\rightarrow (Q\circ T_1)$. Thus the mapping function of $Q_L$, as defined in \cref{eq:func_lift2},
does give a mapping $(R,Q\circ\rho):(D_2,Q\circ T_2)\rightarrow (D_1,Q\circ T_1)$ in the category \textbf{Inv}. We verify that
$Q_L$ is a contravariant functor on \textbf{Dir} to \textbf{Inv}.

Any natural transformation $k_1:Q\rightarrow P$ induces a transformation on the lifted functors, $k_L:Q_L\rightarrow P_L$, obtained
by composition of the transformation $k$ with the identity transformation of each $T$, as
\begin{equation*}
	k_L(D,T)=(D,k\circ T).
\end{equation*}
If $k$ is an equivalence, so is this ``lifted'' transformation.

Just as in the case of composition, the operation of ``lifting'' can itself be regarded as a functor ``Lift'', defined on a 
suitable category of functors $Q$. In all four \cref{cond1,cond2,cond3,cond4}, this functor ``Lift'' is covariant.

In all these cases the functor $Q$ may originally contain any number of additional variables. The lifted functor $Q_L$ will
then involve the same extra variables with the same variance. With proper caution the lifting process may also be applied 
simultaneously to a functor $Q$ with two variables, both of which are groups.

\section{Functors which commute with limits}\label{sec:func_comm_lim}
Certain operations, such as the formation of the character groups of discrete or compact groups, are known to ``commute''
with the passage to a limit. Using the lifting operation, this can be formulated exactly.

To illustrate, let $Q$ be a covariant functor on $\mathbf{G}_0$ to $\mathbf{G}_0$, and $Q_L$ the corresponding
covariant lifted functor on \textbf{Dir} to \textbf{Dir}, as in \cref{cond1} of \cref{sec:lift}. Since $\Lim_{\rightarrow}$
is a covariant functor on \textbf{Dir} to $\mathbf{G}_0$, we have two composite functors
\begin{equation*}
	\Lim_{\rightarrow}\circ\,Q_L\text{ and } Q\circ \Lim_{\rightarrow},
\end{equation*}
both covariant on \textbf{Dir} to $\mathbf{G}_0$. There is also an explicit natural transformation
\begin{equation}\label{eq:lim_nat}
	\omega_1:(\Lim_{\rightarrow}\circ\,Q_L)\rightarrow (Q\circ \Lim_{\rightarrow}),
\end{equation}
defined as follows. Let the pair $(D,T)$ be a direct system of groups in the category \textbf{Dir}, and let $\lambda(d)$ be
the projection
\begin{equation*}
	\lambda(d):T(d)\rightarrow \Lim_{\rightarrow}T,\,d\in D.
\end{equation*}
Then, on applying the mapping function of $Q$ to $\lambda$, we obtain the natural transformation
\begin{equation*}
	Q\lambda(d): Q T(d)\rightarrow Q[\Lim_{\rightarrow}T].
\end{equation*}
\Cref{thm:dir-const} now gives a homomorphism
\begin{equation*}
	\omega_1(T):\Lim_{\rightarrow}[Q\circ T]\rightarrow Q[\Lim_{\rightarrow} T],
\end{equation*}
or, exhibiting $D$ explicitly, a homomorphism
\begin{equation*}
	\omega_1(D,T):\Lim_{\rightarrow}Q_L(D,T)\rightarrow Q[\Lim_{\rightarrow} (D,T)].
\end{equation*}
We verify that $\omega_1$, so defined, satisfies the naturality condition.

Similarly, to treat \cref{cond2}, consider a contravariant functor $Q$ on $\mathbf{G}_0$ to $\mathbf{G}$ and the lifted functor
$Q_L$ on \textbf{Dir} to \textbf{Inv}. We then construct an explicit natural transformation
\begin{equation}\label{eq:nat_case2}
	\omega_2:(Q\circ\Lim_{\rightarrow})\rightarrow(\Lim_{\leftarrow}\circ\, Q_L)
\end{equation}
(note the order!), defined as follows. Let the pair $(D,T)$ be in \textbf{Dir}, and let $\lambda(d)$ be the projection
\begin{equation*}
	\lambda(d):T(d)\rightarrow \Lim_{\rightarrow}T,\,d\in D.
\end{equation*}
On applying $Q$, we get
\begin{equation*}
	Q\lambda(d):Q[\Lim_{\rightarrow}T]\rightarrow QT(d).
\end{equation*}
\Cref{thm:inv-uniq2} for inverse systems now gives a homomorphism
\begin{equation*}
	\omega_2(D,T):Q[\Lim_{\rightarrow}(D,T)]\rightarrow Q[\Lim_{\leftarrow}Q_L(D,T)].
\end{equation*}

In the remaining \cref{cond3,cond4} similar arguments give natural transformation
\begin{align}
	\omega_3 &:(Q\circ\Lim_{\leftarrow})\rightarrow(\Lim_{\leftarrow}\circ\, Q_L),\label{eq:nat_case3}\\
	\omega_4 &:(\Lim_{\rightarrow}\circ\, Q_L)\rightarrow (Q\circ\Lim_{\leftarrow}).\label{eq:nat_case4}
\end{align}
\begin{thdef}
	The functor $Q$ defined on groups to groups is said to commute (more precisely to $\omega$-commute) 
	with $\Lim_{\leftrightarrows}$ if the appropriate one of the four natural transformations $\omega$
	above is an equivalence.
\end{thdef}

In other words, the proof that a functor $Q$ commutes with $\Lim_{\rightarrow}$ requires only the verification that the 
homomorphisms defined above are isomorphisms. The naturality condition holds in general!

To illustrate these concepts, consider the functor $C$ which assigns to each discrete group $G$ its commutator
subgroup $C(G)$, and consider a direct system $T$ of groups, indexed by $D$. Then the lifted functor $Q$
(case 1 of \cref{sec:lift}) applied to the pair $(D,T)$ in \textbf{Dir} gives a new direct system of groups,
still indexed by $D$, with the groups $T(d)$ of the original system replaced by their commutator
subgroups $CT(d)$, and with the projections correspondingly cut down. It may be readily verified that
this functor does commute with $\Lim_{\rightarrow}$.

Another functor $Q$ is the subfunctor of the identity which assigns to each discrete abelian group $G$ the subgroup
$Q(G)$ consisting of those elements $g\in G$ such that there is for each integer $m$ an $x\in G$ with $mx=g$ 
(that is, of those elements of $G$ which are divisible by every integer), $Q$ is a covariant functor with arguments
and values in the subcategory $\mathbf{G}_{0a}$ of discrete abelian groups. This functor $Q$ clearly does not
commute with $\Lim_{\rightarrow}$, since one may represent the additive group of rational numbers as a direct limit
cyclic group $\mathbb{Z}$ for which each subgroup $Q(\mathbb{Z})$ is the group consisting of zero alone.

The formation of character groups gives further examples. If we consider the functor $\chr$ as a contravariant
functor on the category $\mathbf{G}_{0a}$ of compact abelian groups, the lifted functor $\chr_L$ will be covariant
on the appropriate subcategory of \textbf{Dir} to \textbf{Inv} as in \cref{cond2} of \cref{sec:lift}. This lifted functor
$\chr_L$ applied to any direct system $(D,T)$ of discrete abelian groups will yield an inverse system of compact
abelian group, indexed by the same set $D$. Each group of the inverse system is the character group of the
corresponding group of the direct system, and the projections of the inverse system are the induced mappings.

On the other hand, there is a contravariant functor $\chr$ on $\mathbf{G}_{ca}$ to $\mathbf{G}_{0a}$. In this case
the lifted functor $\chr_L$ will be contravariant on a suitable subcategory of \textbf{Inv} with values in 
\textbf{Dir}, just as in \cref{cond3} of \cref{sec:lift}. Both these functors $\chr$ commute with $\Lim$.

\chapter[Applications to topology]{Applications to topology\footnote{General reference:~\cite{lef1942}.}}\label{ch:topology}
\section{Complexes}\label{sec:comp}
An abstract complex $K$ (in the sense of W.~Mayer) is a collection
\begin{equation*}
	\{C^q(K)\},\quad q\in\mathbb{Z},
\end{equation*}
of free abelian discrete groups, together with a collection of homomorphisms
\begin{equation*}
	\partial^q:C^q(K)\rightarrow C^{q-1}(K)
\end{equation*}
called \emph{boundary homomorphism}, such that
\begin{equation*}
	\partial^q\partial^{q+1}=0.
\end{equation*}

By selecting for each of the free groups $C^q$ a fixed basis $\{\sigma_i^q\}$ we obtain a complex which is substantially
an abstract complex in the sense of A. W. Tucker. The $\sigma_i^q$ will be called $q$-dimensional cells. The boundary
operator $\partial$ can be written as a finite sum
\begin{equation*}
	\partial\sigma^q=\sum_{\sigma^{q-1}}[\sigma^q : \sigma^{q-1}]\sigma^{q-1}.
\end{equation*}
The integers $[\sigma^q : \sigma^{q-1}]$ are called incidence numbers, and satisfy the following conditions:
\begin{enumerate}
	\item Given $\sigma^q, [\sigma^q : \sigma^{q-1}]\ne 0$ only for a finite number of $(q-1)$-cells $\sigma^{q-1}$.\label[cond]{inc1}
	\item Given $\sigma^{q+1}$ and $\sigma^{q-1}, \sum_{\sigma^q}[\sigma^{q+1} : \sigma^{q}][\sigma^q : \sigma^{q-1}]=0$.\label[cond]{inc2}
\end{enumerate}
\Cref{inc1} indicates that we are confronted with an abstract complex of the closure finite type. Consequently we shall define
(in \cref{sec:hom-grp}) homology based on finite chains and cohomologies based on infinite cochains.

Our preferences for complexes \`{a} la W.~Mayer is due to the fact that they seem to be best adapted for the exposition of the homology
theory in terms of functors.

Given two abstract complexes $K_1$ and $K_2$, a chain transformation
\begin{equation*}
	k:K_1\rightarrow K_2
\end{equation*}
will mean a collection $k=\{k^q\}$ of homomorphisms,
\begin{equation*}
	k^q:C^q(K_1)\rightarrow C^q(K_2),
\end{equation*}
such that
\begin{equation*}
	k^{q-1}\partial^{q} = \partial^q k^q.
\end{equation*}

In this way we are led to the category $\mathbf{K}$ whose objects are the abstract complexes (in the sense of W.~Mayer) and whose
mappings are the chain transformations with obvious definition of the composition of chain transformations.

The consideration of simplicial complexes and of simplicial transformations leads to a category $\mathbf{K}_s$. As is well known,
every simplicial complex uniquely determines an abstract complex, and every simplicial transformation a chain transformation.
This leads to a covariant functor on $\mathbf{K}_s$ to $\mathbf{K}$.

\section{Homology and cohomology groups}\label{sec:hom-grp}
For every complex $K$ in the category $\mathbf{K}$ and every group $G$ in the category $\mathbf{G}_{0a}$ of discrete abelian groups
we define the groups $C^q(K,G)$ of the $q$-dimensional chains of $K$ over $G$ as the tensor product
\begin{equation*}
	C^q(K,G)= G\otimes C^q(K),
\end{equation*}
that is, $C^q(K,G)$ is the group with the symbols
\begin{equation*}
	g c^q,\text{ for } g\in G, c^q\in C^q(K)
\end{equation*}
as generators, and
\begin{equation*}
	(g_1 + g_2)c^q = g_1 c^q + g_2 c^q, \qquad g(c_1^q+c_2^q) = g_1 c^q + g_2 c^q
\end{equation*}
as relations.

For every chain transformation $k:K_1\rightarrow K_2$ and for every homomorphism $\gamma:G_1\rightarrow G_2$ we define a homomorphism
\begin{equation*}
	C^q(k,\gamma):C^q(K_1,G_1)\rightarrow C^q(K_2,G_2)
\end{equation*}
by setting
\begin{equation*}
	C^q(k,\gamma)(g_1 c_1^q)= \gamma(g_1)k^q(c_1^q)
\end{equation*}
for each generator $g_1 c_1^q \in C^q(K_1,G_1)$.

These definitions of $C^q(K,G)$ and of $C^q(k,\gamma)$ yield a functor $C^q$ covariant in $\mathbf{K}$ and in $\mathbf{G}_{0a}$ with
values in $\mathbf{G}_{0a}$. This functor will be called the $q$-chain functor.

We define an homomorphism
\begin{equation*}
	\partial^q(K,G):C^q(K,G)\rightarrow C^{q-1}(K,G)
\end{equation*}
by setting
\begin{equation*}
	\partial^q(K,G)(gc^q)= g\partial c^q
\end{equation*}
for each generator $gc^q$ of $C^q(K,G)$. Thus the boundary operator becomes a natural transformation of the functor $C^q$ into
the functor $C^{q-1}$
\begin{equation*}
	\partial^q:C^q\rightarrow C^{q-1}.
\end{equation*}
The kernel of this transformation will be denoted by $Z^q$ and will be called the $q$-cycle functor. Its object function is
the group $Z^q(K,G)$ of the $q$-dimensional cycles of the complex $K$ over $G$.

The image of $C^q$ under the transformation $\partial^q$ is a subfunctor $B^{q-1}=\partial^q(C^q)$ of $C^{q-1}$.
Its object function is the group $B^{q-1}(K,G)$ of the of the $(q-1)$-dimensional boundaries in $K$ over $G$.

The fact that $\partial^q\partial^{q+1}=0$ implies that $B^q(K,G)$ is a subgroup of $Z^q(K,G)$. Consequently $B^q$
is a subfunctor of $Z^q$. The quotient functor
\begin{equation*}
	H^q= Z^q/B^q
\end{equation*}
is called the $q$th homology functor. Its object function associates with each complex $K$ and with each discrete
abelian coefficient group $G$ the $q$th homology group $H^q(K,G)$ of $K$ over $G$. The functor $H^q$ is covariant
in $\mathbf{K}$ and $\mathbf{G}_{0a}$ and has values in $\mathbf{G}_{0a}$.

In order to define the cohomology groups as functors we consider the category $\mathbf{K}$ as before and the
category $\mathbf{G}_a$ of topological abelian groups. Given a complex $K\in\mathbf{K}$ and a group $G\in\mathbf{G}$
we define the group $C_q(K,G)$ of the $q$-dimensional cochains of $K$ over $G$ as
\begin{equation*}
	C_q(K,G)=\hm(C^q(K),G).
\end{equation*}
Given a chain transformation $k:K_1\rightarrow K_2$ and a homomorphism $\gamma:G_1\rightarrow G_2$ we define
a homomorphism
\begin{equation*}
	C_q(k,\gamma):C_q(K_2,G_1)\rightarrow C_q(K_1,G_2)
\end{equation*}
by associating with each homomorphism $f\in C_q(K_2,G_1)$ the homomorphism $\bar{f}=C_q(k,\gamma)f$, defined as
follows:
\begin{equation*}
	\bar{f}(c_1^q)= \gamma[f(k^q c_1^q)],\qquad c_1^q\in C^q(K_1).
\end{equation*}
By comparing this definition with the definition of the functor $\hm$, we observe that $C_q(k,\gamma)$ is in fact
just $\hm(k^q,\gamma)$.

The definition of $C_q(K,G)$ and $C_q(k,\gamma)$ yield a functor $C_q$ contravariant in $\mathbf{K}$, covariant in
$\mathbf{G}_a$, and with values in $\mathbf{G}_a$. This functor will be called the $q$th cochain functor.

The coboundary homomorphism
\begin{equation*}
	\delta_q(K,G):C_q(K,G)\rightarrow C_{q+1}(K,G)
\end{equation*}
is defined by setting, for each cochain $f\in C_q(K,G)$,
\begin{equation*}
	(\delta_q f)(c^{q+1})= f(\partial^{q+1} c^{q+1}).
\end{equation*}
This leads to a natural transformation of functors
\begin{equation*}
	\delta_q:C_q\rightarrow C_{q+1}.
\end{equation*}
We may observe that in terms of the functor ``$\hm$'' we have $\delta_q(K,G)=\hm(\partial^{q+1},e_G)$.

The kernel of the transformation $\delta_q$ is denoted by $Z_q$ and is called the $q$-cocycle functor. The image
functor of $\delta_q$ is denoted by $B_{q+1}$ and is called the $(q+1)$-coboundary functor. Since
$\partial^q\partial^{q+1}=0$, we may easily deduce that $B_q$ is a subfunctor of $Z_q$.
The quotient\hyp{}functor
\begin{equation*}
	H_q=Z_q/B_q
\end{equation*}
is, by definition, the $q$th cohomology functor. $H_q$ is contravariant in $\mathbf{K}$, covariant in
$\mathbf{G}_a$, and has values in $\mathbf{G}_a$. Its object function associates with each complex $K$
and each topological abelian group $G$ the (topological abelian) $q$th cohomology group $H_q(K,G)$.

The fact the the homology groups are discrete and have discrete coefficient groups, while the cohomology
groups are topologized and have topological coefficient groups, is due to the circumstance that
the complexes considered are closure finite. In a star finite complex the relation would be reversed.

For ``finite'' complexes both homology and cohomology groups may be topological. Let $\mathbf{K}_f$
denote the subcategory of $\mathbf{K}$ determined by all those complexes $K$ such that all the groups
$C^q(K)$ have finite rank. If $K\in \mathbf{K}_f$ and $G$ is a topological group, then the group
$C^q(K,G)=G\otimes C^q(K)$ can be topologized in a natural fashion and consequently $H^q(K,G)$ will
be topological. Hence both $H^q$ and $H_q$ may be regarded as functors on $\mathbf{K}_f$ and
$\mathbf{G}_a$ with values in $\mathbf{G}_a$. The first one is covariant in both  $\mathbf{K}_f$
and $\mathbf{G}_a$, while the second one is contravariant in $\mathbf{K}_f$ and covariant in
$\mathbf{G}_a$.

\section{Duality}\label{sec:duality}
Let $G$ be a discrete abelian group and $\chr(G)$ be its (compact) character group (see \cref{sec:char}).

Given a chain
\begin{gather*}
	c^q\in C^q(K,G)
	\shortintertext{where}
	c^q=\sum_i g_i c^q_i,\quad g_i\in G, c^q_i\in C^q(K),
\end{gather*}
and given a cochain 
\begin{equation*}
	f\in C_q(K,\chr G),
\end{equation*}
we may define the Kronecker index
\begin{equation}
	\ki(f,c^q)=\sum_i(f(c^q_i),g_i).
\end{equation}
Since $(c^q_i)$ is an element of $\chr G$, its application to $g_i$ gives an element of the group $P$
of reals reduced$\mod{1}$. The continuity of $\ki(f,c^q)$ as a function of $f$ follows from the definition
of the topology in $\chr G$ and in $C_q(K,\chr G)$.

As a preliminary to the duality theorem, we define an isomorphism
\begin{equation}\label{eq:dual_iso}
	\tau^q(K,G):C_q(K,\chr G)\rightleftarrows \chr C^q(K,G),
\end{equation}
by defining for each cochain $f\in C_q(K,\chr G)$ a character
\begin{gather*}
	\tau^q(K,G)f:C^q(K,G)\rightarrow P
	\shortintertext{as follows:}
	(\tau^q f,c^q)=\ki(f,c^q).
\end{gather*}

The fact that $\tau^q(K,G)$ is an isomorphism is a direct consequence of the character theory. In \cref{eq:dual_iso}
both sides should be interpreted as object functions of functors (contravariant in both $K$ and $G$), suitably
compounded from the functors $C^q,C_q$ and $\chr$. In order to prove that \cref{eq:dual_iso} is natural, 
consider
\begin{equation*}
	k:K_1\rightarrow K_2\in\mathbf{K},\qquad\gamma:G_1\rightarrow G_2\in\mathbf{G}_{0a}.
\end{equation*}
We must prove that
\begin{equation}\label{eq:dual_nat}
	\tau^q(K_1,G_1)C_q(k,\chr\gamma)=[\chr C^q(k,\gamma)]\tau^q(K_2,G_2).
\end{equation}
If now
\begin{equation*}
	f\in C_q(K,_2G_2),\qquad c^q\in C^q(K_1,G_1),
\end{equation*}
then the definition of $\tau^q$ shows that \cref{eq:dual_nat} is equivalent to the identity
\begin{equation}\label{eq:dual_eq}
	\ki(C_q(k,\chr\gamma)f,c^q) = \ki(f,C^q(k,\gamma)c^q).
\end{equation}
It will be sufficient to establish \cref{eq:dual_eq} in the case when $c^q$ is a generator of $C^q(K_1,G_1)$,
\begin{equation*}
	c^q=g_1c^q_1,\qquad g_1\in G_1, c^q_1\in C^q(K_2).
\end{equation*}
\begin{proof}
	Using the definitions of the terms involved in \cref{eq:dual_eq} we have on the one hand
	\begin{align*}
	\ki(C_q(k,\chr\gamma)f,g_1c^q_1) &= ([C_q(k,\chr\gamma)f]c^q_1,g_1)\\
				&=(\chr\gamma[f(k c^q_1)]g_1\\ &=(f(k c^q_1,\gamma g_1),
	\end{align*}
	and on the other hand
	\begin{equation*}
		\ki(f,C^q(k,\gamma)g_1 c^q_1) = \ki(f,(\gamma g_1)(k c^q_1)) = (f(k c^q_1,\gamma g_1).
	\end{equation*}
	This completes the proof of the naturality of \cref{eq:dual_eq}.
\end{proof}

Using the well known property of the Kronecker index
\begin{equation*}
	\ki(f,\partial^{q+1} c^{q+1}) = \ki(\delta_q f,c^{q+1}),
\end{equation*}
one shows easily that under the isomorphisms $\tau^q$ of \cref{eq:dual_iso}
\begin{align*}
	\tau^q[Z_q(K,\chr G)]&= \an B^q(K,G),\\
	\tau^q[B_q(K,\chr G)]&= \an Z^q(K,G)
\end{align*}
with ``$\an$'' defined as in \cref{sec:char}. Both $\an (B^q;C^q)$ and $\an (Z^q;C^q)$ are functors
covariant in $K$ and $G$; the latter is a subfunctor of the former, so that $\tau^q$ induces a natural
isomorphism
\begin{equation*}
	\sigma^q:Z_q(K,\chr G)/B_q(K,\chr G)\rightleftarrows\an B^q(K,G)/\an Z^q(K,G).
\end{equation*}
The group on the left is $H_q(K,\chr G)$. The group on the right is, according to \cref{eq:ann_comp}, 
naturally isomorphic to $\chr Z^q(K,G)/B^q(K,G)$. All told we have a natural isomorphism:
\begin{equation*}
	\rho^q:H_q(K,\chr G)\rightleftarrows\chr H^q(K,G).
\end{equation*}

This is the customary Pontrjagin\hyp{}type duality between homology and cohomology. Thus we have 
established the naturality of this duality.

\section{Universal coefficient theorems}\label{sec:uni-coeff}
The theorems of this name express the cohomology groups of a complex, for an arbitrary coefficient
group, in terms of the integral homology groups and the coefficient group itself. A quite general
form of such theorems can be stated in terms of certain groups of group extensions;\footcite[757--831]{eilenberg42}
hence we first show that the basic constructions of group extensions may be regarded as functors.

Let $G$ be a topological abelian group and $H$ a discrete abelian group. A factor set of $H$ in $G$ is a function
$f(h,k)$ which assigns to each pair $(h,k)$ of elements in $H$ an element $f(h,k)\in G$ in such wise that
\begin{equation*}
	f(h,k)=f(k,h),\qquad f(h,k)+f(h+k,l)=f(h,k+l)+f(k,l)
\end{equation*}
for all $h$,$k$, and $l$ in $H$. With the natural addition and topology, the set of all factor set $f$ of $H$
in $G$ constitute a topological abelian group $\fc(G,H)$. If $\gamma:G_1\rightarrow G_2$ and $\eta:H_1\rightarrow H_2$
are homomorphisms, we can defined a corresponding mapping
\begin{equation*}
	\fc(\gamma,\eta):\fc(G_1,H_2)\rightarrow\fc(G_2,H_1)
\end{equation*}
by setting
\begin{equation*}
	[\fc(\gamma,\eta)f](h_1,k_1)= \gamma f(\eta h_1,\eta k_1)
\end{equation*}
for each factor set $f\in\fc(G_1,H_2)$. Thus it appears that $\fc$ is a functor, covariant on the
category $\mathbf{G}_a$ of topological abelian groups and contravariant in the category $\mathbf{G}_{0a}$
of discrete abelian groups.

Given any function $g(h)$ with values in $G$, the combination
\begin{equation*}
	f(h,k)= g(h)+g(k)-g(h+k)
\end{equation*}
is always a factor set; the factor sets of this special form are said to be transformation sets,
and the set of all transformation sets is a subgroup $\tra(G,H)$ of the group $\fc(G,H)$.
Furthermore, this subgroup is the object function of a subfunctor. The corresponding quotient
functor
\begin{equation*}
	\ex=\fc/\tra
\end{equation*}
is thus covariant in $\mathbf{G}_a$, contravariant in $\mathbf{G}_{0a}$, and has values in $\mathbf{G}_a$.
Its object function assigns to the groups $G$ and $H$ the group $\ex(G,H)$ the so\hyp{}called abelian
group extension of $G$ by $H$.

Since $C_q(K,G)=\hm(C^q(K),G)$ and since $C^q(K,\mathbb{Z})= \mathbb{Z}\otimes C^q(K)=C^q(K)$ where 
$\mathbb{Z}$ is the additive group of integers, we have
\begin{equation*}
	C_q(K,G)=\hm(C^q(K,\mathbb{Z}),G).
\end{equation*}
We, therefore, may define a subgroup
\begin{equation*}
	A_q(K,G)=\an Z^q(K,\mathbb{Z})
\end{equation*}
of $C_q(K,G)$ consisting of all homomorphisms $f$ such that $f(z^q)=0$ for $z^q\in Z^q(K,\mathbb{Z})$.
Thus we get a subfunctor $A_q$ of $C_q$, and one may show that the coboundary functor $B_q$ is a subfunctor
of $A_q$ which, in turn, is a subfunctor of the cocycle functor $Z_q$. Consequently, the quotient functor
\begin{equation*}
	Q_q= A_q/B_q
\end{equation*}
is a subfunctor of the cohomology functor $H_q$, and we may consider the quotient functor $H_q/Q_q$.
The functors $Q_q$ and $H_q/Q_q$ have the following object functions
\begin{align*}
	Q_q(K,G) &= A_q(K,G)/B_q(K,G),\\
	(H_q/Q_q)(K,G) &= H_q(K,G)/Q_q(K,G)\cong Z_q(K,G)/A_q(K,G).
\end{align*}

The universal coefficient theorem now consists of these three assertions:\footcite[808]{eilenberg42}
\begin{gather}
	Q_q(K,G)\text{ is a direct factor of } H_q(K,G).\label{eq:uni-coeff1}\\
	Q_q(K,G)\cong\ex(G,H^{q+1}(K,\mathbb{Z})).\label{eq:uni-coeff2}\\
	H_q(K,G)/Q_q(K,G)\cong\hm(H^q(K,\mathbb{Z}),G).\label{eq:uni-coeff3}
\end{gather}

Both the isomorphisms \labelcref{eq:uni-coeff2,eq:uni-coeff3} can be interpreted as equivalences of functors.
The naturality of these equivalences with respect to $K$ has been explicitly verified,\footcite[815]{eilenberg42}
while the naturality with respect to $G$ ca be verified without difficulty. We have not been able to prove and
we doubt that the functor $Q_q$ is a direct factor of the functor $H_q$ (see \cref{sec:funct_prod}).

\section{\u{C}ech homology groups}\label{sec:cech-hom}
We shall present now a treatment of the \u{C}ech homology in terms of functors.

By a covering $U$ of a topological space $X$ we shall understand a finite collection:
\begin{equation*}
	U = \{A_1,\dotsc,A_n\}
\end{equation*}
of open sets whose union is $X$. The sets $A_i$ may appear with repetitions, and some of
them may be empty. If $U_1$ and $U_2$ are two such coverings, we write $U_1\prec U_2$
whenever $U_2$ is a refinement of $U_1$, that is, whenever each set of the covering $U_2$
is contained in some set of the covering $U_1$. With this definition the coverings $U$ of
$X$ form a directed set which we denote by $C(X)$.

Let $\xi:X_1\rightarrow X_2$ be a continuous mapping of the space $X_1$ into the space
$X_2$. Given a covering
\begin{equation*}
	U= \{A_1,\dotsc,A_n\}\in C(X_2),
\end{equation*}
we define
\begin{equation*}
	C(\xi)U= \{\xi^{-1}(A_1),\dotsc,\xi^{-1}(A_n)\}\in C(X_1)
\end{equation*}
and we obtain an order preserving mapping
\begin{equation*}
	C(\xi): C(X_2)\rightarrow C(X_1).
\end{equation*}
We verify that the functions $C(X), C(\xi)$ define a contravariant functor $C$ on the category
$\mathbf{X}$ of topological spaces to the category $\mathbf{D}$ of directed sets.

Given a covering $U$ of $X$ we define, in the usual fashion, the nerve $N(U)$ of $U$. $N(U)$ is
a finite simplicial complex; it will be treated, however, as an object of the category $K_f$ of
\cref{sec:hom-grp}.

If two coverings $U_1\prec U_2$ of $X$ are given, then we select for each set of the covering
$U_2$ a set of the covering $U_1$ containing it. This leads to a simplicial mapping of the
complex $N(U_2)$ into the complex $N(U_1)$ and therefore gives a chain transformation
\begin{equation*}
	k:N(U_2)\rightarrow N(U_1).
\end{equation*}
This transformation $k$ will be called a projection. The projection $k$ is not defined uniquely
by $U_1$ and $U_2$, but it is known that any two projections $k_1$ and $k_2$ are chain homotopic
and consequently the induced homomorphisms
\begin{align}
	H^q(k,e_G) &: H^q(N(U_2),G)\rightarrow H^q(N(U_1),G),\label{eq:cech-hom}\\
	H_q(k,e_G) &: H_q(N(U_1),G)\rightarrow H_q(N(U_2),G)\label{eq:cech-cohom}
\end{align}
of the homology and cohomology groups do not depend upon the particular choice of the projection
$k$.

Given a topological group $G$ we consider the collection of the homology groups $H^q(N(U),G)$
for $U\in C(X)$. These groups together with the mappings \cref{eq:cech-hom} form an inverse
system of groups defined on the directed set $C(X)$. We denote the inverse system by $\overline{C}^q(X,G)$
and treat it as an object of the category \textbf{Inv} (\cref{sec:dir-inv}).

Similarly, for a discrete $G$ the cohomology groups $H_q(N(U),G)$ together with the mappings 
\labelcref{eq:cech-cohom} form a direct system of groups $\overline{C}_q(X,G)$ likewise
defined on the directed set $C(X)$. The system $\overline{C}_q(X,G)$ will be treated as an 
object of the category \textbf{Dir}.

The functions $\overline{C}^q(X,G)$ and $\overline{C}_q(X,G)$ will be object functions of 
functors $\overline{C}^q$ and $\overline{C}_q$. In order to complete the definition we shall
define the mapping functions  $\overline{C}^q(\xi,\gamma)$ and $\overline{C}_q(\xi,\gamma)$ for
given mappings
\begin{equation*}
	\xi:X_1\rightarrow X_2,\qquad \gamma:G_1\rightarrow G_2.
\end{equation*}
We have the order preserving mapping
\begin{equation}\label{eq:cech-map1}
	C(\xi):C(X_2)\rightarrow C(X_1)
\end{equation}
which with each covering
\begin{equation*}
	U = \{A_1,\dotsc,A_n\}\in C(X_2)
\end{equation*}
associates the covering
\begin{equation*}
	V = C(\xi) = \{\xi^{-1}A_1,\dotsc,\xi^{-1}A_n\}\in C(X_1).
\end{equation*}
Thus to each set of the covering $V$ corresponds uniquely a set of covering $U$; this yields a 
simplicial mapping
\begin{equation*}
	k:N(V)\rightarrow N(U),
\end{equation*}
which leads to the homomorphisms
\begin{align}
	H^q(k,\gamma) &: H^q(N(V),G_1)\rightarrow H^q(N(U),G_2),\label{eq:cech-map2}\\
	H_q(k,\gamma) &: H_q(N(U),G_1)\rightarrow H_q(N(V),G_2)\label{eq:cech-map3}
\end{align}
\Crefrange{eq:cech-map1}{eq:cech-map3} define the transformations
\begin{align*}
	\overline{C}^q(\xi,\gamma) &:\overline{C}^q(X_1,G_1)\rightarrow \overline{C}^q(X_2,G_2)\in\mathbf{Inv},\\
	\overline{C}_q(\xi,\gamma) &:\overline{C}_q(X_2,G_1)\rightarrow \overline{C}_q(X_1,G_2)\in\mathbf{Dir}.
\end{align*}
Hence we see that $\overline{C}^q$ is a functor covariant in $\mathbf{X}$ and in $\mathbf{G}_a$ with values in \textbf{Inv}
while $\overline{C}_q$ is contravariant in $\mathbf{X}$, covariant in $\mathbf{G}_{0a}$ with values in \textbf{Dir}.

The \u{C}ech homology and cohomology functors are now defined as
\begin{equation*}
	\overline{H}^q=\Lim_{\leftarrow}\overline{C}^q,\qquad \overline{H}_q=\Lim_{\rightarrow}\overline{C}_q.
\end{equation*}
$\overline{H}^q$ is covariant in $\mathbf{X}$ and in $\mathbf{G}_a$ with values in $\mathbf{G}_a$, while $\overline{H}_q$
is contravariant in $\mathbf{X}$, covariant in $\mathbf{G}_{0a}$, and has values in $\mathbf{G}_{0a}$. The object functions
$\overline{H}^q(X,G)$ and $\overline{H}_q(X,G)$ are the \u{C}ech homology and cohomology groups of the space $X$ with the
group $G$ as coefficients.

\section{Miscellaneous remarks}\label{sec:misc}
The process of setting up the various topological invariants as functors will require the construction of many categories.
For instance, if we wish to discuss the so\hyp{}called relative homology theory, we shall need the category $\mathbf{X}_S$
whose objects are the pairs $(X,A)$, where $X$ is a topological space and $A$ is a subset of $X$. A mapping
\begin{equation*}
	\xi:(X,A)\rightarrow (Y,B)\in\mathbf{X}_S
\end{equation*}
is a continuous mapping $\xi:X\rightarrow Y$ such that $\xi(A)\subset B$. The category $\mathbf{X}$ may be regarded
as the subcategory of $\mathbf{X}_S$ in the category $\mathbf{X}_b$ defined by the pairs $(X,A)$ in which the sets $A$
consists of a single point, called the base point. This category $\mathbf{X}_b$ would be used in a functorial
treatment of the fundamental group and of the homotopy groups.

\addchap{Appendix}
\addsec{Representations of Categories}\label{sec:rep-cat}
The purpose of this appendix is to show that every category is isomorphic with a suitable subcategory of the
category of sets $\mathbf{S}$.

Let $\mathbf{A}$ be any category. A covariant functor $T$ on $\mathbf{A}$ with values in $\mathbf{S}$ will be
called a representation of $\mathbf{A}$ in $\mathbf{S}$. A representation $T$ will be called \emph{faithful}
if for every two mappings, $\alpha_1,\alpha_2\in\mathbf{A}$, we have $T(\alpha_1)=T(\alpha_2)$ only if
$\alpha_1=\alpha_2$. This implies a similar proposition for the objects of $\mathbf{A}$. It is clear that a
faithful representation is nothing but an isomorphic mappings of $\mathbf{A}$ onto some subcategory of $\mathbf{S}$.

If the functor $T$ on $\mathbf{A}$ to $\mathbf{S}$ is contravariant, we shall say that $T$ is a dual representation.
$T$ is then obviously a representation of the dual category $\mathbf{A}^\text{op}$, as defined in \cref{sec:func_comb}.

Given a mapping $\alpha:A_1\rightarrow A_2$ in $\mathbf{A}$, we shall denote the domain $A_1$ of $\alpha$ by $d(\alpha)$
and the range $A_2$ of $\alpha$ by $r(\alpha)$. In this fashion we have
\begin{equation*}
	\alpha:d(a)\rightarrow r(a).
\end{equation*}

Given an object $A\in\mathbf{A}$ we shall denote by $R(A)$ the set of $\alpha\in\mathbf{A}$, such that $A=r(\alpha)$.
In symbols
\begin{equation}\label{eq:rep_dom}
	R(A)= \{\alpha | \alpha\in\mathbf{A}, r(\alpha)=A\}.
\end{equation}
For every mapping $\alpha\in\mathbf{A}$ we defined a mapping
\begin{equation}\label{eq:rep_map}
	R(\alpha):R(d(\alpha))\rightarrow R(r(\alpha))
\end{equation}
in the category $\mathbf{S}$ by setting
\begin{equation}\label{eq:rep_nat}
	[R(\alpha)]\xi=\alpha\xi,\qquad \forall\xi\in R(d(\alpha)).
\end{equation}
This mapping is well defined because if $\xi\in R(d(\alpha))$, then $r(\xi)=d(\alpha)$, so that $\alpha\xi$ is defined
and $r(\alpha\xi)=r(\alpha)$ which implies $\alpha\xi\in R(r(\alpha))$.
\begin{theorem}
	For every category $\mathbf{A}$ the pair of functions $R(A),R(\alpha)$, defined above, establishes a faithful
	representation $R$ of $\mathbf{A}$ in $\mathbf{S}$.
\end{theorem}
\begin{proof}
	We first verify that $R$ is a functor. If $\alpha=e_A$ is an identity then the definition \labelcref{eq:rep_nat}
	implies that $[R(\alpha)]\xi=\xi$, so that $R(\alpha)$ is the identity mapping of $R(A)$ into itself. Thus
	$R$ satisfies \cref{eq:funct_id}. \Cref{eq:funct_prod1} has already been verified. In order to verify
	\cref{eq:funct_prod2} let us consider the mappings
	\begin{equation*}
		\alpha_1:A_1\rightarrow A_2\qquad\alpha_2:A_2\rightarrow A_3.
	\end{equation*}
	We have for every $\xi\in R(A_1)$,
	\begin{equation*}
		[R(\alpha_2\alpha_1)]\xi=\alpha_2\alpha_1\xi=[R(\alpha_2)]\alpha_1\xi=[R(\alpha_2)R(\alpha_1)]\xi,
	\end{equation*}
	so that $R(\alpha_2\alpha_1)=R(\alpha_2)R(\alpha_1)$. This concludes the proof that $R$ is a representation.

	In order to show that $R$ is faithful, let us consider two mappings $\alpha_1,\alpha_2\in\mathbf{A}$ and let
	assume that $R(\alpha_1)=R(\alpha_2)$. It follows from \cref{eq:rep_map} that $R(d(\alpha_1))=R(d(\alpha_2))$,
	and, therefore, according to \cref{eq:rep_dom}, $d(\alpha_1)=d(\alpha_2)$. Consider the identity mapping
	$e=e_{d(\alpha_1)}=e_{d(\alpha_2)}$. Following \cref{eq:rep_nat} we have
	\begin{equation*}
		\alpha_1 =\alpha_1 e = [R(\alpha_1)]e = [R(\alpha_2)]e = \alpha_2 e = \alpha_2,
	\end{equation*}
	so that $\alpha_1=\alpha_2$. This concludes the proof of the theorem.
\end{proof}

In a similar fashion we could define a faithful dual representation $D$ of $\mathbf{A}$ by setting
\begin{equation*}
	D(A)= \{\alpha | \alpha\in\mathbf{A}, d(\alpha)=A\}.
\end{equation*}
and
\begin{equation*}
	[D(\alpha)]\xi= \xi\alpha
\end{equation*}
for every $\xi\in D(r(\alpha))$.

The representations $R$ and $D$ are the analogues of the left and right regular representations in group theory.

We shall conclude with some remarks concerning partial order in categories. Most of the categories which we have
considered have an intrinsic partial order. For instance, in the categories $\mathbf{S,X}$, and $\mathbf{G}$ the
concept of subset,subspace and subgroup furnish a partial order. In view of \cref{eq:rep_dom}, $A_1\ne A_2$ implies
that $R(A_1)$ and $R(A_2)$ are disjoint, so that the representation $R$ destroys this order completely. The problem
of getting ``order preserving representations'' would require probably a suitable formalization of the concept of
a partially ordered category.

As an illustration of the type of arguments which may be involved, let us consider the category $\mathbf{G}_0$ of
discrete groups. With each group $G$ we can associate the set $R_1(G)$ which is the set of elements constituting
the group $G$. With the obvious mappings function, $R_1$ becomes a covariant functor on $\mathbf{G}_0$ to
$\mathbf{S}$, that is, $R_1$ is a representation of $\mathbf{G}_0$ in $\mathbf{S}$. This representation is not
faithful, since the same set may carry two different group structures. The group structure of $G$ is entirely
described by means of a ternary relation $g_1 g_2 = g$. This ternary relation is nothing by a subset
$R_2(G)\subset R_1(G)\times R_1(G)\times R_1(G)$. All of the axioms of group theory can be formulated in terms
of the subset $R_2(G)$. Moreover a homomorphism $\gamma:G_1\rightarrow G_2$ includes a mapping $R_2(\gamma):
R_2(G_1)\rightarrow R_2(G_2)$. Consequently $R_2$ is a subfunctor of a suitably defined functor
$R_1\times R_1\times R_1$. The two functors $R_1$ and $R_2$ together give a complete description of
$\mathbf{G}_0$, preserving the partial order.
\printbibliography
\end{document}
